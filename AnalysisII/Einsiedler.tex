\documentclass{article}
\title{Analysis II Einsiedler Version}
\author{Benjamin Dropmann}
\usepackage{geometry}
\usepackage{tikzit}
\input{AnalysisStyle.tikzstyles}
\usepackage{xcolor}
\usepackage{amssymb}
\usepackage{amsfonts}
\usepackage{amsmath}
\usepackage{multicol}
\usepackage{titlesec}
\titlespacing*{\subsubsection}{0pt}{1.2ex}{.1ex plus .2ex minus .2ex}

\newcommand{\mspc}{\hspace{0.7cm}}
\newcommand{\smspc}{\hspace{0.3cm}}

\newcommand{\kk}[1]{\left<\left<{#1}\right>\right>}
\newcommand{\dd}[1]{\hspace{0.2cm}\text{d{#1}}}

\newcommand{\satz}[1]{\subsubsection*{Satz {#1}}}
\newcommand{\korollar}[1]{\subsubsection*{Korollar {#1}}}
\newcommand{\beweis}{\\\textbf{Beweis }}
\newcommand{\beispiel}[1]{\subsubsection*{Beispiele {#1}}}
\newcommand{\bemerkung}[1]{\subsubsection*{Bemerkung {#1}}}
\newcommand{\theorem}[1]{\subsubsection*{Theorem {#1}}}
\newcommand{\lemma}[1]{\subsubsection*{Lemma {#1}}}
\newcommand{\definition}[1]{\subsubsection*{Definition {#1}}}
\newcommand{\behauptung}[1]{\subsubsection*{Behauptung {#1}}}

\geometry{margin=1.5cm}
\begin{document}
\maketitle
\section{Wiederholung}
\definition{Taylor-Polynom} Sei eine funktion $f:(a,b)\rightarrow\mathbb{C}$ um einen punkt $x_0\in (a,b)$ die $n$-mal differenzierbar ist. Dann ist der Polynom 
\[p_{x_0,n}^f(x)=\sin_{k=0}^n\frac{f^{(k)}(x_0)}{k!}(x-x_0)\]
\satz{Taylor Approximation} sei $f:(a,b)\rightarrow\mathbb{C}$ eine $n+1$ mal stetig differenzierbare funktion und $x_0,x\in (a,b)$ dann gilt\[f(x)=P_{x_0,n}^f(x)+R_{x_0,n}^f(x)\]
Wobei diese $R_{x_0,n}^f(x)$ der restglied ist und ich den mit \[R_{x_0,n}^f(x)=\int_{x_0}^xf^{(x+1)}(t)\frac{(x-t)^n}{n!}\dd{t}\]
Für $M_{n+1}=\max\lbrace |f^{(n+1)}(t)|$ mit $t\in (x,x_0)$ gilt $|R_{x_0,n}^f(x)|\le\frac{M_{n+1}|x-x_0|}{(n+1)!}$
\definition{} Eine Funktion $f(a,b)\rightarrow\mathbb{C}$ heisst analytisch falls es zu jedem $x_0\in (a,b)$ $\exists R>0$ so dass \[f(x)=\sum_{n=0}^\infty \frac{f^{(n)}(x_0)}{n!}(x-x_=)!\smspc\forall x:|x-x_0|<R\]
\subsection*{8.6 Numerische Integration}
\satz{} Sei $f:[a,b]\rightarrow\mathbb{R}$ ,ot $n\in \mathbb{N}$ und $h=\frac{b-a}{h}$, $x_e=a+lh$ für  $l\in\lbrace  0,1,\cdots,n\rbrace$  Falls $f$ stetig differenzierbar  ist dann gilt \[\int_a^b f(t)\dd{t}=h\cdot(f(x_0)+f(x_1)+\cdots+f(x_{n-1}))+F_1\] Wobei $F_1$ unser fehler ist: $F_1\le \frac{M_1(b-1)^2}{2n}$
mit $M_1=\max\lbrace|f'(x)|:x\in[a,b]\rbrace$
Falls die Funktion $2$-mal stetig differenzierbar ist dann gilt: 
\[\int_a^bf(t)\dd{t}=\frac{h}{2}\left(f(x_0)+2f(x_1)+\cdots+2f(x_{n-1}))+f(x_n)\right)+F_2\mspc F_2=\frac{M_2(b-a)^3}{6n^2}\]
Man kann auch für $f$ vier mal differenzierbar und $n$ gerade eine solche abschätzung machen, dies ist die Simpson Regel:
\[\int_a^b f(t)\dd{t}=\frac{h}{3}\left(f(x_0)+4f(x_1)+2f(x_2)+4f(x_3)+\cdots+4f(x_{n-1})+f(x_n)\right)+F_4\mspc F_4\le\frac{M_4(b-1)^5}{45n^4}\]
\section*{9. Metrische Räume}
\subsection*{9.1 Konvergenz in Metrische Räume}
\definition{Metrischer Raum} Ein Metrischer Raum ist eine Menge $X$ mit eine Abbildung $d:X\times X\rightarrow \mathbb{R}_{\ge 0}$ Mit folgende Eigenschaften
\begin{itemize}
  \item[i.]{\textbf{Definitheit} $\forall x,y\in X$ gilt $d(x,y)=0\Leftrightarrow x_y$}
  \item[ii.]{\textbf{Symmetrie} $d(x,y)=d(y,x)\smspc\forall x,y\in X$}
  \item[iii.]{\textbf{Dreiecksungleichung} $\forall x,y,z\in X$ gilt $d(x,z)\le d(x,y)+d(y,z)$}
\end{itemize}
Das $d$ hier ist einfach eine Abstandsfunktion, die einen Abstand zwischen zwei elemente der Menge anteilt. Diese Abbildung ist die Metrik.
$X$ kann $\mathbb{R}$, $\mathbb{C}$ oder sogar $\mathbb{R}^n$ sein, die Metrik $d$ kann irgendeine Funktion sein:
\begin{itemize}
  \item{$d(x,y)=|x-y|$ bei $X=\mathbb{R}$ oder $\mathbb{C}$}
  \item{$d_2(x,y)=||x-y||_2=\sqrt{\sum_{j=1}^d (x_j-y_j)^2}$}
  \item{$d_1(x,y)=||x,y||_1=\sum_{j=1}^d|x_j-y_j|$}
  \item{$d_\infty=||x-y||_\infty=max_i|x_i-y_i|$}
\end{itemize}
\definition{Norm} Ein Normierter, Reeller Vektorraum ist ein Vektorraum $V$ uber $\mathbb{R}$ gemeinsam mit eine Abbildung $||\cdot||:V\rightarrow R_{\ge0}$ mit fonlgende Eigenschaften
\begin{itemize}
  \item{\textbf{Definitheit} $||v||=0\Leftrightarrow v=0$}
  \item{\textbf{Homogeneitat} $||tv||=|t|\cdot||v||$}
  \item{\textbf{Dreiecksungleichung} $||u+v||\le||u||+||v||$}
\end{itemize}
Diese Abbildung $||\cdot||$ wird norm genannt wenn diese Axiome $\forall v,u\in V$ gelten
\lemma{}Eine Norm auf $V$ definiert eine Metrik $d(v,w)=||v-w||$ \beweis Die Axiome der Norm passen mit dieser Definition mit den Axiomen der Metrik.
\beispiel{Paris/SNCF-Metrik}
Diese Metrik ist auf $X=\mathbb{C}$ definiert, Die Metrik bekommt ihr namen vom Fakt dass es in Frankreich mit der bahn, sehr leicht ist von irgendwo nach Paris hinzukommen, und von Paris aus irgendwo anders zu gehen.
Die Metrik ist also \[d(z,w)=\left\lbrace\begin{matrix}|z|+|w|&\text{Wenn }\not\exists \lambda\smspc w=\lambda z\\|z-w|&\text{Wenn} \exists \lambda \smspc w=\lambda z\end{matrix}\right.\]
Diese Metrik ist sehr Komisch aber respektiert immer doch die Axiome. Die Konvergenz, die bald definiert wird, hat Komische Konvergenz da sogar wenn punkte optisch sehr nah miteinander aussehen, sind die wegen der Metrik doch nicht. Deswegen, nehmen wir öftestens Normen auf Teilmengen als Metriken.
\definition{Konvergenz in einem Metrischen Raum} Sei $X$ ein Metrischer Raum, und $x_n$ eine Folge in $X$ und $z\in X$ Wir sagen dass $x_n$ gegen $z$ konvergiert und schreiben $\lim_n\rightarrow\infty x_n=z$ falls \[\forall \varepsilon>0 \exists N\smspc \forall n>N: \smspc d(x_n,z)<\varepsilon\]
\definition{Offene Ball} Sei $X$ ein Metrischer Raum. Der Offene Ball um $x_0\in X$ mit radius $r>0$ ist durch \[B_r(x_0)=\left\lbrace \left. x\in X\right|d(x,x_0)<r\right\rbrace\] definiert. Eine Menge $U\subset X$ heisst umgebung von $x_0$ falls es ein $\exists r>0$ so dass $B_r(x_0)=\subseteq U$. Mann kann auch dieser Bälle benutzen um die Konvergenz zu definieren. 
\beispiel{Der Raum der Stetigen Funktionen} Sei $a<b\in\mathbb{R}$ Wir definieren $V=C([a,b])$ ein Vektorraum. Wir definieren dann die Norm $||f||_\infty=max_{x\in[a,b]}|f(x)|$ Für eine Stetige funktion $f\in V$ Dann ist $V$ ein Reeller Vektorraum, $||\cdot||$ ist eine Norm und die Konvergenz von $f_n\in V$ gegen $f\in V$ ist gleichbedeutend zur gleichmässige Konvergenz 
\beweis Angenommen $f_n\in V$ Konvergiert für $n\rightarrow \infty$ gegen $f \in V$. Diese aussage ist analog zu:
\[\forall \varepsilon >0\smspc \exists N\in \mathbb{N} \text{ so dass } \forall n>N \smspc d(f_n,f)=||f_n-f||_\infty<\varepsilon\]
Die unendlich norm hier ist definiert wie $\max_{[a,b]}(f_n-f)$ Also gilt $\forall x\in [a,b]$\[|f_n(x)-f(x)|\le||f_n-f||_\infty<\varepsilon\] Doch den $N$ haben wir vor den $x$ gewählt, daher ist dies auch gliechmässig konvergent.
\[\forall \varepsilon< 0\smspc \exists N\in \mathbb{N}\smspc \forall n< N\smspc\forall x\in[a,b]\text{ gilt } |f_n(x)-f(x)|\le\varepsilon\]
\bemerkung{} $||\cdot||_\infty$ ist eine Norm:
\begin{itemize}
  \item[i]{$||f||_\infty=0\Longleftrightarrow f=0$}
  \item[ii]{$||\lambda f||_\infty=max_{[a,b]}|s\cdot f(x)|=|s|\cdot max_{[a,b]}|f(x)|=|s|\cdot||f||_\infty$}
  \item[iii]{$f_1,f_2\in V$ dann ist $||f_1+f_2||_\infty=max_{[a,b]}\underset{=|f_1(x)|+|f_2(x)|}{\underbrace{|f_1(x)+f_2(x)|}}\le||f_1||_\infty+||f_2||_\infty$ (dies ist nicht klar aber es beweist die Dreiecksungleichung)}
\end{itemize}
\subsubsection*{Wichtige Normen und Metriken}
\begin{itemize}
  \item{Euklidische Norm: Die Vom Standard Inneres Produkt\[||v||=<v,v>=\sqrt{\sum_{i=1}^n v_i^2}\]}
  \item{Manhatten Norm: Grid city-norm: Sei $v=(v_1,v_2)$ dann ist die Manhatten norm durch: $||(v_1,v_2)||=|v_1|+v_2|$ definiert}
  \item{Supremum norm: $||f||_\infty=max_{x\in[a,b]}|f(x)$}
\end{itemize}
Analog zu jeder Norm, gibt es auch die dazugehörige Metrik.
\subsection{Topologische grundbegriffe}
\definition{} Sei $(X,d)$ ein Metrischer Raum. Eine Teilmenge $o\subset X$ heisst offen falls gilt: \[\forall x_0\in o\smspc \exists \varepsilon >0\text{ so dass } B_\varepsilon \subset o\]Eine Teilmenge $A\subseteq X$ heisst abgeschlossen falls $X\backslash A$ offen ist.
\beispiel{} Der Offene Ball ist wirklich offen:Sei ein Offenes Ball $B_r(x_0)\subseteq X$ dann gilt $\forall x\in B_r(x_0)$ $d(x,x_0)<r\longrightarrow \varepsilon =r-d(x,x_0)$ und dann ist $B_\varepsilon(x)\subseteq B_r(x_0)$ wass die vorherigfe definition anpasst.
\beispiel{} $\lbrace x_0\rbrace\subseteq X$ ist eine Abgeschlossene Menge: Wir mussen also zeigen dass $O=X\backslash \lbrace x_0 \rbrace$ offen ist. Sei $x\in O$ und $\varepsilon = d(x,x_0)>0$ dann gilt $B_\varepsilon(x)\subseteq O$
\lemma{} Sei $X$ eine Metrischer Ruam. \begin{itemize}\item{Jeder Endliche Schnitt von offenen Mengen ist offen.}\item{Jede Beliebige vereinigung von Offenen Mengen ist Offen und} \item{Jede Endliche Vereinigung von abgeschlossenen Mengen ist abgeschlossen.}\item{Jeder Belibieger Durchnitt von abgeschlossenen Mengen ist abgeschlossen.}\end{itemize}
\lemma{} Sei $X$ ein Metrischer Raum. Eine Teilmenge $O\in X$ ist Offen genau dann wenn fur jede Konvergente Folge $(x_n)\in X$ mit grenzwert $\lim_n{\rightarrow \infty} x_n\in O$  auch gilt dass fur alle bis auf endlich viele $n$ wir $x_n\in O$ haben.
\beweis Angenommen wir haben $O\subseteq X$ eine Offene Teilmenge und $x_n$ in $X$ mit $\lim_{n\rightarrow \infty}x_n=z\in O$ Dann $\exists \varepsilon >0$ mit $B_\varepsilon (z)\subseteq O$. Wegen $lim_{n\rightarrow \infty}x_n=z$ gibt es ein $N$ mit $x_n\in V_\varepsilon(z)\subseteq O$ $\forall n>N$.\newline
Angenommen, $O\subseteq X$ ist nicht offen, daher gilt\[\exists z\in O\text{ so dass } \forall \varepsilon>0\smspc B_\varepsilon (z)\not\subseteq O\] Wir setzen $\varepsilon=\frac{1}{n}$ und finden ein $x_n$ mit $x_n\in B_{\frac{1}{n}}(z)\backslash O$ es gilt $\lim_{n\rightarrow \infty} x_n=z\in O$ doch $x_n\not\in O\smspc \forall n\in \mathbb{N}_0$
\lemma{} Eine Teilmenge $A\subseteq X$ ist abgeschlossen genau dann wenn fur jede Folge $x_n$ in A mit $\lim_{n\rightarrow\infty}x_n\in X$ auch $\lim_{n\rightarrow \infty}n\in A$ gilt.
\beweis Wir nehmen an dass $A$ abgeschlossen ist. mit $x_n\in A$ fur $n\in \mathbb{N}$ und $\lim_{n\rightarrow\infty}x_n\in X$ Falls der Grenzwert $\lim_{n\rightarrow\infty}x_n\in X\backslash A$ dann würde aus dem ersten Teil folgen dass $x_n\in X\backslash A$ fur die aller meisten $n$. Doch wie in den ersten Teil des beweises davor, kann dies nicht gelten, also \[\Longrightarrow \lim_{n\rightarrow\infty}x_n\in A\]
\newline Falls aber $A$ nicht abgeschlossen ist so ist $O=X\backslash A$ nicht offen und wir können eine Folge $x_n\in A$ finden mit $\lim_{n\rightarrow\infty}x_n=z\in X\backslash A$. Der beweis ist mir nicht klar... $\approx 1$h 23.02
\bemerkung{} Sei $X$ ein Metrischer Raum und $Y\subseteq X$ eine Teilmenge. Dann ist auch $Y$ ein Metrischer Raum. Wir können die metrik $d:X\times X\rightarrow \mathbb{R}$ auf $d:Y\times Y\rightarrow \mathbb{R}$ einschränken, dann kann man von diesen $Y$ offene und abgeschlossen Mengen definieren.
\beispiel{} Sei $X=\mathbb{R}\supseteq Y=[-1,1]$ Wir können dann offene Bälle in $Y$ betrachten: $B^Y_1(1)=(0,1]\subset Y$ offen
\lemma{} Sei $X\supseteq Y$ wie vorhin, dann ist eine Teilmenge $O_Y\subseteq Y$ genau dann offen genau dann wenn es eine offene Teilmenge $O_X\subseteq X$ gibt mit $O_Y=O_X\cap Y$. Die gleiche Aussage gilt analog für abgeschlossene Mengen.
\beweis Angenommen $O_X\subseteq X$ ist offen. Wir definieren dann $O_Y=Y\cap O_X$ und zeigen dass $O_Y$ offen in $Y$ ist. Sei $y\in O_Y$ dann gibt es ein $\varepsilon >0$ mit $B_\varepsilon^X(y)\subseteq O_X$ und es gilt also
\[\Longrightarrow B_\varepsilon^Y(y)=B_\varepsilon^X(y)\cap Y\subseteq O_X\cap Y=O_Y\]
Der Zweite Teil ist dass wir die Offene Menge in $X$ bilden müssen. $O_Y\subseteq Y$ offen. $\forall Y\in O_Y$ $\exists \varepsilon_y$ mit $B_{\varepsilon_y}^Y(y)\subseteq O_Y$ da $O_Y$ von annahme offen in $Y$ ist. Wir definieren jetzt \[O_X=\bigcup_{y\in O_Y}B_{\varepsilon_y}^X(y)\]
Dies ist Offen in $X$. Behauptung, $O_Y=Y\cap O_X$ dann muss ich die Zwei richtungen dieser gleichung zeigen.
\begin{itemize}
  \item[\textit{i.}]{Sei $y\in O_Y$ dann gilt dass $y\in B_{\varepsilon_y}^Y(y)\subseteq O_X$ (klar da $y$ der Zentrum vom Ball $B_{\varepsilon_y}^X(y)$ ist) was die richtung $\subseteq$ beweist}
  \item[\textit{ii.}]{Sei umgekehrt $z\in Y\cap O_X$ also existiert ein $y\in O_Y$ mit $z\in B_{\varepsilon_y}^X(y)$ da so mein $O_X$ definiert ist. Aber mein $z$ ist ja auch in $Y$ und ich kann auch $z\in B_{varepsilon_y}^Y(y)\subseteq O_Y$ was dann die richtung $\supseteq$ beweist.}
\end{itemize}
Wenn dieses Lemma gilt sagt man das es Relativ offen ist.
\lemma Seien $A,O\subseteq \mathbb{R}$ nichtleer und nach oben beschränkt.\begin{itemize}
  \item{Falls $A$ abgeschlossen dann gilt $sup(A)\in A$}
  \item{Falls $O$ offen ist dann gilt $sup(O)\not\in O$}
\end{itemize}
\definition{Innere Punkte} Sei $X$ ein metrischer Raum und $B\subseteq X$ 
\begin{multicols}{2}
\begin{itemize}
  \item[\textit{i.}]{Wir sagen $x\in B$ ist eine Inneres Punkt falls es ein $\varepsilon > 0$ gibt mit $B_\varepsilon(x)\subseteq B$. }
  \item[\textit{ii.}]{Dass Innere von B ist definiert durch $B^\circ=\left\lbrace x|\exists \varepsilon > 0, B_\varepsilon(x)\subseteq B\right\rbrace$ dies ist die grösste Offene Menge die immer noch in $B$ platz hat.}
  \item[\textit{iii.}]{Wir Sagen $x\in X$ ist ein Randpunkt von $B$ falls $\forall \varepsilon > 0$ gilt $B_\varepsilon(x)\cap B\neq\emptyset\neq B_\varepsilon(x)\cap X\backslash B$ }
  \item[\textit{iv.}]{Wir definieren den Rand durch $\partial B=\left\lbrace x\in X|x\text{ ist ein Randpunkt}\right\rbrace$}
  \item[\textit{v.}]{Der Abschluss ist $\overline{ B}\cup \partial B=\left\lbrace x\in X|\forall \varepsilon >0:\smspc B_\varepsilon(x)\cap B\neq\emptyset\right\rbrace$ }
\end{itemize}
\vfill\null\columnbreak\begin{center}
\scalebox{2}{\tikzfig{mengenAnteile}}
\end{center}
\end{multicols}
\definition{Häufungspunkte} Sei $X$ ein Metrischer Raum, sei $B\subseteq X$ eine Teilmenge von $X$, wir sagen $x_0\in X$ ist ein Häufungspunkt der Menge $B$ Falls $\forall \varepsilon >0$ $\exists x\in B_\varepsilon(x)\cup B\backslash \lbrace x_0\rbrace$
Für eine Folge $(z_n)$ ist $x_0$ ein Häufungspunkt falls $\forall \varepsilon >0$ und $N\in\mathbb{N}$ $\exists n>N$ so dass $d(z_n,x_0)<\varepsilon$
\definition{Dichte Mengen}
Eine Menge $B\subset X$ heisst dicht falls $\forall x_0\in X$ $\forall \varepsilon >0$ $\exists x\in B$ mit $d(x,x_0)<0$ Die menge ist also dicht wenn der Abschluss der Ganze raum ist, oder wenn $B\cup H=X$ wobei $H$ die häufungspunkte sind.
\beispiel{Dichte Mengen}
\begin{itemize}
  \item[\textit{i.}]{$\left\lbrace a+b:\smspc a,b\in\mathbb{Q}\right\rbrace$ ist dicht in $\mathbb{C}$}
  \item[\textit{ii.}]{$\mathbb{Q}^n\subset\mathbb{R}^n$ ist dicht.}
\end{itemize}
\definition{Zusammenhängende Mengen}Sei $X$ ein Metrischer Raum, Eine Menge $B\subset X$ heisst abgeschloffen (clopen) falls die Menge Abgeschlossen und Offen in $X$ ist.
Der Metrische Raum $X$ heisst zusammenhängend falls $\lbrace\rbrace$ und $X$
die einzigen Abgeschloffene Mengen von $X$ sind.
\bemerkung{} $X\in\mathbb{R}$ ist genau dann zusammenhängend als eigenstädinger Metrischer Raum, wenn $X$ ein Intervall ist. Diese Proposition ist sehr wichtig, da sehr viele sachen in Analysis von Intervallen abhängen.
\beweis Wir nehmen zuerst an das $X$ kein Intervall ist, dann $\exists z\in\mathbb{R}\backslash X$ so dass $B_1=(-\infty,z)\cap X\neq \emptyset$ und $B_2=(z,\infty)\cap X\neq\emptyset$ Daraus folgt also dass $B_1$ und $B_2$ abgeschloffen sind, und $X$ ist nicht zusammenhängend. Wir haben gerade zwischen infimum und den supremum eine zahl gefunden, die nicht in der Menge drinn ist (es muss sie geben sonst ist es ein Intervall. Und da 
$B_1$ und $B_2$ komplemente voneinander sind, und beide offen sind, dann sind sie abgeschloffene Menge in $X$. \newline
Falls die Menge $X$ doch ein Intervall ist, und $B\subset X$ ist eine abgeschloffene Teilmenge, mit $B\neq\emptyset$ und $B\neq X$ Da $B$ nicht lehr ist $\exists a\in B$ und $\exists b\in X\backslash B$ wenn $a<b$ nicht gilt, dann wähle ich ein neues $B$ bis dies gilt. Da $B$ offen ist, gilt $\exists \varepsilon >0$ so dass $B_\varepsilon^X(a)\subseteq B$ Daher gilt $[a,a+\varepsilon)\subseteq (a-\varepsilon,a+\varepsilon)\cap X$
Wir definieren also $s=\sup(B\cap[a,b])=\sup(B\cup(B\cup(a,b))$ daher ist das $B=A\cup X$ und die Menge $A$ abgeschlossen in $\mathbb{R}$ ist. Daher sind $B\cap[a,b]$ und $B\cap (a,b)$ auch noch abgeschlossen. und wir finden dass $s\in B\cap [a,b]$ weil es offen ist, und $s\not\in B\cap(a,b)$ da es Offen ist, was ein widerspruch ist.
\lemma{} Sei $X$ ine Metrischer Raum, und $y_1,y_2\subseteq X$ zwei zusammenhängende Teilmengen, falls $y_1\cap y_2\neq\emptyset$ dann ist $y_1\cup y_2$ zusammenhängend.
\beweis Angenommen $B\subseteq y_1\cup y_2$ ist eine relativ abgeschloffene Teilmenge, sei $z\in y_1\cap y_2$. Wir nehmen an dass $B\neq \emptyset$ und $x_1\in B\cup y_1$ Dann gilt $B\cap y_1$ abgeschloffen ist, da $y_1$ zusammenhängend ist, gilt $v\cap y_1=y_1$, und insbesondere gilt also dass $z\in B$ und dass $B\cap y_2=y_2$ und ist $B=y_1\cup y_2$.
\subsection*{Stetigkeit}
\definition{Stetigkeit} Seien $X,Y$ zwei metrische Räume und $f:X\rightarrow Y$ eine Funktion, wir sagen dass $f$ auf $x_0$ $\varepsilon$-$\delta$ stetig ist falls \[\forall \varepsilon >0\smspc\exists \delta > 0\smspc\forall x\in X:d(x,x_0)<\delta\Longrightarrow d(f(x),f(x_0))<\varepsilon\]
\definition{Folgenstetigkeit} Man sagt dass $f$ Folgenstetig ist falls für jede FOlge $(x_n)\in X$ mit grenzwert $\lim_{n\rightarrow \infty} x_n=x_0$ auch $\lim_{n\rightarrow \infty} f(x_n)=f(x_0)$
\lemma{} Es gilt $f$ ist in $x_0\in X$ $\varepsilon$-$\delta$ stetig genau dann wenn $f$ in $x_0$ foglenstetig ist.
\beweis $\Longrightarrow$ angenommen $f$ ist $\varepsilon$-$\delta$ stetig und $x_n$ ist eine Folge mit $\lim_{n\rightarrow\infty}=x_0$. Sei $\varepsilon>0$, dann $\exists\delta >0$ so dass $d(x,x_0)\delta\Rightarrow d(f(x),f(x_0))<\varepsilon$. Da $\lim_{n\rightarrow\infty}x_n=x_0$ ist, gilt $\exists N$ so dass $d(x_n,x_0)<\delta$ $\forall n>N$. Es folgt also dass $d(f(x_n),d(x_0))< \varepsilon$ $\forall n>N$. Da $\varepsilon > 0$ beliebig war, gilt also $\lim_{n\rightarrow\infty}f(x_n)=f(x_0)$ Was dies Richtung beweist\newline
$\Longleftarrow$ Angenommen $f$ ist in $x_0$ nicht $\varepsilon$-$\delta$ stetig. Dann gilt \[\exists\varepsilon_0 >0\smspc\forall \delta >0\smspc \exists x_\delta \in X\text{ mit } d(x_\delta,x_0)<\delta\text{ aber } d(f(x_\delta)),f(x_0))\ge \varepsilon_0\]
wir setzten $\delta=\frac{1}{n}$ und finden eine Folge $x_n$ in $X$ mit $d(x_n,x_0)<\frac{1}{n}\rightarrow 0$ und $d(f(x_n),f(x_0))\ge\varepsilon_0$ also $\lim_{n\rightarrow\infty}x_n=x_0$ aber $\lim_{n\rightarrow \infty}f(x_n)\neq f(x_0)$ und daher ist $f$ auch nicht Folgenstetig ist.
\definition{} $f$ ist stetig falls es in jedem punkt $\varepsilon$-$\delta$ stetig ist.
\bemerkung{} Seien $X,Y$ metrsiche Räume und $f:X\rightarrow Y$ eine Funktion, äquivalent sind folgende aussagen:
\begin{itemize}
  \item[\textit{i.}]{ $f$ ist Stetig}
  \item[\textit{ii·}]{$f$ ist an jedem Punk folgenstetig}
  \item[\textit{iii.}]{Das urbild einer Offene Menge ist Offen , $O\subseteq Y$ ist offen, dann ist $f^{-1}(O)\subseteq X$ ist auch offen. (analog für abgeschlossen.)}
\end{itemize}
\beweis Angenommen, $f$ ist Stetig, in jedem punkt ist $f$ $\varepsilon$-$\delta$ stetig. Sei $O\subseteq Y$ offen, wir wollen zeigen dass $f^{-1}(O)\subseteq X$ dann auch offen ist. Sei also $x_0\in f^{-1}(O)$ dann gilt $f(x_0)\in O$ Dann gibt es ein $\varepsilon > 0$ mit $B_\varepsilon^Y(f(x_0))\subseteq O$ und daher $\exists \delta>0$ os dass $d(x,x_0)<\delta\Longrightarrow d(f(x),f(x_0))<\varepsilon$ also $f(x)\in O$ und $x\in f^{-1}(O)$ also gilt $B_\delta^X(x_0)\subseteq f^{-1}(O)$
$f^{-1}(O)$ ist also eine Offene Teilmenge von $X$.
\end{document}
