\documentclass{article}
\title{Lineare Algebra II}
\author{Benjamin Dropmann}
\usepackage{geometry}
\usepackage{amsfonts}
\usepackage{amsmath}
\newcommand{\mspc}{\hspace{0.7cm}}
\geometry{margin=1.5cm}
\begin{document}
\maketitle
\section{Polynome}
\subsection{polynomdivision}
Seien $f$ und $g\ne0$ zwei polynome in $K[x]$ dann $\exists q(r), r(r)\in K[x]$ mit $deg(r)=0$ oder $deg((r)<deg(g)$ und $f=qg+r$.\newline
\textbf{Korollar 9.0.4}: Sei $f(x)\in K[x],f(x)=0$ sei $\lambda\in K$ so dass $f(\lambda)=0$. Dann $\exists q(x)\in K[x]$ so dass $ f(x)=(x-\lambda)q(x)$\newline
\textbf{Beweis} $\exists q(x),r(x)\in K[x] \hspace{0.5cm}deg(r)<deg(x-\lambda)=1$ so dass $f(x)=(x-\lambda)(q(x)+r(x), \rightarrow r\in K\Rightarrow f(\lambda)=0$\newline
\textbf{Korollar 9.0.6} Sei $f(x)\in K[x], deg(f)=n>0$ Dann hat $f(x)$ höchstens $n$ Nullstellen. (Fundamentaler satz der Algebra sehr ähnlich).\newline
\textbf{Beispiel 9.0.7} Es sei $f(x)=x+1(x^2+1)$, als poly in $\mathbb{R}[x]$ hat es nur eine nullstelle $x=-1$.\\
Als polynom in $\mathbb{C}[x]$ gilt $f(x)=(x+1)(x+i)(x-i)$\newline
\textbf{Theorem 9.0.8 Fundamentaler Satz der Algebra} Es sei $f(x)\in \mathbb{C}[x], deg(f)=n>0$ dann hat $f(x)$ in $\mathbb{C}[x]$ genau $n$ nullstellen. Dass heisst es existieren $\exists \lambda_1,...,\lambda_n$ nicht unbedingt verschieden, so dass $f(x)=(x-\lambda_1)\cdot\left.\right.\cdots\left.\right.\cdot(x-\lambda_n)$ Wir sagen $\mathbb{C}$ is Algebraisch abgeschlossen.\\
\textbf{9.0.11}: sei $f(x)\in K[x], \lambda\in K$ so dass $f(\lambda=0$ Die Ordnung der Nullstelle (Vielfachheit) $\lambda$ is die Ganze zahl $n\ge1$ so dass $\exists q(x)\in K[x]$ so dass \[f(x)=x-\lambda)^nq(x)\]
\textbf{beispiele 9.0.12}\begin{itemize}
\item[1.]{$f(x)=x+1(x^2+1) $ Einfache nullstelle $\lambda=-1$ daher ist die ordnung $1$}
\item[2.]{$p>2\hspace{0.5cm}g(x)=x^p\in\mathbb{F}_p[x]\mspc$ }
\end{itemize}
$\mathbb{F}_p=[a_nx^n+...+a_1x+a_0|n\ge0,a_i\in\mathbb{F}_p]$ Und $g(x)=x^p-1=(x-1)^p$ (leicht ausrechnen)
\\\textbf{Bemerkung 9.0.13} Analogien $\mathbb{Z}\leftrightarrow K[X]$\newline
\begin{center}
\begin{tabular}{c|c}
	$\mathbb{Z}$&$K[x]$\\\hline
	$\pm1$&$K\backslash 0$\\
	Primzahlen&Unzerlegbare Polynome grad$<$0\\
	$\mathbb{Z}/_{p\mathbb{Z}}=\mathbb{F_p}$&$f(x)$ ist unzerlegbar: $K[x]/_{f(x)}$ Körper
\end{tabular}
\end{center}
\section{Eigenwerte und Eigenvektoren}
\textbf{Definition 10.1.1} $V/K$ Vektorraum, $T:V\rightarrow V$ Endomorphismus.
\begin{itemize}
\item[1.]$\lambda\in K$ ist ein Eigenwert von T wenn $\exists v\in V, v\neq 0_v$ so dass $T(v)=\lambda v)$
\item[2.]Ein solches V heisst Eigenvektor  mit Eigenwert $\lambda$
\end{itemize}
\textbf{Bemerkung 10.1.12} Wenn v Eigenveltor von T ist, $T(v)=\lambda v$ dann ist auch $\alpha v$ Eigenvektor von T mit Eigenwer $\lambda, \forall \alpha\in K, \alpha \neq0$\newline
\textbf{Beispiele 10.1.3 Rechnung von eigenwerte und Eigenvektoren}
\begin{itemize}
\item[1.]{$A= \begin{pmatrix}1&2\\2&1\end{pmatrix}$ Eigenwerte $\lambda = 3$ und $\lambda=-1$\\
\[A\cdot\begin{pmatrix}x\\b\end{pmatrix} = \lambda\cdot\begin{pmatrix}x\\b\end{pmatrix}\]
Wir kommen dann auf
\[\begin{pmatrix}1x&2y\\2x&1y\end{pmatrix}=\lambda\cdot\begin{pmatrix}x\\b\end{pmatrix}\]
und also
\[2x+y=\lambda x\]
\[x+2y=\lambda y\]
Wir bekommen also
\[y((1-\lambda)^2-4)=0\]
$y\neq0, x\neq0$ Da die nullvektoren keine Eigenvektoren sind $\Rightarrow (1-\lambda)^2=4\Rightarrow \lambda=[-1,3]$ Warum spezifisch zwei?}
\item[2.]{$B=\begin{pmatrix}1&-2\\1&4\end{pmatrix}$
 Wir Suchen ein $\lambda$ sodass $b(v)=\lambda\cdot v$ für $v\in \mathbb{R} ^2, v\neq0$
 \[ \left(B-\lambda\begin{pmatrix}1&0\\0&1\end{pmatrix}\right)v=\begin{pmatrix}0\\0\end{pmatrix}\] Alsow für welche $\lambda$ ist $B-\lambda\begin{pmatrix}1&0\\0&1\end{pmatrix}$ nicht invertierbar (wann ist der kern nicht trivial) $\Leftrightarrow$ Für welche $\lambda \in K$ ist $det\left(B-\lambda\begin{pmatrix}1&0\\0&1\end{pmatrix}\right)=0?$
 \[det\left(\begin{pmatrix}1-\lambda&-2\\1&4-\lambda\end{pmatrix}\right)=(1-\lambda)(4-\lambda)\Rightarrow \lambda =[2,3]\]
 Und jetzt fur die Eigenvektoren: für $\lambda = 2$
 \[b(v)=2v\Rightarrow v=\alpha\begin{pmatrix}-1\\2\end{pmatrix}, \alpha\neq0\]}
\end{itemize}
\textbf{Satz 10.1.4} $T:V\rightarrow V$ linear. Dann gilt: $\lambda \in K$ eigenwert von $T\Leftrightarrow ker(T-\lambda1_v)=0$

\section{Eigenwerttheorie}
$T:V\rightarrow V$ linear, dann ist $\lambda \in K$ eine Eigenvector wenn $\exists v\in V,v\neq 0_v$ so dass $Tv=\lambda v$. Hier merken wir dass der skalar eines Eigenvektors, auch ein egeinvektor ist, und dass die addition von zwei vektoren mir den selben eigenwert, auch ein Eigenvektor ist, also hat dies die Struktur eines unterraums...\newline
Wir sind auf dem Folgenden Satz gekommen. Sei $T:V\rightarrow V$ linear, dann gilt $\lambda \in K$ ist genau dann Eigenwert von $T$ wenn $ker(T-\lambda I_n)\neq\left\lbrace\emptyset\right\rbrace$\newline
\textbf{Beweis} $\lambda \in K$ Eigenwert $\Leftrightarrow \exists v\in V, v\neq 0_v$ so dass $Tv=\lambda v\Leftrightarrow (T-\lambda I_n)v=0_v$ Und daher ist $v\in ker(T-\lambda I_n)$ 
\newline Das ist Praktisch da wenn $(T-\lambda I_n$ nicht injektiv ist dann ist $ker(T-\lambda I_n)\neq \emptyset$ und wenn die Determinante nicht null ist dann ist $T-\lambda I_n$ kein endomorphismus.\newline
\textbf{Bemerkung} 0 ist ein Eigenwert wenn $T$ kein isomorphismus ist\newline
\textbf{Korollar} Folgende aussagen sind äquivalent: \begin{itemize}
\item{$\lambda$ ist ein Eigenwert von $T$}
\item{$ker(T_\lambda I_n)\neq =0_v$}
\item{$T-\lambda I_n$ ist kein Isomorphismus}
\item{$det(T-\lambda I_n)=0$}
\end{itemize}
Der Beweis ist eine zusammenfassung von vorherigen beweisen
\newline Mit dieses wissen kann man Finden dass es hochstens $n$ Unterschliedliche eigenwerte gibt, da die mit einen grad $n$ polynom definiert sind.
\subsection{Das charachteristische polynom} \textbf{Definition 10.2.1} Sei $A\in M_{n\times n}(K)$. dann ist $X_a(x)=det(A-x1_n)$ das charakteristische polyom von A
Für eine $2\times 2$ Matrix ist dann \[X_A(X)=x^2-\underset{Tr(A)}{\underbrace{(a-d)}}x+\underset{det(A)}{\underbrace{ad-bc}}\]
Kleine errinerung, die Trace ist die Summe der Diagonale elemente. Diese bemerkung gilt auch für $3\times3$. Wir rechnen jetzt für $n\times n$. Der Konstante term von $det(A-xI_n)$ ist $det(A)$ (da es der Fall bei $x=0$ ist)
\newline
Insbesondere:
\[X_1(x)=x^2-2x+1=(x-1)^2\]
\textbf{Definition 10.2.3} $T:V\rightarrow V$ linear dann ist $X_T(x)=det([T]_b^b-xI_n)$ Für eine Basis $B$ von $V$.\newline
\textbf{10.2.4} $X_T(x)$ ist wohldefiniert.\newline
\textbf{Beweis}
\[[T]^C_C=[D]^B_C[T]^B_B[D^{-1}]^C_B\] 
Multiplikativität von det:
\[det([T]^C_C-xI_n)=det([D]^B_C[T]^B_B[D^{-1}]^C_B-xI_n)=det([D]^B_C[T]^B_B[D^{-1}]^C_B-xD^{-1}D)=det(D)det([T]^B_B-xI_n)det(D^{-1})\]
Was unsere  aussage zustimmt.
\newline Da  das  Charakteristische  Polynom unabhängig von der Wahl der Basis, ist sie Eindeutig und daher Wohldefiniert.
\newline\textbf{Theorem 10.2.5} Es sei $T:V\rightarrow V$ linear, dann gilt dass\[\left\lbrace\text{Eigenwerte von }T\right\rbrace=\left\lbrace\lambda\in K|X_T(\lambda)=0\right\rbrace\]
\textbf{Lemma 10.2.6} Sei $A=(a_{ij})\in M_{n\times n}(K)$ Eine Obere Dreiecksmatrix. Dann ist das Charakteristische Polynom \[X_A(x)=\Pi_{i=1}^n(a_{ii}-x)\]
\newline
$Tr:M_{n\times n}(K)\rightarrow K\mspc A=(a_{ij}\rightarrow \sum a_{ii})$ Ist wohldefiniert.
\newline\textbf{Definition 10.2.7} Sei $T:V\rightarrow V$ linear  dann ist $Tr(T)=Tr([T]^B_B)$ Wobei $B$ eine Basis, Wohldefiniert (\textbf{Satz 10.2.8}).
\newline\textbf{Beweis} Zu Zeigen, wenn $\mathbb{C}$ eine andere Basis ist, und  $D=[id_v]_\mathbb{C}^B$ eine Basiswechselmatrix ist, dann gilt: \[Tr[T]^B_B=Tr(D^{-1}[T]^\mathbb{C}_\mathbb{C})\] Hier Bleibt nichts übrig ausser es auszurechnen, aber es funktioniert, es
reicht aus  zu zeigen, Wenn $M_1,M_2\in M_{n\times n}(K)$ dann gilt $Tr(M_1\cdot M_2)=Tr(M_2\cdot M_1)$ Da wenn dass gilt dann kürzt sich der $D, D^{-1}$. Dass ist eine Explizite berechung.
\textbf{Satz 10.2.9} Sei $T:V\rightarrow V$ linear, dann gilt \[X_T(x)=(-1)^nx^n +(-1)^{n-1}Tr(T)x^{n-1}+\cdots+det(T)\]
\newline \textbf{Beweis} Es sei $A=[T]^B_B$

\end{document}
