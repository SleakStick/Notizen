\documentclass{article}
\usepackage{pgfplots}
\pgfplotsset{compat=1.18}
\usepackage{multicol}
\usepackage{tikzit}
\input{PhysiksStyle.tikzstyles}
\usepackage{geometry}
\usepackage{titlesec}
\titlespacing*{\subsubsection}{0pt}{1.2ex}{.1ex plus .2ex minus .2ex}
\usepackage{amsmath}
\usepackage{amsfonts}
\usepackage{amssymb}
\geometry{margin=1.5cm}
\author{Benjamin Dropmann}

\newcommand{\mspc}{\hspace{0.4cm}}
\newcommand{\experiment}{\\[2ex]\textbf{Experiment }}

\begin{document}
\begin{abstract}
\subsection{Bonusprogramm}
12 serien* 5 punkte / serie$\rightarrow$ 60 oubkte\begin{itemize}
\item[\textbullet]{0-19 Kein bonus}
\item[\textbullet]{20-39 Lineare interpolation}
\item[\textbullet]{$>$40 maximaler bonus 39, 40 schon maximler bonu}
\end{itemize}
\end{abstract}
\section{Wellen}
\subsection{Federwelle} Gutes model für eine Festkoper, Transversale Anregung, senkrecht zur länge, Longitundonale Anregung, entlang der  längle des feder dings. (Elektromagnetik schwer da es in beide richtungen geht). Es gelten Folgende Bedingungen für die federwelle:
\begin{itemize}
\item[\textbullet]Jede masse Schwingt um ihre ruhelage, (wie eine Pendel)
\item[\textbullet]Jede masse bleibt in ruhe bis die welle sie erreicht
\item[\textbullet]Ruckrehr zur ruhelage
\end{itemize}
\subsubsection{Amplitude} der Welle $\xi(x,t)$ ($x$ : Ort fur die Seilwelle, $t$ zeit)
\subsubsection{Dispersion}: Form des wellenpakets der anregung bleibt unverändert
$\xi(x,t=0)=f(x)$ $f(x)$ ist die form des wellenpakets\\$x-a$ fuhrt zu einer translation der Wele ohne ändergun seiner form:
\[c\rightarrow x-a\rightarrow \xi(x-a,t=0)\rightarrow (x+a)\]
\[a=vt \rightarrow f(x\pm vt)\]
\[\xi(x,t)=f(x\pm vt\] $v$ ist hier die \textbf{Phasengeschwindigkeit} der Welle.
\subsubsection{Harmonische Wellen} Vom Harmonischen Oszillator, Wellengleichung herleiten, allgeimene wellengleichung finden. Eine Harmonische Welle ist eine sinus (cosinus ist besser) kurve
\[\xi(x,t)=\xi_0\cdot\sin(k(x\pm vt=f(x\pm vt)\]
\[\text{Wellenzahl }k(x+\lambda)=kx+2\pi \rightarrow k\lambda = 2\pi \rightarrow k=\frac{2\pi}{\lambda}\]
$k$ ist die Wellenzahl (so dass was im sinus ist dimensionslos ist)  \[\lambda:{Wellenlange}\]
In zwei dimensionen ist $k$ä ist ein vektor und zeigt uns die wellen direktion aus (longitude, oder senkrecht)
\\Kreisfrequenz: $\omega=2\pi\nu=2\pi\frac{1}{T}$ Wo $T$ die Periode ist.\\$\xi(x,t)=\xi_0\sim(k(x\pm vt))=\sin(kx+kvt)=\xi_0\sin(kx\pm\omega t)$
\subsubsection{Wellenglaichung in einer Dimension}
\[\xi(x,t)=\xi_0e^{i(kx\pm \omega t}\]
wir leiten nach der zeit ab:
\[\frac{\delta\xi}{\delta t}=\xi_0(-kv)cos(k(x-vt))\]
\[\frac{\\delta^2\xi}{\delta^2t}=\xi_0(-kv)^2sin(k(x-vt))\]
ncahc dem ort ableiten
\[\frac{\delta\xi}{\delta x}=\xi_0kcos(k(x-vt))\]
\[\frac{\delta^2\xi}{\delta^2x}=\xi_0k^2sin(k(x-vt))\]
Zusammestellen $\frac{\\delta^2\xi}{\delta^2t}=v^2\frac{\\delta^2\xi}{\delta^2x}$
\[\frac{\delta^2\xi}{\delta^2t}-v^2\frac{\\delta^2\xi}{\delta^2x}=0\]
Wie sieht dann die algemeine lösung aus?
\[\xi(x,t)=f(x-vt)+g(x+vt)\]
Ableitung nach der Zeit
\[\frac{\delta\xi}{\delta t}=\frac{\delta f(x-vt)}{\delta t}+\frac{\delta f(x+vt}{\delta t}=\frac{\delta f(\alpha(x,t)}{\delta t}+\frac{\delta g(\beta(x,t)}{\delta t}\]
Weiter und weiter ableiten und herumschreiben:
\[\frac{\delta f}{\delta \alpha}(\alpha(-v)+\frac{g(\beta)}{\delta \beta}(\beta(v)\]
Für die zweite ableitung gilt diese hergehensweise auch, und wir finden dass die gleicung oben erfüllt ist und dass folgende gleichung gilt:
\[\frac{\delta^2\xi}{\delta^2t}-v^2\frac{\delta^2\xi}{\delta^2x}=0\]
Gute frage: jede sinusfunktion erfullt das; wir haben nur angenommen dass $x$ und $t$ einen anhang ($f$ in diesen fall) haben
\experiment DNA dings, sehr wenig reibug zwischung elemente $\rightarrow$ rucktreibendende kraft sehr gering. Auch reflektion. 

\subsubsection{Transversale Wellen}
\[\xi(z,t)=Af(z-vt)\]
Diesmal ist $k$  aber ein Vektor
\[xi(z,t)=A\cos(kz\omega t)\hat{x}\]
Anhange zur spannung(seilwelle) Seilwelle$\rightarrow$ Wellengleichung
Wir nehmen viele kliene massenelemente den seil entlang, Dann haben wir zwei kräfte, den seil hoch/entlang und die spannung des seils/nach unten. Wir haben jetzt für eine massen element zwei funktionen $xi(x)$ und $\xi(x+dx)$ Wir brauchen also der unterschid zwischen diese zwei kräfte, die nicht entgegengesetzt sind wegen der breite des massenelements.
\[\Delta S_y=S\sin(\alpha')-S\sin(\alpha)\]
Herumdingen
\[\Delta S_y=S\frac{\delta^2\xi}{\delta x^2} dx\]
Ich habe verpasst Elastizität modul?
\subsubsection{Räumliche verteilung von Wellen}
\[\xi(x,y,z,t)=Af(kz-\omega t)\]
Transversale Welle: Polarisationsrichtung (in x oder y schwingen, transversal aber anders)
\[A=\begin{pmatrix}A_x\\A_y\end{pmatrix}\mspc \xi(t)=\begin{pmatrix}A_x\\A_y\end{pmatrix}\cdot e^{ikz-\omega t}\]
Die wellenzahl wird jetzt zu einem Vektor der beschreibt in welcher richtung diese welle sich ausbretet. 
\[\xi(r, t)=Ae^{i(kr-\omega t)}\] wobei $k=\begin{pmatrix}0\\0\\k_z\end{pmatrix}$
\subsection{Wellengleichung in drei dimensionen}
\[\frac{1}{v^2}\frac{\delta^2\xi}{\delta t^2}-\frac{\delta^2\xi}{\delta x^2}-\frac{\delta^2\xi}{\delta y^2}-\frac{\delta^2\xi}{\delta z^2}=0\]
Laplace operator $\Delta=\nabla^2=\left(\frac{\delta}{\delta x},\frac{\delta}{\delta x},\frac{\delta}{\delta x}\right)\begin{pmatrix}\delta/\delta x\\\delta/\delta y\\\delta/\delta z\end{pmatrix}$ Also es gilt \[\frac{1}{v^2}\frac{\delta^2\vec{\xi}}{\delta t^2}(x,y,z,t)-\Delta\vec{\xi}=0\]
\subsection{Kugelwellen}
Beispiel punktformige Lichtquelle. Hier ist $k$ nicht mehr wohldefiniert, da die welle sich in alle richtungen ausbreitet.
\[\vec{\xi}_0\cdot e^{i(\vec{k}\vec{r}-\omega t}\]
\[\frac{\delta \vec{\xi}}{\delta x}=ikxe^{i(\vec{k}\vec{r}- \omega t}\]
In alle richtungen und zweimal ableiten, und wir finden:
\[\Delta\vec{\xi}(\vec{r},t)=-k^2\vec{\xi}(\vec{r},t)\]
Und dann dasselbe mit der zeit:
\[\frac{1}{v^2}\frac{\delta^2\vec{\xi}}{\delta t^2}-\Delta\vec{\xi}=\xi\left[-\frac{\omega^2}{v^2}+k^2\right]\vec{\xi}\mspc v=\frac{\omega}{k}\]
Mit der Kugel symmetrie $\Delta = \frac{\delta^2}{\delta x^2}+\frac{\delta^2}{\delta y^2}+\frac{\delta^2}{\delta z^2}$
bekommen wir dann 
\[\frac{\delta \phi}{\delta x}=\left(\frac{\delta r}{\delta x}\frac{\delta}{\delta r}+\frac{\delta \theta}{\delta x}\frac{\delta}{\delta \theta}+\frac{\delta \phi}{\delta x}\frac{\delta}{\delta \phi}\right)\phi\]
\[\frac{\delta r}{\delta x}=\sin(\theta)\cos(\phi)\]
\[\frac{\delta\theta}{\delta x}=\frac{\delta}{\delta x}\left[\arccos\left(\frac{z}{r}\right)\right]=\frac{1}{\sqrt{1-\frac{z^2}{r^2}}}\frac{\delta}{\delta x}\frac{z}{r}=\frac{1}{\sqrt{1-\frac{z^2}{r^2}}}\left(-\frac{1}{2}\frac{1}{r^3}2xz\right)\]
\[=\frac{1}{r}\cos(\theta)\cos(\phi)\]
Dasselbe geht jetzt mit $\frac{\delta \phi}{\delta x}$ (Schreibe ich nicht hin)
\subsection{Kugelwellen, Wellengleichungen in 3d}$\vec{k}\cdot\vec{r}= |\vec{k}|\cdot|\vec{r}|=k\cdot r$ Da $k$ und $r$ immer parallel laufen (dank der Kugelsymmetrie)
\[\vec{\xi}(r,t)=\frac{\vec{A}_1}{r}f_1(kr-\omega t)+\frac{\vec{A}_2}{r}f_2(kr-\omega t)\]
Diese lösung erfullt die differentialgleichung.
\subsection{Energietransport} Die Geschwindigketi eines massenstücks $v=\frac{\delta \xi(\vec{r},t}{\delta t}$\\
Die kinetische energie dieses massenstuck $dT=\frac{1}{2}\left(\frac{\delta \xi}{\delta t}\right)^2 dm$
Energie dichte $\frac{dT}{dV}$
\\
Elastische energie:
\[E_{el}=\int_0^{\Delta l}(\Delta l')d(\Delta l')=A\int_0^{\Delta l}E\frac{\Delta l^2}{l}=\frac{1}{2}(A\cdot l)E\left(\frac{\Delta l}{l}\right)^2\]
\[\frac{\Delta l}{l}=\frac{\delta \xi}{\delta x}\Rightarrow\frac{1}{2}E\left(\frac{\delta \xi}{\delta x}\right)^2\]
Energie dichte:
\[\frac{dT}{dV}=\frac{1}{2}\rho v^2 f'^2\]
\[\frac{dE_{el}}{dV}=\frac{1}{2}Ef'^2\]
Pro volumen gerechnet ist die elastische und kinetische energie dieselbe. Die Gesamtenergie ist alow $\frac{dW}{dV}=\rho v^2f'^2$\\
\subsection{Wellenfunction $\xi$}
\[\xi(x,t)=\]
Die Form einer welle: $f(x)=\xi(x,t=0)$ ist der initiale gefrorene status einer Welle. Für jetzt, ist die Form konstant (dämpfungen sind benachlässigt). Wegen der Form der Welle, ist ort und Zeit nicht unabhängig, da die Form der Welle nur den Ort definiert bei einer konstanter Zeit.\\
Nach eine Zeit hat sich die Welle bus zum punkt $x\pm vt$ ausbreitet, wobeir $v$ die Wellengeschwindigkeit ist. Die Wellenfunktion ist also\[\xi(x,t)=f(x\pm vt)\]
\subsection{Harmonische Welle}
Eine harmonische Welle ist beschrieben durch ein sinus oder ein cosinus:\[\xi(x,t)=A\cdot\sin(k(x\pm vt)\] wobei $k$ die Wellenzahl (später Wellenvektor) $\left[\frac{1}{m}\right]$, ist, $A$ die Amplitude. Es gilt $k=\frac{2\pi}{\lambda}$ wo lambda die Wellenlänge ist. Also man kann die folgende vereinfachung machen:
\[k(x\pm vt)=kx+kvt=kx+\omega t\]
wobei $\omega$ die Kreisfrequenz ist.\\
\subsection{Wellengleichung}
Wir leiten nahc der Zeit ab
\[\frac{\delta\xi}{\delta t}=A(-\omega)\cos(kx-\omega t)\]
\[\frac{\delta^2 \xi}{\delta t^2}=a-\omega^2A\sin(kx-\omega t)\]
Und jetzt nach dem Ort:
\[\frac{\delta\xi}{\delta x}=Ak\cos(kx-\omega t)\]
\[\frac{\delta^2 \xi}{\delta x^2}=-k^2A\sin(kx-\omega t)\]
Wie setzen dies zusammen und Bekommen:
\[=\frac{\omega^2}{k^2}\frac{\delta^2\xi}{\delta x}=v^2\frac{\delta^2\xi}{\delta x^2}\]
\[\frac{\delta^2 \xi}{\delta t^2}-v^2\frac{\delta^2\xi}{\delta x^2}\]
Die Allgemeine Lösung olgt folgender Form:
\[f(x-vt)-g(x+vt)\]
\subsubsection{Arten von Wellenverbreitung}
\begin{itemize}
\item{Transversalwelle: Auslenkung senkrecht zur geschwindigkeit}
\item{Longitudonalwell, die Auslenkung ist parallel zur ausbreitung der Welle}
\end{itemize}
\subsubsection{Energietransport einer Welle}
Eine welle transportiert kinetische und elastische energie, die Beträge dieser beiden einergien ist im volumen (flachenelement oder distanz) immer gleich.
\subsubsection{Tipps zur Serie 1}
\begin{itemize}
  \item[$1.1_a$]{Vollständige Wellenfunktion finden, (wichtige dingen oben sind hilfreich)}
\item[$1.1_b$]{Die orte einfch einsetzen und die trigonometrische vereinfachen mit den mathemathischen hilfsmitteln der formelsammlung}
\item[$1.2_a$]{Uhr und Lineal, so dass man zeiten und abständen messen kann. Welche gössen sind gegeben, und welche sin messbar? damit vereinfachen. Die Wellenlänge kann man (theoretisch messen) also mit der Wellenlänge die distanz ausrechen.}
\end{itemize}
\subsubsection{Stehende Welle} Die Stehende Welle ist einfach eine summe der Zwei wellen die sie aufführt.
\subsubsection{Reflection und Transmission} Transmission ist in derselben richtung als einkommende Welle, Reflektierte welle dagegen
\[\xi_A=Ae^{i(k_1x-\omega t)}\]
\[\xi_R=Ree^{i(-k_1x-\omega t +\delta_R)}\]
\[\xi_R=Te^{e(kx-\omega t +\delta_T)}\]
Wir konnen diese glaichungen mit zwei Parameter (zwei gleichungen) losen.
\newline Wir haben Zwei Bedingungen:
\newline Steigkeit\[lim_{x\rightarrow 0^-}(\xi_a+\xi_R)=lim_{x\rightarrow0^+}\xi_T\]
Und Kraftegleichgewicht:\[S_1\frac{\delta\xi_a}{\delta x}\left.\right|_{x=0}+\frac{\delta\xi_R}{\delta x}\left.\right|_{x=0}=\frac{\delta\xi_T}{\delta x}\left.\right|_{x=0}\]
\[A+Re^{i\delta_R}=Te^{i\delta_T}\]
Imaginärteil: $T\sin(\delta_T)=R\sin(\delta_R)$
\[Kraftgleichung\mspc AS_1k_1=TS_2k_2e^{i\delta_T}+RS_1k_1e^{i\delta_R}\]
\[TS_2k_2\sin(\delta_T)+RS_1k_1\sin(\delta_R)=0=T\sin(\delta_T(S_2k_2+S_1k_1))\]
Und wir bekommen
\[k_i=\frac{\omega}{v_i}\]
\[\alpha=\frac{S_2\delta_2}{S_1\delta_1}\]
Materialparameter $\alpha$ Ist ein Index von den Geschwindigkeiten der Welle in den beiden Materien (für ein seil ist $S$ die Spannung):\[T\sin(\delta_T)(\alpha+1)=0\Rightarrow\sin(\delta_T)=0\]
  \[\delta_T=0\mspc\lim_{v_1\rightarrow v_2}\xi_A=\xi_T\]
  \[\delta_T=\pi\lim_{v_1\rightarrow v_2}\xi_A=-\xi_T\]
Es muss also $\delta_T=0$ sein da der Zweite fall unphysikalisch ist.
\subsubsection{Reflektierte Welle}
Es gibt nochmal die Zwei mmoglichkeiten: $\delta_R=0$ oder $\delta_R=\pi$ 
\[A=T\pm R \text{ oder } A=\alpha T\mp R\]
\[R=\pm\frac{1-\alpha}{1+\alpha}A\mspc T=\frac{2A}{a+\alpha}\]
Spezailfälle:

\begin{itemize}
\item{$\alpha=1\Rightarrow S_1\delta_1=S_2\delta_2\mspc\mspc R=0, T=A$}
\item{$\alpha>1\Rightarrow\delta_R=\pi$ bei $\alpha\rightarrow\infty$ wird alles reflektiert und nichts transmittiert}
\item{$\alpha<0\Rightarrow R\ge0$ und  $\delta_R=0\Rightarrow R=\frac{1-\alpha}{1+\alpha}A,\mspc T=\frac{2A}{\alpha+1}$}
\end{itemize}
\subsubsection{Stehende Wellen} Wir haben jetzt in dem Gedanksexperiment 2 laufende Wellen, und zwei Grenzflächen $\Rightarrow \xi=2A\cos(kx-\frac{\delta_R}{2})\cos(\omega t-\frac{\delta_R}{2})$
\newline Reflexion am hartem Medium:$\alpha>>1, \delta_R=\pi$
\[\xi=2Asin(kx)\sin(\omega t)\]
\subsubsection{Energieverteilung der Stehende Wellen} Kinetische Energiedichte \[\frac{dT}{dV}=\frac{1}{2}\rho\left(\frac{\delta\xi}{\delta t}\right)^2=2\rho A^2\omega^2\sin^2(kx)\cos^2(\omega t)\]
Elastische Energiedichte:
\[\frac{1}{2}E\left(\frac{\delta \xi}{\delta x}\right)^2=2EA^2k^2\cos^2(kx)\sin^2(\omega t)\]
\[k^2=\frac{\omega^2}{v^2}\mspc v^2=\frac{E}{\rho^2}\]
\subsubsection{Eigenschwingungen Einer Seite} $\frac{\delta\xi^2}{\delta t^2}=v^2\frac{\delta^2\xi}{\delta x^2}$ Und daher $v^2=\frac{S}{\rho}$ rechnen rechnen rechnen und wir kommen auf:
\[u(x)=u_0\cos(kx+\phi)=A\cos(kx)+B\sin(kx)\]
Wo $A$ und $B$ ovn Randbedinugen kommen, (z.B feste Seite: $u(x=0)=u(x=l)=0$)
\subsection{Ubungsstunde 2}
\subsubsection{Polarisation} WIr wissen dass die Ausbreitung einer transversalwelle Senkrecht zur ausbreitungsrichtung Steht. Wir nehmen an die Welle breitet sich in der $z$ richtung aus, dann kann die Auslenkung überall auf der $x-<$ Ebene statt finden.\newline Eine Welle hesiit linear falls die Ganze AUslenkung nur in eine Ebene Stattfindet.
Falls mehrere linear POlarisierte Wellen Uberlagert werden, und es zwischen diese Wellen einen Phasenunterschid gibt, dann ensteht eine Elliptisch-Polarisierte Welle
\newline\subsubsection{Beispielaufgben} Angenommen wir haben eune Uberlagerung von:\[y_1(x,t)=5\cos(kx-\omega t)\vec{e_x}\]
\[y_2(x,t)=2\cos(kx+\omega t)\vec{e_y}\]Hier ist die resultierende Welle immer noch linear polarisiert.
Ich habe viel verpasst, laufende Wellen, Stehende Wellen ist keine losung der Wellengleichung.
\newline 
    \subsubsection{Random Facts}
    Sehr nahr zu einer Kugelwellenquelle ist dies keine ein dimensionale Welle, aber sehr weit, kann man es mit viele punktquellen un (eindimensionale quellen)
    \newline
\subsubsection{Kohärenz}
Wie lange ist eine Welle peridodisch (in ort und zeit)?
Eine glühbirne z.B hat in der langen distanz, hat keine Feste Phasenbeziehung zu bestimmte Zeiten und Orten.
Bisher haben wir angenommen dass die Welle unendlich eine sinuswelle.
\newline
Wir mussen interferenzen messen $\rightarrow$ interfometer. Wir messen dieselbe lichtquelle mit unterschiedliche distanzen am selben punkt, wenn da keine phase ist, dann ist die lichtquelle nicht perfekt.
Die Normale interferenz sollte entweder destruktiv oder konstruktiv sein, aber wenn die quelle nicht perfekt, dann ist das ganze nicht perfekt und der maximum ist kleiner als wenn $x_1=x_2$
    \subsubsection{Zwei entgegengesetzte wellen}
    $\vec{r_1}=\begin{pmatrix}-a\\0\end{pmatrix}\mspc\vec{r_1}=\begin{pmatrix}-a\\0\end{pmatrix}$
    \[\xi_1(r,t)=\frac{A}{\sqrt{|r-r_1}}\cos(k|r-r_1-\omega t)\]
    \[\xi_1(r,t)=\frac{A}{\sqrt{|r-r_2}}\cos(k|r-r_2-\omega t)\]
    Wenn $|r-r_1|-|r-r_2|=n\lambda$ dann ist der inteferenz am maximum ($n\in \mathbb{N}$)
    Die maxima sind dann an den punkte
    \[\sqrt{(x+a)^2+y^2}=n\lambda+\sqrt{(x-a)^2+y^2}\]
    Hyperbel schar
    \newline
\subsection{Reflexion+Transmission}
Seilwelle mit eine dicke änderung in der mitte, da die welle von der Spannung abhängt und von der seildichte. Daher auch die ausbreitungsgeschwindigkeit.
    Einlaufende Welle $\xi_A=Ae^{i(kx_1-\omega t)}$\newline
    Reflektierte welle: $\xi_A=Ae^{i(kx_1-\omega t+\delta_R)}$\newline
    Transmittierte welle: $\xi_A=Ae^{i(kx_1-\omega t+\delta_T)}$\newline
Es mussen folgende sachen Gelten:\[lim_{x\rightarrow 0_-}(\xi_A+\xi_R)=lim_{x\rightarrow x_+}\xi_T\]
Die Vertikale kräfte links und rechts der Grenzfläche mussen gleich sein.
\newline
\subsubsection{Wiederholung} Sei ein Seil mit eine Grenzfläche wo sich der Seil ändert, dann gibt es eine Einlaufende, Transmittierte und Reflektierte Welle. Man kann folgende gleichung aufstellen:
\[\alpha=\frac{k_2\cdot S_2}{k_1\cdot S_1}=\sqrt{\frac{S_2\cdot\rho_2}{S_1\cdot\rho_1}}\text{ Wobei } k \text{ der Wellenvektor ist und } S \text{ Die seilspannung ist.}\]
Wir konnen dann alles in unterschiedliche fälle einschachteln \begin{itemize}
  \item{$\alpha>1\mspc\delta_R=\pi, \mspc R=\frac{\alpha+1}{\alpha-1}\cdot A,\mspc T =\frac{2A}{\alpha+1}$ und dieser Fall entspricht einen Festen Ende, also es ist wie wenn wir eine Dunne schnur die zu eine sehr dicke schnur geht.}
  \item{$\alpha<1 \mspc \delta_R=0,\mspc R=\frac{1-\alpha}{1+\alpha}\cdot A,\mspc R=\frac{2A}{1+\alpha}$ Dieser Fall entspricht einem Losen Ende, also wenn die Dicke schnur zu eine Sehr dünne Schnur wird. }
  \item{$\alpha=1$ Dann ist es als ob die schnur gleich geblieben wäre. }
\end{itemize}
\subsubsection{Stehende Wellen} Bei einer Stehende Welle gibt es Knoten und bäuche, die Bedingung einer Stehende Welle (im beispiel der Saite )ist $n\frac{\lambda}{2}=l$ für eine länge von $l$:\[\lambda_n=\frac{2l}{n}\Rightarrow\omega_n=k_nv= \frac{n\pi}{l}\cdot\underset{=v}{\underbrace{\sqrt{\frac{S}{\rho}}}}=n\omega_1\]
\subsubsection{Fourier Transformation}
Sei eine Welle die als Summe von Sinus und Cosinus Wellen beschrieben werden Kann:
\[f(t)=c_0\sum_{n+1}^\infty a_n \cos(n\omega_0 t)+b_n\sin(n\omega_0t)\]
Hier ist $c_0$ eine konstante die die Ganze Welle verschiebt: \[c_0=\frac{1}{T}\int_{\frac{-T}{2}}^{\frac{T}{2}}f(t)dt\] 
\[a_n=\frac{2}{T}\int_{\frac{-T}{2}}^{\frac{T}{2}} f(t)\cos(n\omega_0t)dt\] und 
\[b_n=\frac{2}{T}\int_{\frac{-T}{2}}^{\frac{T}{2}} f(t)\sin(n\omega_0t)dt\]
\experiment
Wir wollen eine Perfekte Rechteckwelle herstellen, wir nehmen also eine Grundwelle und legen dazu ihre ungerade harmonischen, mit jeweils kleinere Amplituden, so dass wir dann $\lim_{n\rightarrow \infty}$ eine Perfekte Reschteckwelle bekommen. Mit dieser methode kann man jede Welle herstellen.
\newline
\subsection{Beugung, Brechung und Dispersion}
Für die Beugung ist das Video von Veritasium sehr gut.\newline
\subsubsection{Prinzip von Huygens} Jede Wellenfront ist eine Überlagerung von Kugelwellen. Dieses Prinzip kann erweitert werden: \newline
Wenn z.B. licht in einen Medium eintrifft, dann werden die partikel aufgeregt in dem sie die Photone einnehmen, dann werden sie wieder ausgestrahlt und aus dieser vorstellung dieser Kugelwelle kann man die wellenfront sehr gut approximieren. Im skrip sind dazu sehr hübsche abbildungen (1.38).
\newline\begin{center}\scalebox{0.8}{\tikzfig{Wellen_Beugung1}}\end{center} Hier kann man also die Phasenverschiebung beschreiben.($N=$ Anzahl punktquellen und $N=2M+1$) \[\Delta l =k\Delta S = \frac{2\pi}{\lambda}\Delta S=k\delta \sin(\alpha)\] UNd hier ist aber $\delta << r$ wobei $r$ der Abstand zum zuschauer ist.
Wir schauen uns also die Überlagerung der Punktquellen am punkt $P$:\[\xi(\alpha)=\sum_{n=1}^N \frac{a}{r} e^{i(kr_n-\omega t)}\]\[r_n=r+(M+1-n)\Delta S \Rightarrow kr_n=k r+(M+1)\Delta \varphi-n\Delta l\]
\[\xi(\alpha)=\frac{a}{r}e^{i(M+1)\Delta \varphi}\underset{\frac{e^{i\Delta\varphi(2M+2)}-e^{i\Delta\varphi}}{e^{-i\Delta\varphi}-1}}{\underbrace{\left[\sum_{n=1}^{2M+1}e^{-in\Delta \varphi}\right]}}e^{i(kr-\omega t)}\]
Kann man das hier vereinfachen:\[\xi(\alpha)=e^{\frac{i\Delta \varphi}{2}}\cdot e^{i\Delta\varphi(M+1)}\cdot\frac{e^{i\Delta\varphi}-e^{i\Delta\varphi M}}{e^{\frac{-i\Delta \varphi}{2}}-e^{\frac{i\Delta \varphi}{2}}}\]
Und hier alles was übrich bleibt ist \[e^{i\Delta \varphi(M+1)}\cdot\frac{\sin(N\frac{\Delta\varphi}{2})}{\sin(\frac{\Delta\varphi}{2})}\]
Diese Rechnung ist exact im Limes $n\rightarrow \infty$. Die Amplitude der Welle \[\xi(\alpha)=\frac{a}{r}\frac{\sin(N\Delta\varphi/2)}{\sin(\Delta\varphi/2)}\cdot e^{i(kr-\omega t)}\]
Und die Intensität gemittelt über ort und Zeit: \[<I>\approx\frac{a^2}{r^2}\cdot\frac{\sin^2(N\Delta \varphi/2)}{\sin^2(\Delta \varphi/2)}\]
\subsubsection{Beugung} Wir setzen $N\rightarrow \infty$ und $\delta \rightarrow \infty$ und dazu sagen wir $N\cdot \delta =d=konst$ dann ist:\[\underset{\delta\rightarrow\infty}{\lim_{N\rightarrow\infty}}<I>\approx \lim a^2 \frac{\sin^2(\frac{1}{2}kN\delta\sin(\alpha))}{\sin^2(\frac{1}{2}k\frac{d}{N}\sin(\alpha))}\]
Wir kürzen Weiter mit der Approximation $sin(x)=x$ für kleine Winkel: \[\approx\underset{\delta\rightarrow\infty}{\lim_{N\rightarrow\infty}}\mspc\frac{\sin^2(\frac{1}{2}kd\sin(\alpha))}{\frac{1}{4N^2}k^2d^2\sin^2(\alpha)}=\underset{=A^2}{\underbrace{(Na)}}^2\frac{\sin^2(\frac{1}{2}\Delta\varphi)}{(\frac{1}{2}\Delta\varphi)^2}\]
Dank dieses $\frac{sin^2(x)}{x^2}$ haben wir also bei $\frac{1}{2}\Delta\varphi=n\cdot\pi$ nullstellen und je weiter weg man ist, je kleiner die Amplitude. Das alles ist sehr eklärlich mit dem double slit experiment aus der Vorlesung.
\experiment Spalt experiment: Wir haben folgende messungen und werte:\begin{itemize}
  \item{Spaltbreite $d$}
  \item{Phasenverschiebung $\Delta\varphi=k\cdot d\cdot\sin(\alpha)$ (Hier ist $d\cdot\sin(\alpha)$ die projezierte spaltbreite.)}
\end{itemize}
Wir konnen also die Beugung am einzelspalt ausrechnen 
\begin{multicols}{2}
\scalebox{1.8}{\tikzfig{SpaltBeugung}}\vfill\null
\columnbreak
Und hier kann man die sehr wichtige Eigenschaft der Beugung am einzelspalt klarer sehen:\[<I>\approx A^2\frac{\sin^2(\frac{1}{2}\Delta\varphi)}{(\frac{1}{2}\Delta\varphi)^2}\]
\end{multicols}
\subsubsection{Reflexion und Brechung } (A la huygens)
\newline Reflexion: Warum ist die Reflexion nur unter dem selben ausgangswinkel? Es is eine Frage der Konstruktiven interferenz, alle andere Ausgangswinkel interferieren destruktiv.\newline
Brechung: Die Erklärung ist nicht sehr gut.. aber es gilt für die Brechung und konstante fräquenz $\nu$ \[\frac{\sin(\alpha_1)}{\sin(\alpha_2)}=\frac{v_1}{v_2}=\frac{\lambda_1}{\lambda_2}\] 
Der Fermat Prinzip ist simpler und es sagt dass: \textit{Licht sucht sich den Schnellsten Weg.}
\subsubsection{Totalreflexion} Wir haben unser brechungsindex $nE\frac{c}{c_i}>1$ wo $c=$ lichtgeschwindigkeit und $c_i=$ Lichtgeschwindigkeit im medium:\[\frac{\sin(\alpha_1)}{\sin(\alpha_2)}=\frac{v_1}{v_2}\Rightarrow \sin(\alpha_2)=\underset{>1}{\underbrace{\frac{c_2}{c_1}}}\underset{\le 1}{\underbrace{\sin(\alpha_1)}}>1 \]
Und dies kann nie Passieren, also es gibt keine Brechung und keine Reflexion, also wass ist los? Es gibt keine Transmission aber es gibt ein bisschen Reflexion und das licht geht auch die Grenzfläche entlang.\newline
\subsection{Dispersion und Gruppengeschwindigkeit} $v_{ph}:$ Die Phasengewschwindigkeit ist die Geschwindigkeit mit welcher sich ein punkt mit konstanter Phase bewegt. In anderen Worten, wenn wir überlagerte wellen haben, ist die Phasengeschwindigkeit die geschwindigkeit des Punktes den wir auf der Welle "Fest Machen". und die Gruppen geschwindigkeit, ist die Geschwindigkeit der Knoten, in deren die kleinere Wellen minimisieren.
\begin{multicols}{2}
  \scalebox{1}{\tikzfig{Wellenpaket}}
  Links sieht man sehr gut wie ein Wellenpaket definiert ist, die Geschwindigkeit des Maximums ist die Gruppengeschwindigkeit, sie geht in der Selben richtung als die Phasengeschwindigkeit ist aber immer kleiner.
\end{multicols}
Man kann so eine Welle mit folgende Gleichung definierien
\[\xi(x,t)=\frac{1}{\sqrt{2\pi}}\int_{-\infty}^{\infty}A(k)e^{i(kx-\omega(k) t)}\mspc dk\]
Wobei $A(k)$ die Amplitudenfunktion ist und $\omega(k)$ die sogenannte dispersion ist.\newline
Wir schauen uns den allgemeinen fall an wo $k\neq0$. Wir wählen $\varkappa$ so dass $k_0-\varkappa \le k\le k_0+\varkappa$ wobei der $k_0$ bezuglich des "Schwerpunkts" (maximum) des Wellenpakets. (ich finde $\lambda$ zwischen zwei maxima und finde damit $k_0$).
Wir konnen also Folgende vereinfachung machen:\[\xi(x,t)=\frac{1}{\sqrt{2\pi}}\int_{k-\varkappa}^{k+\varkappa}A(k) e^{i(kx-\omega(k)t)}\mspc dk=\frac{1}{\sqrt{2\pi}}\int_{-\varkappa}^{\varkappa}A(k_0+\varkappa') e^{i(k(\varkappa')x-\omega(k)t)}\mspc d\varkappa'\]
Dass wird mit $\omega=\omega(k)$ um $k=k_0$ herum, also: $\omega(k)=\omega(k_0)+\varkappa' \frac{d\omega}{dk}\left.\right|_{k=k_0}$ und also:\[\xi(x,t)=\frac{1}{\sqrt{2\pi}}e^{i(k_0x-\omega(k_0)t)}\int_{-\varkappa}^{\varkappa} A(k_0+\varkappa')e^{i(\varkappa'(x-\omega't))}\mspc d\varkappa'\]
Dass integral ist auch seine eigene funktion die von $k$ und $\varkappa$ abhängt, diese funktion nennt man \textit{enveloppe}. Man schreibt also:
\[\xi(x,t)\approx \frac{1}{\sqrt{2\pi}} e^{i(kx-\omega(k_0)t)} \cdot G(x-v_gt)\] wobei $v_g=\frac{d\omega}{dk}\left.\right|_{k=k_0}$ die Gruppengeschwindigkeit ist und $v_{ph}=\frac{\omega}{k}$ die Phasengeschwindigkeit.\newline
Wenn $\frac{d\omega}{dk}=\frac{\omega}{k}\Rightarrow\omega=v\cdot k$ dann ist $v=v_g=v_{ph}$ und man nennt dass eine Lineare Dispersion, wenn die fräquenz linear vom wellenvektor abhängt.\newline
Wenn $\frac{dv_p}{dk}<0\mspc \frac{dv_p}{d\lambda}>0$ ist es die Normale dispersion.
\newline Wenn $\frac{dv_p}{dk}>0\mspc\frac{dv_p}{d\lambda}<0$ Heisst es Anormale dispersion.\newline
Wenn wir lichtwellen in einem Medium haben dann gibt es ein Brechungsindex (der von der Wellenlänge abhängt) und daher $n=n(\lambda)$ und daher gibt es Dispersion. 
\subsection{Dopplereffekt} Was hier neu ist, ist dass die Quelle und der Beobnachter konnen sich bewegen.
\subsubsection{Beobachter ruht, Quelle bewegt sich} Sagen wir dass der beobachter die Wellenberge zählt über einen zeitraum $\Delta t$ , dann ist die anzahl wellenberge $n_\lambda=\nu_Q\Delta t$
man kann auch sagen: $n_\lambda=\nu_Q\Delta t+\frac{v_B\Delta t}{\lambda}$ Also die fräquenz das der Beobachter misst ist: \[\nu_b=\frac{n_\lambda}{\Delta t}=\nu_Q+\frac{v_Q}{\lambda}\]
\subsubsection{Beobachter bewegt, Quelle ruht} Wenn der beobachter sich bewegt, ist die situation gleich, die relative geschwindigkeit ist zu nehmen.

%---------------Elektrostatik----------------------

\section{Elektrostatik} Hier geht es um ruhende Ladungen.
\subsubsection{Elektrische Ladung}\begin{multicols}{2} \begin{itemize}\item{Eine Elektrische ladung ist ähnlich zu einer Masse aber fur die Elektrostatik.} \item{Hier gibt es auch zwei typen von Ladungen; positiv und negativ.} \item{Es wird im Coulomb gemessen} \item{Es gilt die Ladungserhaltung für ein geschlossenes system.} \item{Wenn ein Positives und Negatives teilchen mit der selben Ladung, dann sieht es von aussen als ob es keine Ladung gäbe.} \item{Ladungen gibt es nur in Diskrete Einheiten $e=1,6\cdot10^{-19}C$}\item{Ladungen sind punktquellen}\end{itemize}\end{multicols}
\subsubsection{Coulombgesetz} Die Kraft zwischen zwei Ladungen ist wie folgt gegeben:\[F_{2,1}=k\frac{q_1\cdot q_2}{r_{2,1}^2}\]Wo $q_1, q_2$ die Ladungen der teilchen ist. Da es kein $-$ vorzeichen gibt, stossen sich zwei ähnlich geladene Teilchen ab.
\newline Haben wir viele Ladungen im system, dann ist die Kraft auf dem Teilchen $j$ \[\vec{F_{j}}=\frac{1}{4\pi\varepsilon_0}\sum_{i\neq j}\frac{q_i\cdot q_j}{r_{i,j}^2}\]
\subsubsection{Energie einer Ladungsverteilung} \[W=\int_\infty^{r_{2,1}}-\vec{F_{2,1}}(\vec{r)\vec{ds}}\] und hier ist auch $\vec{ds}=\vec{r_{2,1}}dr$ Wenn man also $F_{2,1}$ einsetzt bekommt man:\[\frac{1}{4\pi\varepsilon_0}\frac{q_1\cdot q_2}{r}\left.\right|^\infty_{r_{2,1}}=\frac{1}{4\pi\varepsilon_0}\frac{q_1\cdot q_2}{r_{2,1}}\]
  Dies kann man auch auf $n$ Teilchen verallgemeinern in derselben art wie wir die Kraft auf $n$ Teilchen verallgemeinert haben. Und daher auch die Gesamtenergie.
\subsubsection{Coulomb konstante vom kristallgitter} Sei ein Kristallgitter, dann ist seine Coulomb Wechselwirkungskraft eine konstante zahl die die Summe der Coulomb kräfte zwischen ein teilchen und alle andere.
\subsection{Das elektrische Feld}Das Elektrische Feld Ist gegeben durch die Kraft die eine probeladung spüren wurde am punkt:\[\vec{E}(\vec{r_0})=\frac{1}{4\pi\varepsilon_0}\sum_{i=1}^n\frac{q_i}{|\vec{r_0}-\vec{r_i}|^3}(\vec{r_0}-\vec{r_i}))=\frac{\vec{F_0}}{q_0}\]
\[\vec{=q\vec{E}\mspc\vec{E}=\lim_{q\rightarrow0} \frac{\vec{F}}{q}}\]
\subsubsection{Feldlinien} Der Elektrische Feld ist ein Vektorfeld, als sind die Feldlinien ähnlich zu der vom Vektorveld
\subsubsection{Ladungsverteilungen} \[\vec{E}(\vec{r})=\frac{1}{2\pi\varepsilon_0}\cdot\int_{R^3}\frac{\rho(\vec{r_i})}{|\vec{r}-{\vec{r'}}|^3}(\vec{r}-\vec{r'})d\vec{r'}\]
\subsection{Das Gaussche Gesetz} Der Fluss $d\Phi=\vec{E}\cdot\vec{da}$ Daher ist der Gesamtfluss \[\Phi=\int d\Phi=\int_{\partial V}\vec{E}\vec{da\Phi=\int d\Phi=\int_{\partial V}\vec{E}\vec{da}}\] Wobei $\vec{da}$ ein infinitesimales flächenvektor.
Der Fluss ist die Menge von Feld der durch eine Fläche geht. Feld der durch eine Fläche geht. Für eine Ladungsverteilung ist der fluss gegeben durch:\[\Phi=\int_{\partial V}\vec{E}\vec{da}=\frac{1}{\varepsilon_0}\sum q_i=\frac{1}{\varepsilon_0}\int_V\rho(\vec{r'})d\vec{r}'\]
Man merkt auch die Schreibweise $d\vec{r}=d^3r=dV=dx\cdot dy\cdot dz$
\subsubsection{Ladunsverteilung auf einer Kugeloberfläche}\[\rho(\vec(r)=\left\lbrace\begin{matrix}0&r<R\\\rho_0&r=R\\0&r>0\end{matrix}\right.\] Und daher kann man auch \[q=\int \rho(\vec{r})d\vec(r)\Rightarrow \vec(E)\vec(r)\left\lbrace\begin{matrix}0&r<R\\\frac{1}{4\pi\varepsilon_0}\cdot\frac{q}{r}&r>R\end{matrix}\right.\]
Dasselbe kann man mit einen Zylinder und eine Unendliche fläche Machen. Beispiele Stehen im skript.
%--------------------bei 2.6 aufnahme starten--------------------
\subsection{Die Energie Des Feldes}Fur eine Kugelaschale gilt:\[\rho(r)=\left\lbrace\begin{matrix}\frac{q}{4\pi R^2d}&\mspc&R<r<R+d\\0&mspc&\text{sonst}\end{matrix}\right.\]
Der Feld ausserhalb ist wie ein Feld von einer Punktquelle:
\subsubsection{Druck}\[p=\frac{\text{Kraft}}{\text{Fläche}}=\frac{dqE(\vec{r})}{dA}=E_{Aus}\cdot\frac{q}{4\pi R^2d}\int_0^d\frac{r}{d}dr=\varepsilon_0\cdot E_{Aus}^2\frac{1}{2}\]
Und dann die Energiedichte $u=p=\frac{dW}{dV}$ und $u=\frac{\varepsilon_0}{2}E^2$ Daher \[U=\int_V\frac{\varepsilon_0}{2}E^2dV\]
\subsubsection{Das Elektrische Potential} Wir gehen wie mit der Gravitation, vom Punkt $a$ zum punkt $b$ und dann ist die Arbeit \[W_{ba}=-\int_b^aq\vec{E}\mspc d\vec{s}\]Und dann geht auch:\[\oint\vec{E}\mspc d\vec{s}=0\]
Also hier ist der Potentialdifferenz (was später mit Spannung zu tun hat) \[ \phi_{ba}=\int_a^b\vec{E}\mspc d\vec{s}\] \[\text{grad}(\phi)\Rightarrow\vec{\nabla}d\vec{s}=d\phi=\frac{\partial\phi}{\partial x}+\frac{\partial\phi}{\partial y}+\frac{\partial\phi}{\partial z}\]
Die Ladung und der Potential mehrere Teilchen, ist analog aber als summe über $n$ Ladungen definiert.
\subsubsection{Potentiale Einfacher Ladungsverteilung} Der Einfachste teil ist der Platten condensator
\begin{multicols}{2}
$\phi_{ba}=\-\int_a^b\vec{E} ds=E(z_b)-z_a)=E\Delta z)$\newline Und dann ist die Energie $W_{ba}=q_0E\Delta z$
\vfill\null\columnbreak
\scalebox{1}{\tikzfig{Plattencondensator1}}
\end{multicols}
\subsubsection{Potential einer Punktladung} Wir haben eine Punktladung $\phi_{ba}=-\int_a^b \vec{E}ds$ wobei \[\vec{E}d\vec{s}=\frac{1}{4\pi \varepsilon_0}...\] Und dann ist der Potential unterschied
\[\phi_{ba}=\phi(b)-\phi(a)=\phi_r(r_0)-0=-\int_\infty^{r_0}\frac{1}{4\pi\varepsilon_0}\frac{1}{r^2}dr=\frac{q}{4\pi\varepsilon_0}\cdot\int_{r_0}^\infty=\frac{1}{4\pi\varepsilon_0}\]
\subsubsection{Potential einer Geladenen Scheibe} Wir Haben jetzt eine Scheibe, der Potential vom Feld dieser Geladenen SCheibe ist
\subsection{Der Satz von gauss} Sei $F()\vec{r})$ ein Vektor feld, dann ist \[\text{div}\vec{F}(\vec{r})=\lim_{V\rightarrow0}\frac{1}{V}\int_{\partial V}\vec{F}d\vec{a}\]
Und \[\Phi=\int_{\partial}\vec{F}d\vec{a}\] Dies heisst dass wenn wir zwei kleine Volumen, dann ist der fluss auf der Grenzseite von 1 nach 2 gleich - der Fluss von der Grenzfläche von 2 nach 1. Also wir konnen:
\[\Phi=\sum_i\int_{\partial V_i}\vec{F}d\vec{a}=\sum_iV_i\int_{\partial V_i}\frac{\vec{F}d\vec{a}}{V_i}\] Und da $\int_{\partial V}\vec{F}d\vec{a}=\int_{V}div(\vec{F}dV)$ Gilt:\[\Phi=\int_{\partial V}\vec{E}d\vec{a}=\frac{1}{\varepsilon}\int_V\rho dV=\int_Vdiv\vec{E}dV\]
Und dann kommt die erste Maxwell gleichung:\[\text{div}(\vec{E}=\frac{\rho}{\varepsilon_0})\]
Physikalish heisst dies das ladungen siend die Quellen der Elektrischen Felder
\subsection{Das Laplace Operator}\[\left.\begin{matrix}\vec{E}=-\vec{\nabla}\Phi\\\vec{\nabla}\vec{E}=\frac{\rho}{\varepsilon_0}\end{matrix}\right\rbrace 
\Delta \Phi=\vec{\nabla}\vec{\nabla}\Phi=\vec{\nabla}(-\vec{E})=\frac{\rho}{\varepsilon_0}\]
\subsection{Der Satz von Stokes} Sei $C=\int_{\partial A} \vec{F}\vec{ds}$ ein Linienintegral dann ist $c=\sum A_i\int_{\partial A}\frac{\vec{F}\cdot\vec{ds}}{...}$ Jai pas abschrieben (deux denieres minutes du vours)
\section{Elektrische Leiter} Da der Untershcied zwischen die Leitfähigkeit vom Isolator und die vom Leiter, isit in der nähe von $10^20$ Daher schauen wir uns den Fall Leitfähigkeit für Leiter $=\infty$ / im isolator $=0$
\newline Im Leiter, ist der Elektrische Feld $0$, die ladungen Arrangieren sich so dass ihr Feld, wenn summiert mit dem externen Feld null ist, und beim isolator gibt es keine Freie Ladungen.
\experiment Der elektrische Feld ist an einer Spitze sehr gekrümmt, deswegen bricht es zu einer Spitze viel einfacher
\subsection{Leiter}Hier ist das Elektrische Potential konstant $\phi$ konstant innerhalb vom Leiter. Die Oberfläche ist eine äquipotential Fläche. Daher steht das Elektrische Feld Senkrecht zur Leiteroberfläche
Der Leiter hat also eine Oberflächenladungsdichte $\sigma$ Also je kleiner De Krümmungsradius, desto grosser $\sigma$
\subsubsection{Beispiel der Geladene Kugeln} Beide Kugeln habem die Felder \[\phi_i=\frac{1}{4\pi\varepsilon_0}\frac{q_i}{r_i}\] Jetzt verbinden wir die zwei Kugeln so dass ihr Potential gleich wird:
\[\frac{1}{4\pi\varepsilon_0}\frac{q_1}{r_1}=\frac{1}{4\pi\varepsilon_0}\frac{q_2}{r_2}\]Und Alles kurzt sich auf:\[\frac{q_1}{q_2}=\frac{r_1}{r_2}\]
\subsection{Das Allgemeine Elektrostatische Problem} Leiter Seien im Vakuum und haben keine Ladung $\Rightarrow \Delta\phi=0$ Wir Haben also Folgende (Neumann) Randbedigungen:\begin{itemize}
  \item{$\phi_k$ ist für alle Leiter Definiiert (Dirichlet Randbegdiung)}
  \item{$Q_k$ ist definiert (Neumann Randbedigung)}
  \item{Eine Mischung aus $\phi_k$ und $Q_k$ ist bekannt}
\end{itemize}
\subsubsection{Influenz} Ladungen im Lieter verschieben isch im externen Elektrischen Feld aber nur Ladunged auf der Oberfläche \[\int \sigma_\text{ind} da=0=\text{Gesamtladung}\]
\subsubsection{Der Eindeutigkeitssatz} Wir haben eine Gegebene Menge von Randbedingungen, Dann gibt es nur eine losung für $\phi(\vec{r})$ und für $\vec{E}(\vec{r})$
\subsubsection{Faraday'sche Käfige} Wir haben eine Hohle leiter Struktur dann haben wir keine Ladung im hohlraum, da es Kein Feld im Leiter gibt, daher auch im hohlraum auch nicht. Es können Keine Elektrische Felder von draussen hereindringen.
$\int \vec{E}\vec{da}=0\Rightarrow \vec{E}=0$
\subsubsection{Spiegelladungen} Was ist der Elektrische Feld Zweier Punktladungen
\end{document}

