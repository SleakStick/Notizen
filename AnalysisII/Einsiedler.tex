\documentclass{article}
\title{Analysis II Einsiedler Version}
\author{Benjamin Dropmann}
\usepackage{geometry}
\usepackage{tikzit}
\input{AnalysisStyle.tikzstyles}
\usepackage{xcolor}
\usepackage{amssymb}
\usepackage{amsfonts}
\usepackage{amsmath}
\usepackage{multicol}
\usepackage{titlesec}
\titlespacing*{\subsubsection}{0pt}{1.2ex}{.1ex plus .2ex minus .2ex}

\newcommand{\mspc}{\hspace{0.7cm}}
\newcommand{\smspc}{\hspace{0.3cm}}

\newcommand{\kk}[1]{\left<\left<{#1}\right>\right>}
\newcommand{\dd}[1]{\hspace{0.2cm}\text{d{#1}}}

\newcommand{\satz}[1]{\subsubsection*{Satz {#1}}}
\newcommand{\korollar}[1]{\subsubsection*{Korollar {#1}}}
\newcommand{\beweis}{\\\textbf{Beweis }}
\newcommand{\beispiel}[1]{\subsubsection*{Beispiele {#1}}}
\newcommand{\bemerkung}[1]{\subsubsection*{Bemerkung {#1}}}
\newcommand{\theorem}[1]{\subsubsection*{Theorem {#1}}}
\newcommand{\lemma}[1]{\subsubsection*{Lemma {#1}}}
\newcommand{\definition}[1]{\subsubsection*{Definition {#1}}}
\newcommand{\behauptung}[1]{\subsubsection*{Behauptung {#1}}}
\hbadness=100000
\geometry{margin=1.5cm}
\begin{document}
\maketitle
\section{Wiederholung}
\definition{Taylor-Polynom} Sei eine funktion $f:(a,b)\rightarrow\mathbb{C}$ um einen punkt $x_0\in (a,b)$ die $n$-mal differenzierbar ist. Dann ist der Polynom 
\[p_{x_0,n}^f(x)=\sin_{k=0}^n\frac{f^{(k)}(x_0)}{k!}(x-x_0)\]
\satz{Taylor Approximation} sei $f:(a,b)\rightarrow\mathbb{C}$ eine $n+1$ mal stetig differenzierbare funktion und $x_0,x\in (a,b)$ dann gilt\[f(x)=P_{x_0,n}^f(x)+R_{x_0,n}^f(x)\]
Wobei diese $R_{x_0,n}^f(x)$ der restglied ist und ich den mit \[R_{x_0,n}^f(x)=\int_{x_0}^xf^{(x+1)}(t)\frac{(x-t)^n}{n!}\dd{t}\]
Für $M_{n+1}=\max\lbrace |f^{(n+1)}(t)|$ mit $t\in (x,x_0)$ gilt $|R_{x_0,n}^f(x)|\le\frac{M_{n+1}|x-x_0|}{(n+1)!}$
\definition{} Eine Funktion $f(a,b)\rightarrow\mathbb{C}$ heisst analytisch falls es zu jedem $x_0\in (a,b)$ $\exists R>0$ so dass \[f(x)=\sum_{n=0}^\infty \frac{f^{(n)}(x_0)}{n!}(x-x_=)!\smspc\forall x:|x-x_0|<R\]
\subsection*{8.6 Numerische Integration}
\satz{} Sei $f:[a,b]\rightarrow\mathbb{R}$ ,ot $n\in \mathbb{N}$ und $h=\frac{b-a}{h}$, $x_e=a+lh$ für  $l\in\lbrace  0,1,\cdots,n\rbrace$  Falls $f$ stetig differenzierbar  ist dann gilt \[\int_a^b f(t)\dd{t}=h\cdot(f(x_0)+f(x_1)+\cdots+f(x_{n-1}))+F_1\] Wobei $F_1$ unser fehler ist: $F_1\le \frac{M_1(b-1)^2}{2n}$
mit $M_1=\max\lbrace|f'(x)|:x\in[a,b]\rbrace$
Falls die Funktion $2$-mal stetig differenzierbar ist dann gilt: 
\[\int_a^bf(t)\dd{t}=\frac{h}{2}\left(f(x_0)+2f(x_1)+\cdots+2f(x_{n-1}))+f(x_n)\right)+F_2\mspc F_2=\frac{M_2(b-a)^3}{6n^2}\]
Man kann auch für $f$ vier mal differenzierbar und $n$ gerade eine solche abschätzung machen, dies ist die Simpson Regel:
\[\int_a^b f(t)\dd{t}=\frac{h}{3}\left(f(x_0)+4f(x_1)+2f(x_2)+4f(x_3)+\cdots+4f(x_{n-1})+f(x_n)\right)+F_4\mspc F_4\le\frac{M_4(b-1)^5}{45n^4}\]
\section*{9. Metrische Räume}
\subsection*{9.1 Konvergenz in Metrische Räume}
\definition{Metrischer Raum} Ein Metrischer Raum ist eine Menge $X$ mit eine Abbildung $d:X\times X\rightarrow \mathbb{R}_{\ge 0}$ Mit folgende Eigenschaften
\begin{itemize}
  \item[i.]{\textbf{Definitheit} $\forall x,y\in X$ gilt $d(x,y)=0\Leftrightarrow x_y$}
  \item[ii.]{\textbf{Symmetrie} $d(x,y)=d(y,x)\smspc\forall x,y\in X$}
  \item[iii.]{\textbf{Dreiecksungleichung} $\forall x,y,z\in X$ gilt $d(x,z)\le d(x,y)+d(y,z)$}
\end{itemize}
Das $d$ hier ist einfach eine Abstandsfunktion, die einen Abstand zwischen zwei elemente der Menge anteilt. Diese Abbildung ist die Metrik.
$X$ kann $\mathbb{R}$, $\mathbb{C}$ oder sogar $\mathbb{R}^n$ sein, die Metrik $d$ kann irgendeine Funktion sein:
\begin{itemize}
  \item{$d(x,y)=|x-y|$ bei $X=\mathbb{R}$ oder $\mathbb{C}$}
  \item{$d_2(x,y)=||x-y||_2=\sqrt{\sum_{j=1}^d (x_j-y_j)^2}$}
  \item{$d_1(x,y)=||x,y||_1=\sum_{j=1}^d|x_j-y_j|$}
  \item{$d_\infty=||x-y||_\infty=max_i|x_i-y_i|$}
\end{itemize}
\definition{Norm} Ein Normierter, Reeller Vektorraum ist ein Vektorraum $V$ uber $\mathbb{R}$ gemeinsam mit eine Abbildung $||\cdot||:V\rightarrow R_{\ge0}$ mit fonlgende Eigenschaften
\begin{itemize}
  \item{\textbf{Definitheit} $||v||=0\Leftrightarrow v=0$}
  \item{\textbf{Homogeneitat} $||tv||=|t|\cdot||v||$}
  \item{\textbf{Dreiecksungleichung} $||u+v||\le||u||+||v||$}
\end{itemize}
Diese Abbildung $||\cdot||$ wird norm genannt wenn diese Axiome $\forall v,u\in V$ gelten
\lemma{}Eine Norm auf $V$ definiert eine Metrik $d(v,w)=||v-w||$ \beweis Die Axiome der Norm passen mit dieser Definition mit den Axiomen der Metrik.
\beispiel{Paris/SNCF-Metrik}
Diese Metrik ist auf $X=\mathbb{C}$ definiert, Die Metrik bekommt ihr namen vom Fakt dass es in Frankreich mit der bahn, sehr leicht ist von irgendwo nach Paris hinzukommen, und von Paris aus irgendwo anders zu gehen.
Die Metrik ist also \[d(z,w)=\left\lbrace\begin{matrix}|z|+|w|&\text{Wenn }\not\exists \lambda\smspc w=\lambda z\\|z-w|&\text{Wenn} \exists \lambda \smspc w=\lambda z\end{matrix}\right.\]
Diese Metrik ist sehr Komisch aber respektiert immer doch die Axiome. Die Konvergenz, die bald definiert wird, hat Komische Konvergenz da sogar wenn punkte optisch sehr nah miteinander aussehen, sind die wegen der Metrik doch nicht. Deswegen, nehmen wir öftestens Normen auf Teilmengen als Metriken.
\definition{Konvergenz in einem Metrischen Raum} Sei $X$ ein Metrischer Raum, und $x_n$ eine Folge in $X$ und $z\in X$ Wir sagen dass $x_n$ gegen $z$ konvergiert und schreiben $\lim_n\rightarrow\infty x_n=z$ falls \[\forall \varepsilon>0 \exists N\smspc \forall n>N: \smspc d(x_n,z)<\varepsilon\]
\definition{Offene Ball} Sei $X$ ein Metrischer Raum. Der Offene Ball um $x_0\in X$ mit radius $r>0$ ist durch \[B_r(x_0)=\left\lbrace \left. x\in X\right|d(x,x_0)<r\right\rbrace\] definiert. Eine Menge $U\subset X$ heisst umgebung von $x_0$ falls es ein $\exists r>0$ so dass $B_r(x_0)=\subseteq U$. Mann kann auch dieser Bälle benutzen um die Konvergenz zu definieren. 
\beispiel{Der Raum der Stetigen Funktionen} Sei $a<b\in\mathbb{R}$ Wir definieren $V=C([a,b])$ ein Vektorraum. Wir definieren dann die Norm $||f||_\infty=max_{x\in[a,b]}|f(x)|$ Für eine Stetige funktion $f\in V$ Dann ist $V$ ein Reeller Vektorraum, $||\cdot||$ ist eine Norm und die Konvergenz von $f_n\in V$ gegen $f\in V$ ist gleichbedeutend zur gleichmässige Konvergenz 
\beweis Angenommen $f_n\in V$ Konvergiert für $n\rightarrow \infty$ gegen $f \in V$. Diese aussage ist analog zu:
\[\forall \varepsilon >0\smspc \exists N\in \mathbb{N} \text{ so dass } \forall n>N \smspc d(f_n,f)=||f_n-f||_\infty<\varepsilon\]
Die unendlich norm hier ist definiert wie $\max_{[a,b]}(f_n-f)$ Also gilt $\forall x\in [a,b]$\[|f_n(x)-f(x)|\le||f_n-f||_\infty<\varepsilon\] Doch den $N$ haben wir vor den $x$ gewählt, daher ist dies auch gliechmässig konvergent.
\[\forall \varepsilon< 0\smspc \exists N\in \mathbb{N}\smspc \forall n< N\smspc\forall x\in[a,b]\text{ gilt } |f_n(x)-f(x)|\le\varepsilon\]
\bemerkung{} $||\cdot||_\infty$ ist eine Norm:
\begin{itemize}
  \item[i]{$||f||_\infty=0\Longleftrightarrow f=0$}
  \item[ii]{$||\lambda f||_\infty=max_{[a,b]}|s\cdot f(x)|=|s|\cdot max_{[a,b]}|f(x)|=|s|\cdot||f||_\infty$}
  \item[iii]{$f_1,f_2\in V$ dann ist $||f_1+f_2||_\infty=max_{[a,b]}\underset{=|f_1(x)|+|f_2(x)|}{\underbrace{|f_1(x)+f_2(x)|}}\le||f_1||_\infty+||f_2||_\infty$ (dies ist nicht klar aber es beweist die Dreiecksungleichung)}
\end{itemize}
\subsubsection*{Wichtige Normen und Metriken}
\begin{itemize}
  \item{Euklidische Norm: Die Vom Standard Inneres Produkt\[||v||=<v,v>=\sqrt{\sum_{i=1}^n v_i^2}\]}
  \item{Manhatten Norm: Grid city-norm: Sei $v=(v_1,v_2)$ dann ist die Manhatten norm durch: $||(v_1,v_2)||=|v_1|+v_2|$ definiert}
  \item{Supremum norm: $||f||_\infty=max_{x\in[a,b]}|f(x)$}
\end{itemize}
Analog zu jeder Norm, gibt es auch die dazugehörige Metrik.
\subsection{Topologische grundbegriffe}
\definition{} Sei $(X,d)$ ein Metrischer Raum. Eine Teilmenge $o\subset X$ heisst offen falls gilt: \[\forall x_0\in o\smspc \exists \varepsilon >0\text{ so dass } B_\varepsilon \subset o\]Eine Teilmenge $A\subseteq X$ heisst abgeschlossen falls $X\backslash A$ offen ist.
\beispiel{} Der Offene Ball ist wirklich offen:Sei ein Offenes Ball $B_r(x_0)\subseteq X$ dann gilt $\forall x\in B_r(x_0)$ $d(x,x_0)<r\longrightarrow \varepsilon =r-d(x,x_0)$ und dann ist $B_\varepsilon(x)\subseteq B_r(x_0)$ wass die vorherigfe definition anpasst.
\beispiel{} $\lbrace x_0\rbrace\subseteq X$ ist eine Abgeschlossene Menge: Wir mussen also zeigen dass $O=X\backslash \lbrace x_0 \rbrace$ offen ist. Sei $x\in O$ und $\varepsilon = d(x,x_0)>0$ dann gilt $B_\varepsilon(x)\subseteq O$
\lemma{} Sei $X$ eine Metrischer Ruam. \begin{itemize}\item{Jeder Endliche Schnitt von offenen Mengen ist offen.}\item{Jede Beliebige vereinigung von Offenen Mengen ist Offen und} \item{Jede Endliche Vereinigung von abgeschlossenen Mengen ist abgeschlossen.}\item{Jeder Belibieger Durchnitt von abgeschlossenen Mengen ist abgeschlossen.}\end{itemize}
\lemma{} Sei $X$ ein Metrischer Raum. Eine Teilmenge $O\in X$ ist Offen genau dann wenn fur jede Konvergente Folge $(x_n)\in X$ mit grenzwert $\lim_n{\rightarrow \infty} x_n\in O$  auch gilt dass fur alle bis auf endlich viele $n$ wir $x_n\in O$ haben.
\beweis Angenommen wir haben $O\subseteq X$ eine Offene Teilmenge und $x_n$ in $X$ mit $\lim_{n\rightarrow \infty}x_n=z\in O$ Dann $\exists \varepsilon >0$ mit $B_\varepsilon (z)\subseteq O$. Wegen $lim_{n\rightarrow \infty}x_n=z$ gibt es ein $N$ mit $x_n\in V_\varepsilon(z)\subseteq O$ $\forall n>N$.\newline
Angenommen, $O\subseteq X$ ist nicht offen, daher gilt\[\exists z\in O\text{ so dass } \forall \varepsilon>0\smspc B_\varepsilon (z)\not\subseteq O\] Wir setzen $\varepsilon=\frac{1}{n}$ und finden ein $x_n$ mit $x_n\in B_{\frac{1}{n}}(z)\backslash O$ es gilt $\lim_{n\rightarrow \infty} x_n=z\in O$ doch $x_n\not\in O\smspc \forall n\in \mathbb{N}_0$
\lemma{} Eine Teilmenge $A\subseteq X$ ist abgeschlossen genau dann wenn fur jede Folge $x_n$ in A mit $\lim_{n\rightarrow\infty}x_n\in X$ auch $\lim_{n\rightarrow \infty}n\in A$ gilt.
\beweis Wir nehmen an dass $A$ abgeschlossen ist. mit $x_n\in A$ fur $n\in \mathbb{N}$ und $\lim_{n\rightarrow\infty}x_n\in X$ Falls der Grenzwert $\lim_{n\rightarrow\infty}x_n\in X\backslash A$ dann würde aus dem ersten Teil folgen dass $x_n\in X\backslash A$ fur die aller meisten $n$. Doch wie in den ersten Teil des beweises davor, kann dies nicht gelten, also \[\Longrightarrow \lim_{n\rightarrow\infty}x_n\in A\]
\newline Falls aber $A$ nicht abgeschlossen ist so ist $O=X\backslash A$ nicht offen und wir können eine Folge $x_n\in A$ finden mit $\lim_{n\rightarrow\infty}x_n=z\in X\backslash A$. Der beweis ist mir nicht klar... $\approx 1$h 23.02
\bemerkung{} Sei $X$ ein Metrischer Raum und $Y\subseteq X$ eine Teilmenge. Dann ist auch $Y$ ein Metrischer Raum. Wir können die metrik $d:X\times X\rightarrow \mathbb{R}$ auf $d:Y\times Y\rightarrow \mathbb{R}$ einschränken, dann kann man von diesen $Y$ offene und abgeschlossen Mengen definieren.
\beispiel{} Sei $X=\mathbb{R}\supseteq Y=[-1,1]$ Wir können dann offene Bälle in $Y$ betrachten: $B^Y_1(1)=(0,1]\subset Y$ offen
\lemma{} Sei $X\supseteq Y$ wie vorhin, dann ist eine Teilmenge $O_Y\subseteq Y$ genau dann offen genau dann wenn es eine offene Teilmenge $O_X\subseteq X$ gibt mit $O_Y=O_X\cap Y$. Die gleiche Aussage gilt analog für abgeschlossene Mengen.
\beweis Angenommen $O_X\subseteq X$ ist offen. Wir definieren dann $O_Y=Y\cap O_X$ und zeigen dass $O_Y$ offen in $Y$ ist. Sei $y\in O_Y$ dann gibt es ein $\varepsilon >0$ mit $B_\varepsilon^X(y)\subseteq O_X$ und es gilt also
\[\Longrightarrow B_\varepsilon^Y(y)=B_\varepsilon^X(y)\cap Y\subseteq O_X\cap Y=O_Y\]
Der Zweite Teil ist dass wir die Offene Menge in $X$ bilden müssen. $O_Y\subseteq Y$ offen. $\forall Y\in O_Y$ $\exists \varepsilon_y$ mit $B_{\varepsilon_y}^Y(y)\subseteq O_Y$ da $O_Y$ von annahme offen in $Y$ ist. Wir definieren jetzt \[O_X=\bigcup_{y\in O_Y}B_{\varepsilon_y}^X(y)\]
Dies ist Offen in $X$. Behauptung, $O_Y=Y\cap O_X$ dann muss ich die Zwei richtungen dieser gleichung zeigen.
\begin{itemize}
  \item[\textit{i.}]{Sei $y\in O_Y$ dann gilt dass $y\in B_{\varepsilon_y}^Y(y)\subseteq O_X$ (klar da $y$ der Zentrum vom Ball $B_{\varepsilon_y}^X(y)$ ist) was die richtung $\subseteq$ beweist}
  \item[\textit{ii.}]{Sei umgekehrt $z\in Y\cap O_X$ also existiert ein $y\in O_Y$ mit $z\in B_{\varepsilon_y}^X(y)$ da so mein $O_X$ definiert ist. Aber mein $z$ ist ja auch in $Y$ und ich kann auch $z\in B_{varepsilon_y}^Y(y)\subseteq O_Y$ was dann die richtung $\supseteq$ beweist.}
\end{itemize}
Wenn dieses Lemma gilt sagt man das es Relativ offen ist.
\lemma Seien $A,O\subseteq \mathbb{R}$ nichtleer und nach oben beschränkt.\begin{itemize}
  \item{Falls $A$ abgeschlossen dann gilt $sup(A)\in A$}
  \item{Falls $O$ offen ist dann gilt $sup(O)\not\in O$}
\end{itemize}
\definition{Innere Punkte} Sei $X$ ein metrischer Raum und $B\subseteq X$ 
\begin{multicols}{2}
\begin{itemize}
  \item[\textit{i.}]{Wir sagen $x\in B$ ist eine Inneres Punkt falls es ein $\varepsilon > 0$ gibt mit $B_\varepsilon(x)\subseteq B$. }
  \item[\textit{ii.}]{Dass Innere von B ist definiert durch $B^\circ=\left\lbrace x|\exists \varepsilon > 0, B_\varepsilon(x)\subseteq B\right\rbrace$ dies ist die grösste Offene Menge die immer noch in $B$ platz hat.}
  \item[\textit{iii.}]{Wir Sagen $x\in X$ ist ein Randpunkt von $B$ falls $\forall \varepsilon > 0$ gilt $B_\varepsilon(x)\cap B\neq\emptyset\neq B_\varepsilon(x)\cap X\backslash B$ }
  \item[\textit{iv.}]{Wir definieren den Rand durch $\partial B=\left\lbrace x\in X|x\text{ ist ein Randpunkt}\right\rbrace$}
  \item[\textit{v.}]{Der Abschluss ist $\overline{ B}\cup \partial B=\left\lbrace x\in X|\forall \varepsilon >0:\smspc B_\varepsilon(x)\cap B\neq\emptyset\right\rbrace$ }
\end{itemize}
\vfill\null\columnbreak\begin{center}
\scalebox{2}{\tikzfig{mengenAnteile}}
\end{center}
\end{multicols}
\definition{Häufungspunkte} Sei $X$ ein Metrischer Raum, sei $B\subseteq X$ eine Teilmenge von $X$, wir sagen $x_0\in X$ ist ein Häufungspunkt der Menge $B$ Falls $\forall \varepsilon >0$ $\exists x\in B_\varepsilon(x)\cup B\backslash \lbrace x_0\rbrace$
Für eine Folge $(z_n)$ ist $x_0$ ein Häufungspunkt falls $\forall \varepsilon >0$ und $N\in\mathbb{N}$ $\exists n>N$ so dass $d(z_n,x_0)<\varepsilon$
\definition{Dichte Mengen}
Eine Menge $B\subset X$ heisst dicht falls $\forall x_0\in X$ $\forall \varepsilon >0$ $\exists x\in B$ mit $d(x,x_0)<0$ Die menge ist also dicht wenn der Abschluss der Ganze raum ist, oder wenn $B\cup H=X$ wobei $H$ die häufungspunkte sind.
\beispiel{Dichte Mengen}
\begin{itemize}
  \item[\textit{i.}]{$\left\lbrace a+b:\smspc a,b\in\mathbb{Q}\right\rbrace$ ist dicht in $\mathbb{C}$}
  \item[\textit{ii.}]{$\mathbb{Q}^n\subset\mathbb{R}^n$ ist dicht.}
\end{itemize}
\definition{Zusammenhängende Mengen}Sei $X$ ein Metrischer Raum, Eine Menge $B\subset X$ heisst abgeschloffen (clopen) falls die Menge Abgeschlossen und Offen in $X$ ist.
Der Metrische Raum $X$ heisst zusammenhängend falls $\lbrace\rbrace$ und $X$
die einzigen Abgeschloffene Mengen von $X$ sind.
\bemerkung{} $X\in\mathbb{R}$ ist genau dann zusammenhängend als eigenstädinger Metrischer Raum, wenn $X$ ein Intervall ist. Diese Proposition ist sehr wichtig, da sehr viele sachen in Analysis von Intervallen abhängen.
\beweis Wir nehmen zuerst an das $X$ kein Intervall ist, dann $\exists z\in\mathbb{R}\backslash X$ so dass $B_1=(-\infty,z)\cap X\neq \emptyset$ und $B_2=(z,\infty)\cap X\neq\emptyset$ Daraus folgt also dass $B_1$ und $B_2$ abgeschloffen sind, und $X$ ist nicht zusammenhängend. Wir haben gerade zwischen infimum und den supremum eine zahl gefunden, die nicht in der Menge drinn ist (es muss sie geben sonst ist es ein Intervall. Und da 
$B_1$ und $B_2$ komplemente voneinander sind, und beide offen sind, dann sind sie abgeschloffene Menge in $X$. \newline
Falls die Menge $X$ doch ein Intervall ist, und $B\subset X$ ist eine abgeschloffene Teilmenge, mit $B\neq\emptyset$ und $B\neq X$ Da $B$ nicht lehr ist $\exists a\in B$ und $\exists b\in X\backslash B$ wenn $a<b$ nicht gilt, dann wähle ich ein neues $B$ bis dies gilt. Da $B$ offen ist, gilt $\exists \varepsilon >0$ so dass $B_\varepsilon^X(a)\subseteq B$ Daher gilt $[a,a+\varepsilon)\subseteq (a-\varepsilon,a+\varepsilon)\cap X$
Wir definieren also $s=\sup(B\cap[a,b])=\sup(B\cup(B\cup(a,b))$ daher ist das $B=A\cup X$ und die Menge $A$ abgeschlossen in $\mathbb{R}$ ist. Daher sind $B\cap[a,b]$ und $B\cap (a,b)$ auch noch abgeschlossen. und wir finden dass $s\in B\cap [a,b]$ weil es offen ist, und $s\not\in B\cap(a,b)$ da es Offen ist, was ein widerspruch ist.
\lemma{} Sei $X$ ine Metrischer Raum, und $y_1,y_2\subseteq X$ zwei zusammenhängende Teilmengen, falls $y_1\cap y_2\neq\emptyset$ dann ist $y_1\cup y_2$ zusammenhängend.
\beweis Angenommen $B\subseteq y_1\cup y_2$ ist eine relativ abgeschloffene Teilmenge, sei $z\in y_1\cap y_2$. Wir nehmen an dass $B\neq \emptyset$ und $x_1\in B\cup y_1$ Dann gilt $B\cap y_1$ abgeschloffen ist, da $y_1$ zusammenhängend ist, gilt $v\cap y_1=y_1$, und insbesondere gilt also dass $z\in B$ und dass $B\cap y_2=y_2$ und ist $B=y_1\cup y_2$.
\subsection*{Stetigkeit}
\definition{Stetigkeit} Seien $X,Y$ zwei metrische Räume und $f:X\rightarrow Y$ eine Funktion, wir sagen dass $f$ auf $x_0$ $\varepsilon$-$\delta$ stetig ist falls \[\forall \varepsilon >0\smspc\exists \delta > 0\smspc\forall x\in X:d(x,x_0)<\delta\Longrightarrow d(f(x),f(x_0))<\varepsilon\]
\definition{Folgenstetigkeit} Man sagt dass $f$ Folgenstetig ist falls für jede FOlge $(x_n)\in X$ mit grenzwert $\lim_{n\rightarrow \infty} x_n=x_0$ auch $\lim_{n\rightarrow \infty} f(x_n)=f(x_0)$
\lemma{} Es gilt $f$ ist in $x_0\in X$ $\varepsilon$-$\delta$ stetig genau dann wenn $f$ in $x_0$ foglenstetig ist.
\beweis $\Longrightarrow$ angenommen $f$ ist $\varepsilon$-$\delta$ stetig und $x_n$ ist eine Folge mit $\lim_{n\rightarrow\infty}=x_0$. Sei $\varepsilon>0$, dann $\exists\delta >0$ so dass $d(x,x_0)\delta\Rightarrow d(f(x),f(x_0))<\varepsilon$. Da $\lim_{n\rightarrow\infty}x_n=x_0$ ist, gilt $\exists N$ so dass $d(x_n,x_0)<\delta$ $\forall n>N$. Es folgt also dass $d(f(x_n),d(x_0))< \varepsilon$ $\forall n>N$. Da $\varepsilon > 0$ beliebig war, gilt also $\lim_{n\rightarrow\infty}f(x_n)=f(x_0)$ Was dies Richtung beweist\newline
$\Longleftarrow$ Angenommen $f$ ist in $x_0$ nicht $\varepsilon$-$\delta$ stetig. Dann gilt \[\exists\varepsilon_0 >0\smspc\forall \delta >0\smspc \exists x_\delta \in X\text{ mit } d(x_\delta,x_0)<\delta\text{ aber } d(f(x_\delta)),f(x_0))\ge \varepsilon_0\]
wir setzten $\delta=\frac{1}{n}$ und finden eine Folge $x_n$ in $X$ mit $d(x_n,x_0)<\frac{1}{n}\rightarrow 0$ und $d(f(x_n),f(x_0))\ge\varepsilon_0$ also $\lim_{n\rightarrow\infty}x_n=x_0$ aber $\lim_{n\rightarrow \infty}f(x_n)\neq f(x_0)$ und daher ist $f$ auch nicht Folgenstetig ist.
\definition{} $f$ ist stetig falls es in jedem punkt $\varepsilon$-$\delta$ stetig ist.
\bemerkung{} Seien $X,Y$ metrsiche Räume und $f:X\rightarrow Y$ eine Funktion, äquivalent sind folgende aussagen:
\begin{itemize}
  \item[\textit{i.}]{ $f$ ist Stetig}
  \item[\textit{ii·}]{$f$ ist an jedem Punk folgenstetig}
  \item[\textit{iii.}]{Das urbild einer Offene Menge ist Offen , $O\subseteq Y$ ist offen, dann ist $f^{-1}(O)\subseteq X$ ist auch offen. (analog für abgeschlossen.)}
\end{itemize}
\textbf{Beweis} Angenommen, $f$ ist Stetig, in jedem punkt ist $f$ $\varepsilon$-$\delta$ stetig. Sei $O\subseteq Y$ offen, wir wollen zeigen dass $f^{-1}(O)\subseteq X$ dann auch offen ist. Sei also $x_0\in f^{-1}(O)$ dann gilt $f(x_0)\in O$ Dann gibt es ein $\varepsilon > 0$ mit $B_\varepsilon^Y(f(x_0))\subseteq O$ und daher $\exists \delta>0$ os dass $d(x,x_0)<\delta\Longrightarrow d(f(x),f(x_0))<\varepsilon$ also $f(x)\in O$ und $x\in f^{-1}(O)$ also gilt $B_\delta^X(x_0)\subseteq f^{-1}(O)$
$f^{-1}(O)$ ist also eine Offene Teilmenge von $X$.
\definition{} Eine Funktion $f:X\rightarrow Y$ heisst gleichmässig stetig falls \[\forall \varepsilon >0\smspc\exists \delta > 0\text{ so dass }\forall x_0,x_1\in X\mspc d(x_0,x_1)<\delta\Rightarrow d(f(x_0),f(x_1))<\varepsilon\]
\definition{Lipschitz Stetig} Eine funktion $f$ ist Lipschitz Stetig falls es eine konstante $L\ge 0$ gibt so dass \[\exists L\ge 0\smspc d(f(x_0),f(x_1))\le Ld(x_0,x_1)\mspc \forall x_0,x_1\in X\]
\definition{} Seien $X,Y$ metrische Räume, $x_0$ ein Häufungspunkt von $X$ und $f:X\backslash\lbrace x_0\rbrace\rightarrow Y$ eine funktion. Wir sagen $f$ Strebt gegen $y_0\in >$ für $x\rightarrow x_0$ ($\lim_{x\rightarrow x_0}f(x)=y_0$) falls \[\forall  \varepsilon>0\smspc\exists\delta >0\smspc\forall x\in X\backslash \lbrace x_0\rbrace\smspc d(x,x_0)<\delta\Rightarrow d(f(x),y_0)<\varepsilon\]
\lemma{Afwahrung der Stetigkeit}
Stetig sind folgende Funktionen
\begin{multicols}{2}
\begin{itemize}
  \item{$+:\mathbb{C}^2\rightarrow\mathbb{C}$}
  \item{$\cdot:\mathbb{C}^2\rightarrow\mathbb{C}$}
  \item{$(\cdot)^{-1}:\mathbb{C}^2\rightarrow\mathbb{C}$}
  \item{Polynome in $\mathbb{C}[t]$ }
  \item{$\exp(x)=e^x$}
  \item{$\sqrt{\cdot}$}
  \item{Alle Trigonometrische Funktionen}
\end{itemize}
\end{multicols}
\begin{itemize}
  \item{\textit{i.} Seien $X,Y,Z$ metrische Räume, Falls $f:X\rightarrow Y$ und $g:Y\rightarrow Z$ stetig sind, dann ist auch $f\circ g:\rightarrow Z$ Stetig}
  \item{\textit{ii.} Eine Funktion $f:X\rightarrow \mathbb{R}^m$ ist stetig genau dann wenn $\pi_j\circ f$ wobei $\pi_j$ die Projektionsfunktion ist, stetig ist $\forall j<n\in \mathbb{N}$}
\end{itemize}
\textbf{Beweis} Sei $f:X\rightarrow Y$ und $f:Y\rightarrow Z$ stetig, Für eine Offene Teilmenge $O\subseteq Z$ gilt dann dass $g^{-1}(O)\subseteq Y$ offen ist und dann $f^{-1}(g^{-1}(O))\subseteq X$ ist offen. Doch die Charakterisierung $(f\circ g)(O)\subseteq X$ ist offen und somit ist die charakteriesierung de rStetigkeit in der vorletzen bemerkung erfüllt.
\newline Sei nun $f:X\rightarrow \mathbb{R}^m$ eine Funktion, falls $f$ stetig ista, dann sind auch $\pi_j\circ f:X\rightarrow \mathbb{R}$ für jedes $j\in \left\lbrace 1,\cdots,m\right\rbrace$
\newline $"\Leftarrow"$ Angenommen dass $f$ $\pi_1\circ f,\cdots, \pi_m\circ f:\mathbb{R}\rightarrow \mathbb{R}$ stetig sind. Und sei $(x_n)$ eine Folge $x$ mit $\lim_{n \rightarrow\infty}x_n=x_0$ also folgt \[\lim_{n \rightarrow \infty}(\pi_j\circ f)(x_n)=\pi_j\circ f(x_0)\smspc \forall i\in \lbrace1,\cdots,m\rbrace\]
und daraus folgt dass $f(x_n)=\left(\pi_1 \circ f(x_1),\cdots ,\pi_m \circ f(x_n)\right)^t \rightarrow f(x_0)$ für $n \rightarrow \infty$ Und da dies für jede Konvergente Folge gilt, ist $f$ stetig.
\newline Das $\pi_i$ ist einfach die $i$-te Komponente der Vektorwertigen funktion.
\beispiel{} wir definieren $f:\mathbb{R}^2\rightarrow \mathbb{R}$ wie Folgt:
\[f(x,y)=\left\lbrace\begin{matrix}\frac{x^2y}{x^4+y^2}\text{ für }(x,y)\neq (0,0)\\0 \text{ für } (x,y)=(0,0)\end{matrix}\right.\]
Die Stetigkeit ist hier nicht ganz klar, da um die $(0,0)$ haben wir keinen Satz der uns sagt dass dies stetig ist. Wenn wir uns die Funktion anschauen, und $x$ gegn $0$ anschauen, und umgekehrt, $y$ gegen 0. Man brauch doch nur einen fall wo $\lim_{x \rightarrow 0}f(x,g(x))\neq 0$ und hier wenn wir $g(x)=x^2$ setzen, dann findet man dass es gegen $\frac{1}{2}$ strebt, und daher ist diese Funktion nicht stetig in $(0,0)$ zumindenst 
\bemerkung{} Um Stetigkeit zu beweisen muss man also durch verknüpungen und schon bewiesene Sätze prüfen, im fall von Fallunterscheidungen ist dies nicht ganz einfach.
\beispiel{}
Sei $X$ ein Metrischer Raum und $A \subseteq X$ eine Nichtleere Teilmenge. Wir definieren \[d(x,a)=\inf_{a\in A}d(x,a)\]
Also der kleinste abstand zwischen $x$ und ein Element in $A$ also ist $d(x,A)=0 \Leftrightarrow x\in A$\newline
Wir wollen jetzt zeigen dass $d(\cdot,A):X \rightarrow \mathbb{R}$ Lipchitz stetig ist mit Lipschitz konstante $1$
\beweis Seien $x_0,x_1\in X$ Für alle $a\in A$ gilt dass $d(x_0,a)\le d(x_0,x_1)+d(x_1,a)$ (dreiecksungleichung) und dass ganze ist $\ge d(x,A)$ und also $ d(x_0,A)-d(x_0,x_1)\le d(x_1,a)$ und mit herumschreiben:
\[\left| d(x_0,A)-d(x_1, A)\right|\le d(x_0,x_1)\] und da Kann man die LIpschitz konstante $1$ vor dem $d(x_0,x_1)$ ablesen.
\lemma{} Seien $X,Y$ metrische Räume und $f:X \rightarrow Y$ stetig. Falls $X$ Zusammenhängend ist, so ist auch $f(X)$ zusammenhängend.
\beweis Angenommen wir haben $B \subset f(X)$, relativ abgeschloffen. Dann $\exists O\subset Y$ offen mit $B=O\cap f(X)$ und dann betrachten wir $f^{-1}(B)$ und bemerken dass es genau gleich $f^{-1}(O)$ ist da die Punkte die nicht in $f(X)$ sind keinen urbild haben. Also $f^{-1}(B)=f^{-1}(O)$ ist offen da $f$ stetig ist. Zusätslich gibt es eine Abgeschlossene Menge $A\subseteq Y$ mit $B=A\cap f(X)$ mit $f^{-1}(A)=f^{-1}(B)$ ist abgeschlossen. Und daher ist dass Urbild abgeschloffen in $X$ woraus folgt dass $f^{-1}(B)=\emptyset$ oder $f^{-1}(B)=X$ und dann ist $B=\emptyset$ oder $B=X$ und die zeigt per definition dass $f(X)$ die ganze Menge ist (da $B \subseteq f(X)$).
\definition{Weg} Ein Weg in einen Metrischen Raum $X$ ist eine Stetige abbildung $\gamma:[a,b] \rightarrow X$ wobei $a<b$ reelle Zahlen sind, $\gamma(a)$ der Startpunkt und $\gamma(b)$ der Endpunkt des Weges ist. 
\bemerkung{Wegzusammenhängend} Wir nennen $X$ Wegzusammenhängend falls es $\forall x_0, x_1\smspc \exists$ ein Weg auf $[a,b]$ mit Startpunkt $x_0$ und Endpunkt $x_1$
\lemma{} Falls $X$ Wegzusammenhängend ist so $X$ auch zusammenhängend. So sind auch Offene Teilmengen $O\in \mathbb{R}^k$ zusammenhängend, genau dann wenn $O$ wegzusammenhängend ist.
\beweis (Fall \textit{2.}) Angenommen $O \subseteq \mathbb{R}^k$ ist offen und zusammenhängend. Wir wollen Zeigen dass $O$ auch wegzusammenhängend ist. ("$\Leftarrow$")\newline
Sei $x_0\in O$ beliebig. Wir definieren dann \[U=\left\lbrace x_1\in O\left|\text{Es gibt einen Weg von }x_0 \text{ nach }x_1\right.\right\rbrace\subseteq O\]
\begin{itemize}
\item[\textit{i.}]{$U$ ist offen:\newline angenommen $x_1\in U \subseteq O$ da $O$ offen ist, dann $\exists \varepsilon >0$ so dass $B_\varepsilon(x_1) \subseteq O$ Durch verknüpfen (Weg von $x_0$ nach $x_1$ und dann im ball von $x_1$ zum neuen Punkt.) von Wegen erhalten wir $B_\varepsilon (x_1)\subseteq U$}
\item[\textit{ii.}]{$U$ ist abgeschlesson, daher ist $O\backslash U$ ist auch offen, Diese Menge nennen wir $Q=O\backslash U$ per definition von $U$ gibt es für $x_1\in Q$ keinen Weg von $x_0$ nach $x_1$ dann gibt es im offenen Ball $B_{\varepsilon}(x_1)$ auch keinen weg, sonst könnten wir einfach von $x_0$ zu diesem Punkt in $B_{\varepsilon}(x_1)\subseteq O$ und dann von dort nach $x_1$ was einen Widerspruch gibt, also gitl $B_{\varepsilon} \subseteq Q$ und $U$ ist Abgeschloffen in $O$ }
\end{itemize}
Daraus folgt dass $O=U$ und $O$ ist Zusammenhängend.
\newline Der Erste Teil des lemmas ist Allgemein für jeden Metrischen Raum, nicht nur Offene gültig.
\newline Sei $x_0\in X$ fest gewählt, dann gibt es noch eine Vorassetzung dass es zu jedem $x\in X$ einene weg $\gamma:[a_x,b_x]\rightarrow X$ von $\gamma(a_x)=x_0$ nach $\gamma(b_x)=x$. Daraus folgt dass $\gamma([a_x,b_x])\subseteq X$ ist zusammenhängend:
\[X=\bigcup_{x\in X}\gamma([a_x,b_x])\] Und daraus folgt (irgendwie) dass $X$ zusammenghängend ist.
\section*{Vollständigkeit}
\definition{} Sei $X$ ein Metrischer Raum. Eine Folge $(x_n)$ in $X$ ist eine Cauchy-Folge falls \[\forall \varepsilon >0 \smspc \exists N\in \mathbb{N}\smspc \forall m,n \ge N:\smspc d(x_m,x_n)<\varepsilon\]
\definition{} Der Metrische Raum $X$ heisst vollständig falls jede Cauchy-Folge in $X$ einen Grenzwert in $X$ besitzt:
\lemma{} Eine Abgeschlossene Teilmenge $A \subseteq \mathbb{R}^d$ ist immer Vollständig.
\beweis Angenommen $x_n$ ist eine Cauchy folge in $A\subseteq \mathbb{R}^d$. Wenn wir die Koordinaten einzeln projezieren und anschauen sind sie auch Cauchy-Folgen, $(\pi_j(x_n))_n$ ist eine Cauchy-Folge in $\mathbb{R}$ $\forall j\in\lbrace 1,\cdots , d\rbrace$ $frall j\in\lbrace 1,cds , d\rbrace$ Daher existiert (1. semester) $z_j=\lim_{n\rightarrow\infty}\pi_j(x_n)$  und dann also $\lim_{n\rightarrow\infty}(x_n)=(z_1,\cdots,z_n)\in \mathbb{R}^d$ Da $A$ abgeschlossen ist 
und $x_1\in A$ folgt nun $z\in A$ was die Vollständigkeit impliziert.
\definition{Lipschitz kontraktion}
Eine Lipschitz kontraktion ist eine Abbildung $T:\mathbb{R}\rightarrow \mathbb{R}$
\[\exists \lambda<1\smspc \forall x_1,x_2\in X\smspc d(T(x_1), T(x_1))\le \lambda d(x_1,x_2)\]
\satz{Banachscher Fixpunktssatz}
Sei $X$ ein Vollständiger Metrischer Raum, und $T:X \rightarrow X$ eine Lipschitz-Kontraktion, Dann existiert ein eindeutig bestimmbarer Fixpunkt $z\in X$ mit $T(z)=z$.
\beweis Angenommen es gibt $2$ Fixpunkte, $y,z\in X$ mit $T(y)=y$ und $T(z)=z$ dann folgt dass $d(y,z)=d(T(y),T(z))\le \lambda d(y,z)$ da $\lambda <1$ ist muss $d(y,z)=0$ sein und $y=z$
Wir wissen jetzt dass es höchstens einen Gibt, wir mussen doch beweisen dass es ihn gibt.
\newline Sei $x_1\in X$ beliebig. Wir definieren rekursiv \[(x_n)_{n\in\mathbb{N}}:\mspc\left\lbrace\begin{matrix}x_2=T(x_1)\\x_3=T(x_2)\\\hdots\\x_{n+1}=T(x_n)\end{matrix}\right.\]
die Bahn des punktes $x_1$ \newline \textit{Beh. I}: Falls $x_n$ einen Grenzwert in $X$ besitzt, dann ist $T(z)=z$ , und daher auch \[T(z)=T(\lim_{n\rightarrow\infty}x_n)=\lim_{n\rightarrow\infty}\underset{=x_{n+1}}{\underbrace{T(x_n)}}=z\]
Also wenn es einen Grenzwert gibt, dann haben wir es gefunden.\newline
\textit{Beh. II} Falls $(x_n)$ eine Cauchy folge ist, dann ist es auch konvergent, es gilt also \textit{Beh. I} und der Satz ist bewiesen
\newline Wir müssen also nur beweisen dass $x_n$ eine Cauchy-Folge ist:
\beweis Wir betrachten \[d(x_2,x_3)=d(T(x_1),T(x_2))\le \lambda d(x_1,x_2)\] Und wir beweisen per induktion dass $d(x_n, x_{n+1})\le \lambda^{n-1}d(x_1,x_2)$, da wenn dies Gilt, dann ist \[d(x_{n+1},x_{n+2})=d(T(x_n),T(x_{n+1}))\le \lambda d(x_n, x_{n+1})\le \lambda \lambda^{n-1}d(x_1,x_2)=\lambda^{(n+1)-1}d(x_1,x_2)\]
Sei also $\varepsilon 0$ dann $\exists N\in \mathbb{N}$ so dass \[\frac{\lambda^{N-1}}{1-\lambda}d(x_1,x_2)<\varepsilon \text{ Da }\lambda <1 \text{ ist}\] Sei nun $n>m\ge N$ dann gilt also
\[d(x_n,x_m)\le d(x_n,x_{n-1})+d(x_{n-1},x_{n-2})+\cdots+d(x_m,x_{m+1})\le \left(\lambda^{n-2}+\lambda^{n-3}+\cdots +\lambda^m\right)d(x_1,x_2)\le \frac{\lambda^{n-1}}{1-\lambda}d(x_1,x_2)\le \varepsilon\]
Also ist $x_n$ eine Cauchy-Folge und daher ist der Satz bewiesen
\subsection*{Kompaktheit}
\definition{Kompaktheit} Ein Metrischer Raum $X$ heisst Folgenkompakt Falls jede Folge $x_n\in X$ eine Konfergente Teilfolge besitzt.
\definition{Beschränktheit} Ein Metrischer Raum $X$ heisst beschränkt Falls $X=B_R(x_0)$ für ein $x_0$ in $X$ und $R<0$
\lemma{} Sei $K \subseteq X$ eine Folgenkompakte Teilmenge eines Raumes $X$ dann ist $K$ beschränkt. Die Umkehrung geht nicht, falls $X=\mathbb{Q}$ dann ist $lbrace x\in Q| 0\le x \le 1\rbrace$ sowohl abgeschlossen und Beschränkt doch nicht kompakt.
\beweis Sei $(x_n)\in K$ eine Folge mit grenzwert $z=\lim_{n\rightarrow\infty}x_n\in X$ Dann existiert eine Teilfolge $x_{n_l}$ mit $\lim_{l \rightarrow\infty}y\in K$ dann ist $z=y\in K$ dies Zeigt, dank den Lemma über characterisierung von offen und Abgeschlossene Mengen, dass $K$ abgeschlossen ist.
\newline Sei $x_0\in K$ für jedes$K\neq B_n^K(x_0)$ dann $\forall n $ $\exists x_n\in K$ mit $d(x_n,x_0)>n$ und dies bedeutet dass es keine Konvergente Teilfolge ovn $x_n$, und die menge ist beschränkt und daher auch Kompakt. $\exists x_n\in K$ mit $d(x_n,x_0)>n$ und dies bedeutet dass es keine Konvergente Teilfolge ovn $x_n$, und die menge ist beschränkt und daher auch Kompakt. 
\theorem{Heine-Borel} Eine Teilmenge $K \subseteq \mathbb{R}^d$ ist kompakt genau dann wenn $K$ abgeschlossen und beschränkt ist.
\beweis Wir müssen noch abgeschlossen und beschränkt zeigen, da die andere richtung i mallgemeinen gilt und wir sie gerade bewiesen haben.
Sei also $K\\subseteq \mathbb{R}^d$ abgeschlossen und beschränkt und sei $(x_n)_n)$ eine Folge in $K$ dann ist $(\pi_j(x_n))_n$ eine beschränkte Folge in $\mathbb{R}$ für jedes $j\in \lbrace 1,\cdots, d\rbrace$
Wir können mithilfe der Resultate im ersten Semester eine Teilfolge finden, so dass $(\pi_j(x_{n_l}))_l$ konvergiert für jedes $j\in\lbrace 1,\cdots, d\rbrace$ daher dass $z=\lim_{l \rightarrow\infty}x_{n_l}\in \mathbb{R}^d$
Und da $K \subseteq \mathbb{R}^d$ abgeschlossen ist, folgt dass $z\in K$ und dass is gemäss die definition der Folgenkompaktheit. 
\definition{Überdeckungskompaktheit}
Sei $X$ ein Metrischer Raum, eine Offene überdeckung von $X$ ist eine Menge $\mathcal{O}$ von offenen Teilmengen von $X$ so dass \[X=\bigcup_{O\in \mathcal{O}}O\]
\newline $X$ heisst abzählbar Überdeckungskompakt falls jede abzählbare offene überdeckung von $X$ eien Endliche teilüberdeckung hat. Also wenn $\mathcal{O}=\lbrace O_1, \cdots O_n\rbrace$ endlich ist so dass $X=O_1\cup \cdots \cup O_n$
\beispiel{} Sei $X=(0,1)$ Dann ist $\mathcal{O}=\left\lbrace(\frac{1}{n},1):n\in \mathbb{N}\right\rbrace$ 
\satz{}
Für einen Metrischen Raum sind folgende Eigenschaften äquivalent
\begin{itemize}
\item[\textit{i.}]{ Jede Folge in $X$ hat eine Konvergente Teilfolge}
\item[\textit{ii.}]{Jede Unendliche Teilmenge $D\subseteq X$ hat einen Häufungspunkt in X.}
\item[\textit{iii.}]{Jede Stetige funktion $f:X \rightarrow \mathbb{C }$ ist beschränkt.}
\item[\textit{iv.}]{Jede Stetige Funktion $f: X \rightarrow \mathbb{R}$ nimmt Maximum und minimumwert ind $X$ an.}
\item[\textit{v.}]{$X$ ist abzählbar überdeckungskompakt.}
\end{itemize}
\end{document} 
