\documentclass{article}
\title{Lineare Algebra II}
\author{Benjamin Dropmann}
\usepackage{geometry}
\usepackage{xcolor}
\usepackage{tikzit}
\input{linAlg.tikzstyles} 
\usepackage{amssymb}
\usepackage{amsfonts}
\usepackage{amsmath}
\usepackage{multicol}
\usepackage{titlesec}
\titlespacing*{\subsubsection}{0pt}{1.2ex}{.1ex plus .2ex minus .2ex}

\newcommand{\mspc}{\hspace{0.7cm}}
\newcommand{\smspc}{\hspace{0.3cm}}

\newcommand{\satz}[1]{\subsubsection*{Satz {#1}}}
\newcommand{\korollar}[1]{\subsubsection*{Korollar {#1}}}
\newcommand{\beweis}{\\\textbf{Beweis }}
\newcommand{\beispiel}[1]{\subsubsection*{Beispiele {#1}}}
\newcommand{\bemerkung}[1]{\subsubsection*{Bemerkung {#1}}}
\newcommand{\theorem}[1]{\subsubsection*{Theorem {#1}}}
\newcommand{\lemma}[1]{\subsubsection*{Lemma {#1}}}
\newcommand{\definition}[1]{\subsubsection*{Definition {#1}}}
\newcommand{\behauptung}[1]{\subsubsection*{Behauptung {#1}}}

\geometry{margin=1.5cm}
\begin{document}
\maketitle
\section*{Polynome}
\subsection*{Polynomdivision}
Seien $f$ und $g\ne0$ zwei polynome in $K[x]$ dann $\exists q(r), r(r)\in K[x]$ mit $deg(r)=0$ oder $deg((r)<deg(g)$ und $f=qg+r$.
\korollar{9.0.4} Sei $f(x)\in K[x],f(x)=0$ sei $\lambda\in K$ so dass $f(\lambda)=0$. Dann $\exists q(x)\in K[x]$ so dass $ f(x)=(x-\lambda)q(x)$
\beweis $\exists q(x),r(x)\in K[x] \hspace{0.5cm}deg(r)<deg(x-\lambda)=1$ so dass $f(x)=(x-\lambda)(q(x)+r(x), \rightarrow r\in K\Rightarrow f(\lambda)=0$
\korollar{9.0.6} Sei $f(x)\in K[x], deg(f)=n>0$ Dann hat $f(x)$ höchstens $n$ Nullstellen. (Fundamentaler satz der Algebra sehr ähnlich).
\beispiel{9.0.7} Es sei $f(x)=x+1(x^2+1)$, als poly in $\mathbb{R}[x]$ hat es nur eine nullstelle $x=-1$.
Als polynom in $\mathbb{C}[x]$ gilt $f(x)=(x+1)(x+i)(x-i)$
\theorem{9.0.8 Fundamentaler Satz der Algebra} Es sei $f(x)\in \mathbb{C}[x], deg(f)=n>0$ dann hat $f(x)$ in $\mathbb{C}[x]$ genau $n$ nullstellen. Dass heisst es existieren $\exists \lambda_1,...,\lambda_n$ nicht unbedingt verschieden, so dass $f(x)=(x-\lambda_1)\cdot\left.\right.\cdots\left.\right.\cdot(x-\lambda_n)$ Wir sagen $\mathbb{C}$ is Algebraisch abgeschlossen.
\lemma{9.0.11} sei $f(x)\in K[x], \lambda\in K$ so dass $f(\lambda=0$ Die Ordnung der Nullstelle (Vielfachheit) $\lambda$ is die Ganze zahl $n\ge1$ so dass $\exists q(x)\in K[x]$ so dass \[f(x)=x-\lambda)^nq(x)\]
\beispiel{9.0.12}\begin{itemize}
\item[1.]{$f(x)=x+1(x^2+1) $ Einfache nullstelle $\lambda=-1$ daher ist die ordnung $1$}
\item[2.]{$p>2\hspace{0.5cm}g(x)=x^p\in\mathbb{F}_p[x]\mspc$ }
\end{itemize}
$\mathbb{F}_p=[a_nx^n+...+a_1x+a_0|n\ge0,a_i\in\mathbb{F}_p]$ Und $g(x)=x^p-1=(x-1)^p$ (leicht ausrechnen)
bemerkung{9.0.13} Analogien $\mathbb{Z}\leftrightarrow K[X]$
\begin{center}
\begin{tabular}{c|c}
	$\mathbb{Z}$&$K[x]$\\\hline
	$\pm1$&$K\backslash 0$\\
	Primzahlen&Unzerlegbare Polynome grad$<$0\\
	$\mathbb{Z}/_{p\mathbb{Z}}=\mathbb{F_p}$&$f(x)$ ist unzerlegbar: $K[x]/_{f(x)}$ Körper
\end{tabular}
\end{center}
\section*{Eigenwerte und Eigenvektoren}
\definition{10.1.1} $V/K$ Vektorraum, $T:V\rightarrow V$ Endomorphismus.
\begin{itemize}
\item[1.]$\lambda\in K$ ist ein Eigenwert von T wenn $\exists v\in V, v\neq 0_v$ so dass $T(v)=\lambda v)$
\item[2.]Ein solches V heisst Eigenvektor  mit Eigenwert $\lambda$
\end{itemize}
\bemerkung{10.1.12} Wenn v Eigenveltor von T ist, $T(v)=\lambda v$ dann ist auch $\alpha v$ Eigenvektor von T mit Eigenwer $\lambda, \forall \alpha\in K, \alpha \neq0$
\beispiel{10.1.3} Rechnung von eigenwerte und Eigenvektoren
\begin{itemize}
\item[1.]{$A= \begin{pmatrix}1&2\\2&1\end{pmatrix}$ Eigenwerte $\lambda = 3$ und $\lambda=-1$
\[A\cdot\begin{pmatrix}x\\b\end{pmatrix} = \lambda\cdot\begin{pmatrix}x\\b\end{pmatrix}\]
Wir kommen dann auf
\[\begin{pmatrix}1x&2y\\2x&1y\end{pmatrix}=\lambda\cdot\begin{pmatrix}x\\b\end{pmatrix}\]
und also
\[2x+y=\lambda x\]
\[x+2y=\lambda y\]
Wir bekommen also
\[y((1-\lambda)^2-4)=0\]
$y\neq0, x\neq0$ Da die nullvektoren keine Eigenvektoren sind $\Rightarrow (1-\lambda)^2=4\Rightarrow \lambda=[-1,3]$ Warum spezifisch zwei?}
\item[2.]{$B=\begin{pmatrix}1&-2\\1&4\end{pmatrix}$
 Wir Suchen ein $\lambda$ sodass $b(v)=\lambda\cdot v$ für $v\in \mathbb{R} ^2, v\neq0$
 \[ \left(B-\lambda\begin{pmatrix}1&0\\0&1\end{pmatrix}\right)v=\begin{pmatrix}0\\0\end{pmatrix}\] Alsow für welche $\lambda$ ist $B-\lambda\begin{pmatrix}1&0\\0&1\end{pmatrix}$ nicht invertierbar (wann ist der kern nicht trivial) $\Leftrightarrow$ Für welche $\lambda \in K$ ist $det\left(B-\lambda\begin{pmatrix}1&0\\0&1\end{pmatrix}\right)=0?$
 \[det\left(\begin{pmatrix}1-\lambda&-2\\1&4-\lambda\end{pmatrix}\right)=(1-\lambda)(4-\lambda)\Rightarrow \lambda =[2,3]\]
 Und jetzt fur die Eigenvektoren: für $\lambda = 2$
 \[b(v)=2v\Rightarrow v=\alpha\begin{pmatrix}-1\\2\end{pmatrix}, \alpha\neq0\]}
\end{itemize}
\satz{10.1.4} $T:V\rightarrow V$ linear. Dann gilt: $\lambda \in K$ eigenwert von $T\Leftrightarrow ker(T-\lambda1_v)=0$

\section*{Eigenwerttheorie}
\textbf{Fibonaccifolgen} sei $V$ der V-R der Fibonnacci Folgen. wir haben $S:V\rightarrow V$ ist die Verschiebungsabbiildung, (die ist definiert in satz 1.1.15) Die Basis war $B=\left\lbrace\mathbb{F}_{0,1}, \mathbb{F}_{1,0}<\right\rbrace$
Und die matrix ist $[S]^B_B=\begin{pmatrix}0&1\\1&1\end{pmatrix}$ und $det(S)=\lambda^2-\lambda-1$ eigenwerte sind also $\phi und \varphi$ und die Eigenfolgen sind $\left\lbrace\mathbb{F}_{\phi,1}, \mathbb{F}_{\varphi,0}<\right\rbrace$ also die diagonal matrix ist dann $[S]^C_C=\begin{pmatrix}\phi&0\\0&\varphi\end{pmatrix}$
\definition{Das charakteristische polynom} Sei $A\in M_{m\times n}(K)$ Dann ist $X_A(x)=det(A-x\mathbb{a}_n)$ das charakteristische polynom von A
\beispiel{10.2.2} $A=\begin{pmatrix}a&b\\c&d\end{pmatrix}$ dann ist $X_a(x)=ichhabe nicht abgeschrieben$ aber der konstante term des carachteristischen polynom ist die Determinante.
$det(A-x1_n)$ Insbesondere $X_{1_2}(x)=x^3-2x+1=(x-1)^2$ 

\definition{10.2.3} $T:V\rightarrow V$ linear dann sei $X_T(x)=det([T]_B^B-x1_n)$ dies ist unabhängig von der wahl der Basis $B$. 10.2.4: $X_T(x)$ ist wohldefiniert\beweis $[T]_c^C=[D]^B_C[T]_B^B[D^{-1}]^C_B$ danns ist \[det([T]_C^C-1_nx) =det([D]^B_C[T]_B^B[D^{-1}]^C_B-1_nx)=det(D[T]^B_BD^{-1}-xDD^{-1})\]\[=det(D([T]^B_B-xT)D^{-1})=det(D)det([T]^B_B-x)det(d^-1=)=\det(D)det([T]^B_B-x)\]
\theorem{10.2.5} Es sei $T:V\rightarrow V$ linear. Dann gilt dans die Eigenverte von $T=\left\lbrace\lambda \in K|X_T(\lambda)=0\right\rbrace$

\lemma{10.2.6}Sei $A=(a_{ij})\in M_{n\times n}(K)$ eine obere Dreiecksmatrix fann gilt \[X_A(x)=\Pi_{n=1}^n(a_{ii}-x)\]
Sei $M=\begin{pmatrix}a&b\\c&cd\end{pmatrix} \Rightarrow X_A=x^2-(a+d)x+ad-bc$
 Trace (noch nachzu sehen) $Tr:M_{n\times n}(K)\rightarrow A=(a_{ij})\rightarrow\sum a_{ii}1$
\definition{10.2.7} Sei $T:V\rightarrow V$ linear dann ist die Spur von $T$ \[Tr(T)=Tr([T]^B_B)\]
\korollar{10.2.8} $Tr(T)$ ist wohldefiniert
\beweis Zu zeigen wann $C$ eine Andere Basis un $D=id]^C_B$ dann gilt \[Tr([T]^B_B)=Tr(D^{-1}[T]^C_CD)\]
Es reicht aus zu zeigen dass wenn $M_1,M-2\in M_{n\times n}(K)$ dann gilt $Tr(M_1M_2)=Tr(M-2M_1)$ (mit explizite rechnung beweisen)
Daher gilt auch 10.2.8
\textbf{Satz10.2.9} es sei $T:v\rightarrow V$ linear dann gilt \[X_T=(-1)^nx^n+(-1)^{n-1}x^{n-1}Tr(T)+...+det(T)\]
\beweis es sei $A=[^B_B]$ Mit induktion kann man beweisen dass wenn es für eine $M_{n-1\times n-1}$ geht dann geht es für $M_{n\times n}$ als übung  zu machen.
Der Zweite beweis geht wie folgt ab:

Sei $B\in M_{n\times n}$ und $b=(b_{ij})$ dann gitl die formel
\[\sum_{\sigma\in S_n}b_{\sigma(1,1)}....b_{\sigma(n,n)}\]
Sei $B=A-x1_n$ und $\sigma\in S_n$ Fur welche $\sigma$ hat \[b_{\sigma(1,1)}b_{\sigma(2,2)}....b_{\sigma(n,n)}\] ein polynom von grad $>$n-1?
Der beweis ist todlich, nacheher schauen ich tippe jetzt was ich nicht verstehe...

$T:V\rightarrow V$ linear, dann ist $\lambda \in K$ eine Eigenvector wenn $\exists v\in V,v\neq 0_v$ so dass $Tv=\lambda v$. Hier merken wir dass der skalar eines Eigenvektors, auch ein egeinvektor ist, und dass die addition von zwei vektoren mir den selben eigenwert, auch ein Eigenvektor ist, also hat dies die Struktur eines unterraums...
Wir sind auf dem Folgenden Satz gekommen. Sei $T:V\rightarrow V$ linear, dann gilt $\lambda \in K$ ist genau dann Eigenwert von $T$ wenn $ker(T-\lambda I_n)\neq\left\lbrace\emptyset\right\rbrace$
\beweis $\lambda \in K$ Eigenwert $\Leftrightarrow \exists v\in V, v\neq 0_v$ so dass $Tv=\lambda v\Leftrightarrow (T-\lambda I_n)v=0_v$ Und daher ist $v\in ker(T-\lambda I_n)$ 
 Das ist Praktisch da wenn $(T-\lambda I_n$ nicht injektiv ist dann ist $ker(T-\lambda I_n)\neq \emptyset$ und wenn die Determinante nicht null ist dann ist $T-\lambda I_n$ kein endomorphismus.
 \bemerkung{} 0 ist ein Eigenwert wenn $T$ kein isomorphismus ist
\korollar{} Folgende aussagen sind äquivalent: \begin{itemize}
\item{$\lambda$ ist ein Eigenwert von $T$}
\item{$ker(T_\lambda I_n)\neq =0_v$}
\item{$T-\lambda I_n$ ist kein Isomorphismus}
\item{$det(T-\lambda I_n)=0$}
\end{itemize}
Der Beweis ist eine zusammenfassung von vorherigen beweisen
 Mit dieses wissen kann man Finden dass es hochstens $n$ Unterschliedliche eigenwerte gibt, da die mit einen grad $n$ polynom definiert sind.
\section*{Das charachteristische polynom} \definition{10.2.1} Sei $A\in M_{n\times n}(K)$. dann ist $X_a(x)=det(A-x1_n)$ das charakteristische polyom von A
Für eine $2\times 2$ Matrix ist dann \[X_A(X)=x^2-\underset{Tr(A)}{\underbrace{(a-d)}}x+\underset{det(A)}{\underbrace{ad-bc}}\]
Kleine errinerung, die Trace ist die Summe der Diagonale elemente. Diese bemerkung gilt auch für $3\times3$. Wir rechnen jetzt für $n\times n$. Der Konstante term von $det(A-xI_n)$ ist $det(A)$ (da es der Fall bei $x=0$ ist)

Insbesondere:
\[X_1(x)=x^2-2x+1=(x-1)^2\]
\definition{10.2.3} $T:V\rightarrow V$ linear dann ist $X_T(x)=det([T]_b^b-xI_n)$ Für eine Basis $B$ von $V$.
\lemma{10.2.4} $X_T(x)$ ist wohldefiniert.
\beweis
\[[T]^C_C=[D]^B_C[T]^B_B[D^{-1}]^C_B\] 
Multiplikativität von det:
\[det([T]^C_C-xI_n)=det([D]^B_C[T]^B_B[D^{-1}]^C_B-xI_n)=det([D]^B_C[T]^B_B[D^{-1}]^C_B-xD^{-1}D)=det(D)det([T]^B_B-xI_n)det(D^{-1})\]
Was unsere  aussage zustimmt.
 Da  das  Charakteristische  Polynom unabhängig von der Wahl der Basis, ist sie Eindeutig und daher Wohldefiniert.
\theorem{10.2.5} Es sei $T:V\rightarrow V$ linear, dann gilt dass\[\left\lbrace\text{Eigenwerte von }T\right\rbrace=\left\lbrace\lambda\in K|X_T(\lambda)=0\right\rbrace\]
\lemma{10.2.6} Sei $A=(a_{ij})\in M_{n\times n}(K)$ Eine Obere Dreiecksmatrix. Dann ist das Charakteristische Polynom \[X_A(x)=\Pi_{i=1}^n(a_{ii}-x)\]
$Tr:M_{n\times n}(K)\rightarrow K\mspc A=(a_{ij}\rightarrow \sum a_{ii})$ Ist wohldefiniert.
\definition{10.2.7} Sei $T:V\rightarrow V$ linear  dann ist $Tr(T)=Tr([T]^B_B)$ Wobei $B$ eine Basis, Wohldefiniert (\textbf{Satz 10.2.8}).
\beweis Zu Zeigen, wenn $\mathbb{C}$ eine andere Basis ist, und  $D=[id_v]_\mathbb{C}^B$ eine Basiswechselmatrix ist, dann gilt: \[Tr[T]^B_B=Tr(D^{-1}[T]^\mathbb{C}_\mathbb{C})\] Hier Bleibt nichts übrig ausser es auszurechnen, aber es funktioniert, es
reicht aus  zu zeigen, Wenn $M_1,M_2\in M_{n\times n}(K)$ dann gilt $Tr(M_1\cdot M_2)=Tr(M_2\cdot M_1)$ Da wenn dass gilt dann kürzt sich der $D, D^{-1}$. Dass ist eine Explizite berechung.
\satz{10.2.9} Sei $T:V\rightarrow V$ linear, dann gilt \[X_T(x)=(-1)^nx^n +(-1)^{n-1}Tr(T)x^{n-1}+\cdots+det(T)\]
 \beweis Es sei $A=[T]^B_B$
\satz{10.2.8} $Tr(T)$ ist wohldefiniert.
\beweis Wenn $C$ eine andere basis ist und $D=\left[id_v\right]^B_C$ dann gilt: $Tr[T]^B_B=Tr(D^{-1}[T]^C_CD)$
 Hier bleibt in theorie nichts anderes als von hand zu zeigen dass $M_1,M_1\in M_{n\times n}(K)$ dann gilt: $Tr(M_1M_2)=Tr(M_2M_1)$. Aber es ist immer noch nicht sehr schon.
\satz{10.2.9} Es sei $T:V\rightarrow V$ linear dann ist \[X_T(x)=(-1)^nx^n+(-1)^{n-1}Tr(T)x^{n-1}+\cdots+det(T)\] Uber den rest kann man nicht viel sagen\beweis
Es sei $A=[T]^B_B$ dann ist $X_A(x)=det(A)$ Aber die $A$ matrix ist sehr gross, dann muss man den beweis per induktion machen (Gute exams aufgabe). Hier ist die zweite idee die wir machen Wir wissen dass $B\in M_{n\times n}(K)$ dann gilt 
\[det(B)=\sum_{\sigma\in S_n}sgn(\sigma)b_{\sigma(1),1}\cdot\cdots b_{\sigma(n),n}\] Sei $B=A-xI_n$ und $\sigma\in S_n$, für welche $\sigma$ ist $b_{\sigma(1),1}\cdot b_{\sigma(2),2}\cdots b_{\sigma(n),n}=(*)$ ein Polynom vom Grad $\ge n-1$?
wenn $\sigma=id$ dann ist \[(*)=(a_{1,1}-x)\cdots(a_{nn}-x)=(-1)^nx^n+(-1)^{n-1}\underset{=Tr(B)}{\underbrace{(a_{1,1}+a_{2,2}+...+a_{n,n})}}x^{n-1}+\text{Restterm von grad <n-1}\] Alle andere moglichkeiten für $\sigma$ müssen also vom grad $<n-1$ sein (da nur auf der Diagonale $a_{j,j}-x$ steht, uberall sonst gibt es kein $x$ und wenn wir nur ein element vertauschen, sind es zwei, und daher ist grad $<n-1$), und daher ist dass zweite vorfaktor vom polynom
Welches dann beweist dass der zweite Restterm $Tr(A)$ ist und also dass unsere gleichung stimmt (der konstante faktor muss ja $=det(A)$ sein)
\korollar{10.2.11} $T:V\rightarrow V$ mir $V$ n-dim hat hochstens n Eigenwerte (da der Charachteristische polynom grad $n$ ist.)
\subsection{Diagonalisierung} Frage: es sei $T:V\rightarrow V$ ein Endomorphismus. Gibt es eine Basis in welche die abbildungsmatrix von $T$  diagonal ist?
\satz{10.3.2} Es seien $\lambda_1,\cdots,\lambda_n$ verschieden eigenwerte von $T$ und $\forall i$ sei $v_i$ ein Eigenvektor mit eigenwert $\lambda_i$ dann sind $v_1,\cdots,v_m$ linear unabhängig.
\beweis Es sei zwei Eigenvektoren, $v_a, v_b$ mit eigenwerte $\lambda_a,\lambda_b$ dabei ist dann $Tv_a=\lambda_av_a$ und $Tv_b=\lambda_bv_b$ wenn aber $v_a=cv_b$ (sie sind nicht linear unabhängig) dann gilt $Tcv_b=\lambda_acv_b$ und damit ist $\lambda_a\cdot c=\lambda_a$ und also sind diese Eigenvektore nicht unterschiedlich, da sie beide den selben Eigenwert haben.

\korollar{10.3.4} Wenn Wir für $T:V\rightarrow V$ linear mit $V$ n-dim, wenn $T$ Genau $n$ verschidene Eigenwerte hat, dann hat $V$ eine Basis die aus $\lambda_1,\lambda_2,\cdots,\lambda_n$ besteht.
\definition{10.3.5} $T:V\rightarrow V$ ist diagonalisierbar wenn $\exists$ Basis von Eigenvektoren existiert.
In diesem Fall ist die Abbildungsmatrix von $T$ bezüglich dieser Basis diagonal, mit den Eigenwerte als einträge in der Matrix.
\bemerkung{10.3.6} Eine $A\in M_{n\times n}$ Matrix ist diagonalisierbar $\Leftrightarrow$ $\exists B\in GL_n(K)$ so dass $B^{-1}AB$ diagonal ist (basiswechselmatrix).
 \lemma{10.3.7} Wenn $A$ Diagonalisierbar mit Eigenwerten $\lambda_1,\cdots,\lambda_1$ ist, dann ist $X_A=\Pi(\lambda_i-x)$

Charachterische Polynom ist: $X_A=det(A-xI_n)$ und seine losungen sind die Eigenwerte der Matrix.
Eine $n$-dim Matrix ist diagonalisierbar falls es $n$ unterschiedliche Eigenwerte gibt, daher wenn es eine Basis von Eigenvektoren gibt. Wir wissen auch dass
\[A=[T]^B_B\Leftrightarrow\exists P\in GL_n(K) \text{ so dass } P^{-1}AP \text{ Diagonal ist }\]
Frage, für welche $A$ gibt es so ein $P$?\begin{itemize}
  \item{Wenn $A$ diagonal ist dann ist $P$ die identität.}
  \item{Wenn $X_A(x)$ $n$ verschiedene Nullstellen hat, beachte, $\begin{pmatrix}1&0\\0&1\end{pmatrix} aber X(x)=(1-x)^2$ also diese bedingung ist nicht ausschlieslich.}
\end{itemize}
Gibt es matizen die Nicht diagonlisierbar sind?
\beispiel{10.3.8} \begin{itemize}
  \item{$A=\begin{pmatrix}1&1\\0&0\end{pmatrix}\Rightarrow X_A(x)=x^2\Rightarrow A$ hat nur einen Eigenwert, $\Rightarrow\begin{pmatrix}0&1\\0&0\end{pmatrix}\begin{pmatrix}x\\y\end{pmatrix}=\begin{pmatrix}0\\0\end{pmatrix}\Rightarrow$ Eigenvektoren sind $\alpha \begin{pmatrix}1\\0\end{pmatrix}$ Daraus kann man aber keine Basis machen, dies ist nicht diagonalisierbar.}
  \item{Es kann auch am Korper liegen dass wir nicht diagonaliseren konnen: $M=\begin{pmatrix}0&-1\\1&0\end{pmatrix}\in M_{2\times2}(\mathbb{R})\Rightarrow X_M(x)=x^2+1$ dass konnen wir nicht in $\mathbb{R}$ faktorisieren, aber in $\mathbb{C}$ geht es mit Eigenwerte $\pm i$, Wir werden immer den Korper vergrossern so dass dieser Fall nicht aufkommt}    
  \end{itemize}
  \beispiel{10.3.9:} der Erste Fall in der Liste lässt sich verallgemeinern, Sei $n\ge1, \lambda \in K$ Wir definieren die \textbf{Jordansche Blockmatrix} \[J_n(\lambda)=\begin{pmatrix}
\lambda & 1 & 0 & \dots & 0 \\
0 & \lambda & 1 & \dots & 0 \\
0 & 0 & \lambda & \ddots & \vdots \\
\vdots & \vdots & \ddots & \ddots & 1 \\
0 & 0 & \dots & 0 & \lambda
\end{pmatrix}\]
Und wir merken also dass $X_{J_n}(x)=(\lambda-x)^n$ Wobei der einzige Eigenwert $x=\lambda$ und die Dazugehorigen EigenVektoren sind dann $\alpha \begin{pmatrix}1\\0\\\vdots\\0\end{pmatrix}$ was natürlich für $n>1$ keine Basis.

\textbf{Folgerungen} Dass Charakterische allein entscheidet nicht ob eine Matrix diagonalisierbar ist. Und dass Problem ist eine Mogliche Diskrepanz zwischen der Ordnung der Nullstelle und die Dimension des aufgespannten Unterraums der Eigenvektoren.
\subsection*{Eigenräume}\definition{10.4.1} Sei $T:V\rightarrow V$ linear und $\lambda$ ein Eigenvektor von $T$. Der Eigenraum, ist der Aufgespannte unterraum vom $\lambda$-Eigenvektor, seine Defintion ist wie Folgt $E_\lambda=ker(T-\lambda id_v)=<\lambda\text{Eigenvektoren}>$
\lemma{10.4.2} $E_\lambda\subset V$ Beweis trivial.
\textbf{10.4.3}\begin{itemize}
  \item{$A=\begin{pmatrix}0&1&1\\1&0&1\\1&1&0\end{pmatrix}$ Dann ist $X_A(x)=-x^3+3x+2$ und dann sind die Eigenwerte $X_A(2)=0$ und dann konnen wir Faktorisieren und es kommt $X_A(x)=-(x-2)(x+1)^2$ und die Dimensionen der Dazugehorigen Eigenräume sind:\[A\begin{pmatrix}x\\y\\z\end{pmatrix}=2\begin{pmatrix}x\\y\\z\end{pmatrix}\Rightarrow E_{\lambda=2}=\left<\begin{pmatrix}1\\1\\1\end{pmatrix}\right>\]
    Und mit $E_{\lambda=-1}=\left<\begin{pmatrix}1\\1\\-2\end{pmatrix},\begin{pmatrix}1\\-2\\1\end{pmatrix}\right>$}
\end{itemize}
Aber mit eine Riesen matrix ist es schwierig zu sagen ob wenn wir alle Eigenräume zusammenstellen, wir eine Basis von $V$ haben, oder nicht. 
\definition{10.4.4} Sei $V$ ein V-R, wir betrachten $U_1, ..., U_k\subset V$ Sei $W=U_1+\cdots U_k$ Dann ist $W$ die Direkte summe von $U_1,...,U_k$, wenn \[\forall w \in W\smspc \exists!u_1\in U_1,..., u_k\in U_k\text{ so dass } w=u_1+...+u_k\] Man schreibt $W=U_1\bigoplus...\bigoplus U_k$
 Ich glaube dies ist äquivalent zu $\bigcap U_i=\left\lbrace0_v\right\rbrace$ Der beweis ist schwierig.
\lemma{10.4.6} Es gilt $W=U_1\bigoplus...\bigoplus U_k$ genau wenn die Gleichung $u_1+...+u_k==_v$ mit $u_i\in U_i \smspc \forall i$ nur die Losung $u_i=0_v\smspc\forall i$ hat. Der beweis ist als übung zum Leser überlassen
\beispiel{10.4.7}\begin{itemize}
\item{$\mathbb{R}^3=\left<\begin{pmatrix}1\\-1\\0\end{pmatrix}\right>\bigoplus\left<\begin{pmatrix}2\\-1\\0\end{pmatrix}\right>\bigoplus\left<\begin{pmatrix}0\\0\\1\end{pmatrix}\right>$ Dies wäare äquivalent zu sagen dass diese drei elemente eine Basis von $\mathbb{R}^3$ sind also ja}
\item{$\mathbb{R}^2=\left<\begin{pmatrix}1\\-1\end{pmatrix}\right>+\left<\begin{pmatrix}2\\-1\end{pmatrix},\begin{pmatrix}-3\\1\end{pmatrix}\right>$ aber Keine Direkte summe da die zweite lineare Hülle unnotige elemente enthält}
  
\end{itemize}
\bemerkung{10.4.8} Wenn $W$ die Direkte Summe von $U_1....U_k$ ist dann gilt dass $\dim(W)=\dim(U_1)+...\dim(U_k)$
\beweis Sei $B_i$ Basis von $U_i$ dann behaupten wir dass $B_1\cup....\cup B_k$ Basis von $W$ ist. Dieser Teil des Beweis ist als Ubung überlassen 
\satz{10.4.9} Es sei $T:V\rightarrow V$ linear und $\lambda_1,...,\lambda_k$ Eigenwerte von $T$ mit $\lambda_i\neq\lambda_j\forall i\neq j$. Sei $W =E_{\lambda_1}+...+E_{\lambda_k}$ Dann gilt $W=E_{\lambda_1}\bigoplus...\bigoplus E_{\lambda_k}$.
\beweis Nehmen wir an dass $\exists u_1,...,u_k\smspc u_i\in E_{\lambda_i}$ und dann da $u_i$ jeweils in unterschiedliche Eigenräume sind, sind die alle von einander linear unabhängig, kann die summe den Nullvektor ergeben:\[\exists u_1,...,u_k\in E_{\lambda_i}\text{ so dass } u_1+...+u_k=0_v\]
Doch $u_1,...,u_k$ sind linear unabhängig und wenn $u_i\neq 0_v\smspc \forall i$ dann kriegen wir ein widerspruch.
 \korollar{10.4.10} Sei $T:V\rightarrow V$ linear mit Eigenwerte $\lambda_1,...,\lambda_k$ dann ist $T$ genau dann Diagonalisierbar, wenn die summe der dimensionen der dazugehorigen Eigenräume, die dimension von $V$ ist: \[T \text{ ist Diagonaliserbar }\Leftrightarrow\dim(V)=\sum_{i=1}^k\dim(E_{\lambda_i})\] 
\subsection*{Algebraische und Geometrische vielfachheit}
\bemerkung{10.5.1} Es sei $n=\dim_K(V)$ mit $T:V\rightarrow V$ Dann hat $X_T(x)$ grad $n$ und wenn $X_T(x)=(\lambda_1-x)^{a_1}\cdot...\cdot (\lambda_k-x)^{a_k}$ mit $\lambda_i\neq\lambda_j\forall i\neq j$ dann ist $n=\sum a_i$
\definition{10.5.2} sei $\lambda$ Eigenwert von $T$ dann ist \begin{itemize}
  \item{Die Geometrische Vieflachheit; $g_\lambda=\dim(E_\lambda)$}
  \item{Algebraische Vielfacheit $a_\lambda$ ist die Ordnung der Nullstelle vom Faktor $\lambda$ in $X_T(x)$}
\end{itemize}
\beispiel{10.5.3} Im beispiel 10.4.3 hatten wir \begin{itemize}\item{$\lambda_1=-1$ und $g_{\lambda_1}=a_{\lambda_1}=-2$}
  \item{$J_n(\lambda)\smspc g_{\lambda}=1\smspc a_\lambda=n$}
  \item{$\lambda I_n \smspc g_\lambda=a_\lambda=n$}
\end{itemize}
Man merkt dass:

\satz{10.5.4} $T:V\rightarrow V$ mir Eigenwert $\lambda$ Dann gilt $g_\lambda\le a_\lambda$
\beweis Sei $v_1,...,v_k$ eine Basis von $E_\lambda,v_k$ eine Basis von $E_\lambda$ und wir erweitern sie zu einer Basis \[B=\lbrace v_1,...,v_k,v_{k+1},...,v_n\rbrace\]
von $V$. Dann ist \[[T]^B_B=\begin{pmatrix}\lambda I_k&C\\0&D\end{pmatrix}\] Dann ist
$\det([T]^B_B-xI_n)=(\lambda-x)^k\cdot\det(D-xI_{n-k})$ das bedeutet dass $k\le a_\lambda$ da im $\det(D-xI_{n-k})$ auch eine Nullstelle vorkommen kann.

\korollar{10.5.5} Es seien $\lambda_1,...,\lambda_k$ unterschiedliche Eigenwerte von $T$, dann gilt: \[T \text{ ist diagonalisierbar }\Leftrightarrow g_{\lambda_i}=a_{\lambda_i}\smspc \forall i\]
\beweis Korollar 10.4.10 sagt dass \[T\text{ ist Diagonalisierbar }\Leftrightarrow V=E_{\lambda_1}\bigoplus...\bigoplus E_{\lambda_k}\Leftrightarrow \dim(V)=\sum\dim(E_{\lambda_i})=\sum g_{\lambda_i}\le\sum a_{\lambda_i}=n=\dim(V)\]
da beide seiten $\dim(V)$ haben, dann ist $\sum g_{\lambda_i}=\sum a_{\lambda_i}$ und da $a_{\lambda_i}\ge g_{\lambda_i}$ ist $a_{\lambda_i}=g_{\lambda_i}\smspc\forall i$
\theorem{10.5.6} Sei $\dim(V)=n$ mit $T:V\rightarrow V$ dann sind folgende aussagen äquivalent:\begin{itemize}
  \item{$T$ ist Diagonalisierbar}
  \item{$\forall\lambda$ gilt $a_\lambda=g_\lambda$}
  \item{Seien $\lambda_1,...\lambda_k$ Eigenwerte, dann gilt $X_T(x)=\Pi(\lambda_i-x)^{g_{\lambda_i}}$}
  \item{$V=\overset{k}{\underset{i=1}{\bigoplus}}E_{\lambda_i}$}
\end{itemize}
Die Beweise sind schon alle vorgeführt gewesen.
Was machen wir mit den Matrizen die man nicht diagonalisieren kann? 
\section*{Das minimale Polynom}
\subsection*{Definition und Erste Eigenschaften}\definition{11.1.1} Sei $T:V\rightarrow V$ linear, dann ist $T^k=\underset{k\text{ mal}}{\underbrace{T\circ...\circ T}}$ und $T^0=id_V$. Die definition ist für Matrizen analog.
\definition{11.1.2} Sei $g(x)=a_dx^d+...+a_1x+a_0\in K$ ein Polynom, dann definieren wir $g(T)=a_dT^d+...+a_1T^1+a_0T^0\in End_k(V)$. Es geht auch mit matrizen.
\beispiel{e 11.1.4}
\begin{itemize}
  \item{$A=\begin{pmatrix}1&2\\-1&1\end{pmatrix}\smspc f(x)=x^2-x+3\Rightarrow f(A)=A$}
  \item{$g(x)=x^n$ dann ist $g(J_n(0))=0_{n\times n}$ im Jordanblock, verschiebt sich die diagonale nach oben rechts.}
\end{itemize}
\satz{11.1.5} Sei $T\in End_K(V)$ dann $\exists g(x)\in K[x]$ so dass $g(T)=0_v$
\beweis $\dim(End_k(V))=n^2\Leftrightarrow\dim(V)=n$ dass heisst dass $T^0,T^1,...T^{n^2}$ sind alle linear unabhängig, und daher:\[\exists a_0,...a_{n^2}\in K \neq 0\text{ so dass }a_0T^0+...+a_{n^2}T^{n^2}=0_v\]
 Aber kann man dieses Polynom finden, und hat es einen zusammenhang mit den Charakteristischen Polynom
\bemerkung{11.1.6} Wenn $g(T)=0_V$ dann gilt auch $(\alpha g)(T)=0_V\smspc \forall \alpha \in K$
\beispiel{e 11.1.7}\begin{itemize}
  \item{Sei $n\ge1\smspc, A=Id_n$ und $g(x)=x-1$ dann gilt $g(A)=0_{n\times n}$}
  \item{Sei $A=\begin{pmatrix}\lambda_1&\left.\right.&0\\\left.\right.&\ddots&\left.\right.\\0&\left.\right.&\lambda_k\end{pmatrix}$ dann haben wir
    $\forall g\smspc g(A)=\begin{pmatrix}g(\lambda_1)&\left.\right.&0\\\left.\right.&\ddots&\left.\right.\\0&\left.\right.&g(\lambda_k)\end{pmatrix}$ Hier konnen wir also $X_A(A)=0_{n\times n}$ nehmen}
\end{itemize}
Gilt dies also für jede Matrix?

Gilt also dass $g(x)=X_A(x)$ für jede Matrix $A$?

\behauptung{} Sei $A\in M_{n\times n}(K)$ dann ist $X_A(A)=0_{n \times n}$ Hier kommen wir später zuruck
\definition{11.1.8} Sei $T:V\rightarrow V$ linear. Das minimale Polynom ist das monische ($\neq 0$) Polynom kleinsetn Grades $m_T(x)\in K[x]$ so dass $m_T(T)=0_V$
 \lemma{11.1.9:} Seien $m(x)$ und $m'(x)$ beide Monisch, vom kleinsten Grad $d\ge1$ so dass $m(T)=m'(T)=0_V$. Dann gilt $m(x)=m'(x)$.
\beweis Nimm an dass $m(x)\neq m'(x)$ Dann sei \[g(x)=m(x)-m'(x)\neq\Rightarrow deg(g)<d\text{ und } g(T)=0_V\]Was ein Widerspruch bringt.
\satz{11.1.10} Sei $T:V\rightarrow V$ Linear und $g(x)\in K[x]$ monisch so dass $g(T)=0_V$ Dann gilt dass $m_T(x)|g(x)$ ($m_T(x)$ teilt $g(x))$)
\beweis Polynom division: $\exists q(x), r(x)\in K[x]$ mit $deg(r)<deg(m)$ so dass $g(x)=m(x)q(x)+r(x)$ und da $g(T)=0_v=q(T)\underset{m(T)=0}{\underbrace{m(T)}}+r(T)$ also $r(T)=0_V\Rightarrow r(x)=0$ 
\beispiel{11.1.11}
\begin{itemize}
  \item{$A=\begin{pmatrix}\lambda&0\\0&\mu\end{pmatrix}\Rightarrow X_A(x)=(x-\lambda)(x-\mu)$ wir wissen dass $X_A(A)=0_{2\times 2}$ und wir wissen das der minimale polynom der Charakteristische Polynom teilt. Wenn also $\mu\neq\lambda\Rightarrow m_A(x)=X_A(x)$ aber wenn $\lambda=\mu\Rightarrow m_A(x)=x-\lambda$}
  \item{Sei $A=\begin{pmatrix}\lambda&0&0\\0&\lambda&0\\0&0&\mu\end{pmatrix}\mspc X_A(x)=(\lambda-x)^2(\mu-x)\Rightarrow m_A(x)=(\lambda-x)(\mu-x) \smspc\lambda\neq\mu$ Wenn $\mu=\lambda\smspc m_a(x)=x-\lambda$}
  \item{$A=J_n(\lambda)\Rightarrow X_A(x)=(\lambda-x)^n\Rightarrow m_A(x)=X_A(x)$ }
  \item{$A=\lambda id_n\Rightarrow X_A(x)=(\lambda-x)^n$ und $m_A(x)=x-\lambda$}
\end{itemize}


\textbf{Cayley Hamilton} $A\in M_{n\times n} \Rightarrow X_A(A)=0_{n\times n}$
\beweis\[(A-xid_n)adj(A-xid_n)=X_A(x)id_n\mspc (*)\] Hier schreibe man $adj(A-xI_n)=(p_{ij}(x))$ wobei $p_{ij}(x)\in K[x]$ mit $deg(p_{ij})\le n-1$
 \bemerkung{11.3.2} \[adj(A-xid_n)=B_{n-1}x^{n-1}+...+B_1x+B_0\smspc B_i\in M_{n\times n }(K)\]\[X_A(x) id_n=(-1)^n(x^n+a_{n-1}x^{n-1}+...+a_0)\]Wir setzen diese letze gleichung in $(*)$ ein und bekommen \[AB_0=(-1)^na_0 id_n\]
\[-B_0+AB_1=(-1)^na_1 id_n\]
und so weiter bis:
\[-B_{n-1}=(-1)^nid_n\]
Und zu zeigen ist \[X_A(A)==_{n\times n}\Leftrightarrow (-1)^nA^n+a_{n-1}A^{n-1}+...+a_1A+a_0id_n=0_{n\times n}\]
Wir konnen jede gleichung vom system mit $A^i$ multiplizieren (wo $i=deg$ der linie) und summieren dass alles zusammen. Wir finden dass die Summe $=0$ und dass $X_A(A)=0_{n\times n}$
\section*{Jordansche Normalform} \definition{ Theorem} sei $\lambda\in K$ und $n\ge1$ der Jordanblock der länge $n$ und Eigenwert $\lambda$ ist folgende Matrix:\[J_n(\lambda)=\begin{pmatrix}\lambda&1&\smspc&0\\\smspc&\ddots&\ddots&\smspc\\0&\smspc&\lambda\end{pmatrix}\]
\lemma{12.1.1} $X_{J_n}(x)=(\lambda-x)^n$ und $\lambda$ ist der Einzige eigenwert, $g_\lambda=1$ und $a_\lambda=n$ mit $m_{J_n}(x)=(-1)^nX_{J_n}(x)$ 
\theorem{12.1.2} Jordansche Normalenfor, Sei $T:V\rightarrow V$ Dann $\exists B$ eine Basis von $V$ so dass 
\[[T]^B_B=\begin{pmatrix}J_{n_1}(x_1)&\smspc&\smspc&0\\\smspc&J_{n_2}(x_2)&\smspc&\smspc\\\smspc&\smspc&\ddots&\smspc\\0&\smspc&\smspc&J_{n_k}(x_k)\end{pmatrix}\]
Dies darstelleung ist eindeutig bis auf die Vertauschung der Blocke. 
\theorem{12.1.3} Sei $A\in M_{n\times n} (K)$ dann $\exists B\in GL_n(K)$ so dass $B^{-1}AB$ die Jordansche Normalenform hat.
\lemma{12.2.2} Sei $C\in M_{n\times n}(K)$ wober $C=\begin{pmatrix}A&0\\0&B\end{pmatrix}$ und $A,B\in M_{(n-2)\times (n-2)}(K)$ Definiert jetzt $U_<e_1,...,e_i>,W=<e_i,...e_n>$ 
Sei $v=u+w\in V\neq0_v,u\in U,w\in W$ Dann ist $v$ denau dann ein Eigenvektor von $T_C$ mit Eigenwert $\lambda$ wenn $T_A(u)=\lambda_u \cap T_B(w)\lambda w$ 
$E_\lambda (T_C)=E_\lambda(T_A)\bigoplus E_\lambda(T_B)\Rightarrow g_\lambda(T_c)=g_\lambda(T_A)+g_\lambda(T_B)$
\satz{12.2.3} Sei $T:V\rightarrow V$ linear und Nimm an $\exists B$ Basis von V So dass $[T]^B_B$ die Diagonal Jordansche normalenform annimt und sei $\lambda$ Eigenwert von $T$ Dann gilt $g_\lambda=\#\lbrace i|1\le i\le k, \alpha_i=\lambda\rbrace$
$a_\lambda=\sum_{\alpha_i=\lambda}\lambda_i=$ Die länge des gsten Jordanblock mit eigenwert $\lambda =s(\lambda)=\max\lbrace n_j|1\le j\le k, \alpha_j=\lambda\rbrace$
\beweis $B=\lbrace b_1^{(1)},...,b_{n,1}^{(1)}, ...... ,b_{1n}^{(k)},...,b_{kn}^{(k)}\rbrace$
Sei $W_i=<b_1^{(i)},...,b_n^{(n)}>\Rightarrow V=\bigoplus W_i$ und $T$ 
 Sei $T_V\rightarrow V$ linear und $\lambda$ ein Eigenwert von $T$ Dann ist \[\tilde{E_\lambda}=\bigcup_{k\ge1}^{\infty}\ker(T-\lambda id_v)^k\] Der Verallgemeinerte Eigenraum von $\lambda$, wo alle Vektoren die sich durch eine Potenz von $T$ auf einen Skalar von sich selber SChicken lassen.
Wenn $T$ Diagonlisierbar ist, dann ist $V$ Die direkte summe der $E_\lambda$'s
\behauptung{} \[V=\bigoplus_{\lambda=\text{Eigenwert}}\tilde{E_\lambda}\]
\lemma{12.3.3} Sei $n=\dim(V)$ dann gilt $\tilde{E_\lambda}=\ker(T-\lambda id_v)^n$
\beweis Sei $V\in \tilde{E_\lambda}$ und sei $k$ minimal so dass $T-\lambda id_v)^kv=0_v$ und $(T-\lambda id_v)^{k-1}v\neq 0$ So ein $k$ muss es per definition Geben. Wir wenden jetzt Lemma 12.2.2 an, dann sind 
$v,sv,\cdots,s^{k-1}v$ linear unabhängig, und daraus folgt $k\le n$
\definition{12.3.4} Sei $v\in \tilde{E_\lambda}$ mit $v\neq 0$ und $k\ge1$ minimal so dass $(T-\lambda id_v)^kv=0_v$ Dann ist $\lbrace v,(T-\lambda id_v)v, \cdots ,(T-\lambda id_v)^{k-1}v\rbrace$ die Jordankette von $v$ der länge $k$
\bemerkung{12.3.5} $(T-\lambda id_v)^{k-1}v=w$ ist ein Eigenvektor von $T$ mit Eigenwert $\lambda$
\beweis $Tw=\lambda w\Leftrightarrow (T-\lambda id_v)w=0_v\Leftrightarrow (T-\lambda id_v)^k v=0_v$
\bemerkung{12.3.6} Jeder eigenvektor von $T$ bildet/ist eine Jordankette von länge $1$.
\beispiel{12.3.7}\begin{itemize}
  \item{Sei $A=J_n(\lambda)\tilde T_A:K^n\rightarrow K^n$ und $S=T_A-\lambda id_{K^n}$ Dann ist die längste Jordankette von länge $n$.}
  \item{Seien $\lambda\neq\mu\in K$ mit \[A=\begin{pmatrix}J_2(\lambda)& & & &0\\ &J_3(\lambda)& & & \\ & & \lambda & & \\ & & & J_2(\mu) & \\0& & & & \mu\end{pmatrix}\] Eine $9\times9$ Matrix, gibt es Folgende Jordanketten von $A$:
    \begin{multicols}{2}\begin{itemize}
      \item[]{$e_5\rightarrow e_4\rightarrow e_3$}
      \item[]{$e_6$}
      \item[]{$e_2\rightarrow e_1$}
      \item[]{$e_8\rightarrow e_7$}
      \item[]{$e_9$}
    \end{itemize}
  \end{multicols}}
\item{Wir haben jetzt: $J_2(\lambda)-\lambda id_n=\begin{pmatrix}0&1\\0&0\end{pmatrix}=S$ Dann ist $\ker{S^2}=K^2$ und doch $\ker(S)=<e_1>$ dass heisst wir haben eine Jordankette $e_2\rightarrow e_1\rightarrow 0$}
\end{itemize}
\lemma{12.3.8} $\tilde{E_\lambda}$ ist ein $T$ invarianter unterraum von $V$. ($T$ invariant heisst: $T(\tilde{E_\lambda})\subset\tilde{E_\lambda}$)
\beweis Sei $v\in\tilde{E_\lambda}\Leftrightarrow(T-\lambda id_v)^n v=0_v$. Zu zeigen ist $Tv\in \tilde{E_\lambda}$ Dies ist klar. Wir nehmen dann $(T-\lambda id_V)^nTv=T(T-\lambda id_V)^n v=T0_V=0_V$ Da $T$ kommutiert
\satz{12.3.9} $\lambda$ ist der Einzige Eigenwert von $T|_{\tilde{E_\lambda}}$
\beweis Sei $v\in \tilde{E_\lambda}$ mit Eigenwert  $\mu$ d.h. $Tv=\mu v$. Sei $k\ge 1$ minimal so dass $(T-\lambda id_V)^kv=0$. Sei $w=(T-\lambda id_v)^{k-1}v$ da die $k-1$ Potenz der Einzige eigenvektor der Jordankette ist.
Dann gilt $(T-\lambda id_V)w=0\Leftrightarrow Tw=\lambda w$ aber $(T-\mu id_v)w=(T-\mu id_V)(T-\lambda id_V)^{k-1}v=(T-\lambda id_V)^{k-1}(T-\mu id_v)v=0_V$ Also gibt es hier einen Widerspruch da ein vektor nicht zwei unterschiedliche Eigenwerte haben kann, d.h. $\lambda= \mu$
\newline Sei $A=\begin{pmatrix}\lambda&1&0&0\\0&\lambda&1&0\\0&0&\lambda&1\\0&0&0&\lambda\end{pmatrix}$ Dann ist $\lambda$ der Eigenwert von $A$ mit $X_A(x)=(\lambda-x)^4$. $\ker(A-\lambda id)=<e_1>=E_\lambda$
Wass passiert wenn wir jetzt $\tilde{E_\lambda}$ suchen: \[\tilde{E_\lambda}=\underset{=E_\lambda}{\underbrace{\ker(A-\lambda id)}}\cup \underset{=<e_1,e_2>}{\underbrace{\ker(A-\lambda id)^2}}\cup \underset{=<e_1,e_2,e_3>}{\underbrace{\ker(A-\lambda id)^3}}\cup \underset{N^4=0_{4\times 4}\Rightarrow \ker=K}{\underbrace{\ker(A-\lambda id)^4}}\]
Also ist $\tilde{E_\lambda}=K$
\korollar{12.3.10} Sei $T'=T|_{\tilde{E_\lambda}}$ Dann gilt dass $X_{T'}(x)=(\lambda-x)^{k\lambda}$ mit $k\lambda\le a_\lambda(T)$
\beweis Dies ist klar da hier hat $T'$  $\lambda$ als einziges eigenwert, und der $X_{T'}(x)$ muss ja $X_T(x)$ Teilen.
\lemma{12.3.11} Seien $\mu,\lambda$ verschieden Eigenwerte von $T$ dann gilt $\tilde{E_\lambda}\cap \tilde{E_\mu}=\lbrace 0_V\rbrace$ und daher gilt $\tilde{E_\lambda}+\tilde{E_\mu}=\tilde{E_\lambda} \oplus \tilde{E_\mu}$
\lemma{12.3.12} Es seien $\lambda_1,\cdots\lambda_k$ verschiedene Eigenwerte von $T$ Dann gilt $\sum_{i=1}^k \tilde{E_{\lambda_i}}=\bigoplus_{i=1}^k \tilde{E_{\lambda_i}}$
\behauptung{} $V$ ist die Direkte summe seiner Verallgemeinerten Eigenräume.
\beweis Die idee ist dass wir ein $T$-invarianten Unterraum $U\subset V$ finden so dass $V=(\bigoplus \tilde{E_{\lambda_i}})\oplus U$
$\ker(T-\lambda -id_v)^n \oplus U =V$ Hier kann man $U=im(T-\lambda id_v)$ nehmen, da $\ker \oplus im = K$
\lemma{12.3.14} Sei $T:V\rightarrow V$ mit $\dim(V)=n$ Linear, $\lambda$ ein Eigenwert und $g(x)=(x-\lambda)^n$ dann ist im$(g(T))$ eom $T$-invarianter unterraum von $V$ und es gilt das $V= \tilde{E_\lambda}\oplus \text{im}(g(T))$
\beweis Zu beweisen: $w\in im(g(T))$ Dass ist einfach, $Tw\neq 0\Leftrightarrow w=g(T)v$ für $v\in V$ $g(T)Tv=Tg(T)v=Tw\Rightarrow Tw\in$ im$(g(T))$
Mit dem Theorem 4.2.9 ($dim(V)=dim(\ker(V))+$im$(V)$) ist jetzt nur zu zeigen ist das $\ker(g(T))\cap$ im$(g(T)) =\lbrace 0_V\rbrace$ Sei $w\in$ im$(g(T))$ also $w=g(T)v$ und wenn $w\in \tilde{E_\lambda}$ dann gilt
$g(T)w=0_v$ das ist aber dass gleiche als $(T-\lambda id_v)^{2n}v=0_v\Rightarrow (T-\lambda id_v)^n v=0_v$
\lemma{12.3.14} Sei $\mu\neq\lambda$ Eigenwerte. Dann gilt $\tilde{E_\mu}\in$  im$(g(T))$ und im$((T-\lambda id_v)^n)$ ist $T$ invariant
\theorem{12.3.15} Sei $T:V\rightarrow V$ linear dann gilt \[V=\bigoplus_{\lambda: \text{ Eigenwert}} \tilde{E_\lambda}\]
\beweis Induction über die $n=\dim V$ Es sei $\lambda$ ein Eigenwert dann gilt vom lemma 12.3.14 dass $V=\tilde{E_\lambda}\oplus$ im$(T-\lambda id_v)^n)$. Sei $U=(T-\lambda id_v)^n$ Dann ist $U$ $T$-invariant und es gilt dass $\dim(U)<n$, dauraus folgt dass \[U =\bigoplus_{\mu\text{ Ew von } T|_{\mu}}\tilde{E_\mu}\Rightarrow V=\tilde{E_\lambda}\oplus\bigoplus_{\mu}\tilde{E_\mu}\] 
Dank dieses Resultat, können wir dann annehmen dass $V=\tilde{E_\lambda}$, daher um Theorem 12.1.2 zu zeigen, reicht es zu zeigen dass jeder $\tilde{E_\lambda}$ eine Jordanbasis besitzt
\subsection*{Beweis der Jordanschen Normalenform}
Wir Fangen mit dem Spezialfall $S:V\rightarrow V$ nilpotent
\theorem{12.4.1} Sei $V$ n-dimensional und $S:V\rightarrow V$ nilpotent dann $\exists k \ge , n, \cdots n_k \le 1$ so dass $n_1+\cdots+n_k=n$ und eine Basis $B$ von $V$ die aus Jordanketten der länge $n_i$ besteht. 
\[B=\lbrace S^{n_1-1}u_1, ..., Su_1, u_1, S^{n_2-1}u_2,..., Su_2, u_2,\cdots ,S^{n_k-1}u_k,..., Su_k, u_k\rbrace \mspc \forall S^{n_i}u_i\neq 0\]
Die $n_i$ sind hier auch eindeutig bestimmt. Diese Basis der solchen Form, heisst Jordanbasis.
\bemerkung{12.4.3}Die abbildungsmatrix $[S]$ in der Jordanbasis $B$ nimmt folgende form an:\[[S]_B^B=\begin{pmatrix}J_{n_1}(0)& & 0\\&\ddots&\\0&&J_{n_k}(0)\end{pmatrix}\] Was die Jordan'sche normalenform annimt
\beweis Dass beweist man mit Induktion über $n$-dim von $V$. $n=1$ ist klar. Dann nehmen wir an es klappt für $dim V<n$ Beachte dass $S(V)\subsetneq V$ weil $\ker(S)\supsetneq 0_v$. Also im Fall $\ker(S)=V$ dann ist $S$ die nullabbildung und das Theorem hält.
Also ist $\ker(S)\neq\lbrace 0_v\rbrace$. \newline
\textbf{Induktionshypothese:} $\exists v_1 ... v_l\in S(V)$ und $b_1, ... ,b_l\ge1$ so dass $\sum b_i=\dim S(V)$ und \[B'=\lbrace v_1, ... S^{b_1-1}v_1, v_2,...,S^{b_2-1}v_2,...,v_l,..S^{b_l -1}v_l\rbrace\]
eine Basis von $S(V)$ ist. Nach unsere Annahme ist $v_i\in im(S)\forall i$ daher $\exists u_i\in V$ so dass $Su_i=v_i$ (uhrbild) und damit kann man immer die Jordankette immer ein stück länger machen:
\[B''=\lbrace u_1, \underset{=v_1}{\underbrace{Su_1}},...,\underset{=S^{b_1-1}v_1}{\underbrace{S^{b_1}u_1}},...u_l,\underset{=v_l}{\underbrace{Su_l}},...,\underset{=S^{b_l-1}v_l}{\underbrace{S^{b_l}u_l}} \rbrace \]
Zur übung steht jetzt zu zeigen dass $B''$ linear unabhängig ist. Hier fehlt noch der teil der in $ker(S)$ ist, $\forall i$ ist doch $ S^{b_i}u_i\in ker(S)$. Wir erweitern also die menge $\lbrace S^{b_1}u_1,...S^{b_l}u_l\rbrace$ zu einer Basis von $ker(S)$ 
\[\lbrace u_1, S^{b_1}u_1,...S^{b_l}u_l, \underset{\notin\text{Jordanketten}}{\underbrace{w_1, w_2, ..., w_m}}\rbrace\]
Jetzt Behaupten wir dass diese Erweiterung eine Basis von $V$ in der gewünschten Form ist. Die Länge der Jordanketten ist also von der unteren Basis ablesbar
\[\lbrace \underset{b_1+1}{\underbrace{u_1,Su_1,...,S^{b_1}u_1}},\underset{b_i+1}{\underbrace{\hdots}},\underset{b_l+1}{\underbrace{u_l,Su_l,...,S^{b_l}u_l}}, \underset{1}{\underbrace{w_1}}, \underset{1}{\underbrace{w_2}}, ...,\underset{1}{\underbrace{w_m}}\rbrace\]
Es bleibt doch noch zu zeigen dass dies eine Basis von $S$ ist:
\begin{itemize}
  \item{Linear unabhängigkeit ist eine übung da der Beweis nicht sehr schwierig ist}
  \item{Zeigen dass diese Basis ein Erzeugendensystem von $V$ ist, merken wir dass $\dim(ker(S))=l+m$ und $\dim(im(\sum b_i))$ $\Rightarrow l+m+\sum b_i$ und unsere basis hat $m+\sum b_i+1$ elemente und da es $l$ mal den $b_i$ gibt gilt $dim(S)=dim(B'')=m+l+\sum b_i$ und unsere Basis ist  ein Erzeugenden System}
\end{itemize}
Der beweis ist komplex, den Beweis mit einen konkreten beispiel auszuführen ist wichtig da diese Sache im Examen seien wird.
\korollar{12.4.4} Es sei $T:V\rightarrow V$ linear, $\lambda$ ein Eigenwert und $U=\tilde{E_\lambda}$ dann existiert eine Basis $B$ von $U$ so dass $\left[\left.T\right|_U\right]^B_B$ in der Jordan normal form ist (JNF).
\beweis Es sei $S:\left.(T-\lambda id_v)\right|_U$ Dann ist $S$ per definition der nilpotenz, nilpotent. Dass heisst das wir per Theorem 12.4.1 diese in JNF stellen können. Und dann ist $\left.T\right|_U=S+\lambda id_v$ und daher ist $T$ in derselben Basis auch in JNF.
\bemerkung{12.4.5} Seien $A,B\in M_{m\times n}(K)$ dann gilt $A$ und $B$ sind ähnlich wenn sie die gleiche JNF haben abgesehen von einer Vertauschung der Blöcke. Die Jordanbasen Jedoch sind nicht eindeutig.
\subsection*{12.5 Berechnung der Jordan normalenform}
\beispiel{12.5.1} Was ist die JNF der Folgenden Matrix?\[A=\begin{pmatrix}1&1&6&-2\\0&1&-3&2\\0&0&1&0\\0&0&-2&2\end{pmatrix}\in M_{4\times4}(\mathbb{C})\]
$\mathbf{X_A(x)}$ \textbf{, Eigenwerte und vielfachheiten finden}. In diesem Fall $X_A(x)= (1-x)^3(2-x)\Rightarrow \lambda_1=1\smspc\lambda_2=2$ Da $1=g_{\lambda_1}\neq a_{\lambda_1}=3$
und $a_{\lambda_2}=g_{\lambda_2}=1$ Dann wissen wir dass Die anzahl jordanblöcke von einen Eigenwert $\lambda$ die geometrische vielfacheit $g_\lambda$ ist, Um zu wissen wie gross die Dazugehörigen JB sind schaut man sich die Dazugehörige Jordanketten an, bzw wir schauen $\forall i\in E_\lambda$ ob $\exists?v\in K$ so dass $(D-\lambda id_V)v=i$ wenn nein dann ist die Kette von länge 1 und wenn ja dann mindestens von länge 2, die Lange des Kurzestens Jordanblock ist auch die Potenz des minimalen Polynoms. \[\Rightarrow[A]_B^B=\begin{pmatrix}J_{a_{\lambda_1}}(\lambda_1)&0\\0&J_{a_{\lambda_2}}(\lambda_2)\end{pmatrix}=\begin{pmatrix}1&1&0&0\\0&1&1&0\\0&0&1&0\\0&0&0&2\end{pmatrix}\]
Wir können jetzt dank der JNF eine Jordankette für $\tilde{E_{\lambda_1}}$. Daher wollen wir $v\in \mathbb{C}^4$ finden so dass \[v, (A-\lambda_1 id_V)v, \hdots , (A-\lambda_1 id_V)^2\] Eine Basis von $\tilde{E_{\lambda_1}}$ ist (wir gehen bis 2 da die länge des blocks 3 ist)
Die Jordanbasis zu finden ist dasselbe als alle Jordanketten zu finden.
\section*{13 Euklidische und Hermitesche Räume}
\subsection*{13.1 Normierte Räume}
\definition{13.1.1} Es sei $K=\mathbb{R}$ oder $\mathbb{C}$ und $V$ ein $K$ vektorraum. Eine Norm auf $V$ ist ein Funktion $||\cdot||:V\rightarrow \mathbb{R}^+$. Eine Norm hat folgende Axiome:
\begin{itemize}
  \item[\textit{i.}]{\textbf{Dreiecksungeichung} $||u+v||\le||u||+||v||$}
  \item[\textit{ii.}]{\textbf{Linearität} $||\lambda v||=|\lambda|\cdot||v||$}
  \item[\textit{iii.}]{\textbf{Wohldefinitheit} $||v||=0\Rightarrow v=0_V$}
\end{itemize}
\beispiel{13.1.2}
\begin{itemize}
  \item{$||v||=\sqrt{a^2+b^2}$}
  \item[]{$||v||=|a|+|b|$}
  \item[]{$||v||=\sqrt[p]{|a|^p+|b|^p}$}
  \item[]{$||v||=max(|a|,|b|)$}
\end{itemize}
\bemerkung{13.1.4} Sei $p\ge1$ und $v=\begin{pmatrix}x_1\\\vdots\\x_n\end{pmatrix}\in\mathbb{R}^n$ Dann definiert $||v||_p=(|x_1|+\hdots+|x_n|)^{\frac{1}{p}}$ eine Norm auf $\mathbb{R}^n$
\beispiel{13.1.5} $V$ ist ein Vektorraum der Stetige funktionen $f:[0,1]\rightarrow \mathbb{R}$ Dann definieren wir $|f|_{max}=max\lbrace|f(x)|, x\in[0,1]\rbrace$
\definition{13.1.6} Sei $v\in V$ ein Einheitsvektor, dann iglt $||v||=1$ und die Distasnz zwischen zwei vektoren $u,v \in V$ ist $||u-v||=d(u,v)$
\bemerkung{13.1.7} Es sei $v\in V, v\neq 0$ Dann ist $\frac{v}{||v||}$ ein Einheitsvektor, nähmlich die normalisierung von $v$.
\bemerkung{13.1.8}Die Menge der Einheitsvektoren kann von der Norm viel varianz ziehen. Wir schauen uns zum beispielGgg die menge von Einheitsvektoren in $\mathbb{R}^2$ an:
\begin{center}
{\ctikzfig{Einheitsvektoren}}
\scriptsize \textit{Die menge der Einheitsvektoren unterschiedlicher p-normen}\normalsize
\end{center}
Für $p=\infty$ kann man auch die maximumsnorm von vorher nehmen.
\subsection*{13.2 Innere Produkte}
\definition{13.2.1} Sei $V$ ein $\mathbb{R}$ vektorraum. Dann ist ein Inneres Produkt auf $V$ ist eine funktion $<\cdot>:V\times V\rightarrow\mathbb{R}$ so dass :
\begin{itemize}
  \item{Symmetrie: $<u,v>=<v,u>\smspc\forall u,v\in V$}
  \item{Bilinearität: $<\alpha u,v> = alpha<u,v>$ und $<u+v,w>=<u,w>+<v,w>$ Und natürlich das gleiche in der zweiten variable}
  \item{Positiv definitheit: $<v,v>>0\smspc\forall v\neq0_v$}
\end{itemize}
Wenn wir $(V,<\cdot>)$ nehmen dann ist dass ein Euklidischer Raum.
\beispiel{13.2.2} Der Standard innere Produkt ist $\mathbb{R}^n\times\mathbb{R}\rightarrow\mathbb{R}$ wobei $<e_i,e_j>=\delta_{ij}$
\beispiel{13.2.3}Sei $V=\mathbb{R}^2$ und sei $u=\begin{pmatrix}x_1\\y_1\end{pmatrix}$ und $v=\begin{pmatrix}x_2\\y_2\end{pmatrix}$ dann ist $<u,v>=2x_1x_2-x_1y_2-y_1x_2+y_1y_2$ dies ist ein inneres produkt da es Symetrisch ist (klar), bilinear da nur lineare termen vorkommen und positiv da es immer positiv ist wenn wir den selben vektor zweimal nehmen.
\bemerkung{13.2.4} Eine wichtige bemerkung ist dass dieses inneres Produkt auch als matrix multiplication dargestellt werden kann.
Zum beispiel kann man der vorherige Beispiel wie folgt definieren: $<u,v>=u^t\begin{pmatrix}2&-1\\-1&1\end{pmatrix}v$. Für dass Standard innere Produkt ist $<u,v>=u^tv$.
\beispiel{13.2.5} Seien $a\le b$ und $C[a,b]=\lbrace f:[a,b]\rightarrow\mathbb{R}\rbrace$ wobei $f$ stetig ist. Wir definieren also $<f,g>=\int_a^b f(x)g(x)dx$ als inneres Produkt. Die Symmetrie und linearität sind offensichtlich und sei $f\in C[a,b]\smspc f\neq0$ Dann $\exists y\in [a,b],\delta>0$ so dass 
$[y-\delta,y+delta]$ und $f(x)\neq 0$ $\forall x \in [y-\delta,y+\delta]$  dann gilt $<f,f>=\int_a^b f(x)^2dx\ge\int_{y-\delta}^{y+\delta}f(x)^2dx\ge2\delta min_{[y-\delta,y+\delta]}f(x)>0$
\definition{13.2.6} Sei $V$ ein $\mathbb{C}$ vektorraum, ein Inneres Produkt ist $<\cdot>:V\times V\rightarrow \mathbb{C}$ Dies ist der Hermitesche Produkt und es ist:
\begin{itemize}
  \item{Linearität der ersten Variable: $<v_1+v_2,w>=<v_1,w>+<v_2,w>$ und $<\alpha v,w>=\alpha<v,w>$} 
  \item{Sesquilinearität in der Zweiten Variable: $<v,w_1+w_2>=<v,w_1>+<v,w_2>$ und $<v,\alpha w>=\overline{\alpha}<v,w>$}
  \item{Hermitesche eigenschaft: $<v,w>=\overline{<w,v>}\Rightarrow <v,v>\in \mathbb{R}$}
  \item{Positivität: $<v,v>>0\forall v\neq 0$}
\end{itemize}
$(V,<\cdot>)$ ist eine Hermitescher oder unitärer raum. Das standard innere Produkt auf $\mathbb{C}^n$ ist $<u,w>=\sum_{i=1}^nx_i\overline{y_i}$
\lemma{13.2.9} Sei $V$ ein  Euklidischer oder Hermitescher VR dann gilt:
\begin{itemize}
  \item{$<0_v,v>=<v,0_v>=0$}
  \item{$<v,w>=0$ dann gilt $v=0$ oder $w=0$}
  \item{Wenn $<v,w_1>=<v,w_2>\smspc\forall v\in V$ dann ist $w_1=w_2$}
\end{itemize}
\satz{13.2.10} Sei $(V,<\cdot>)$ eine Euklidischer oder Hermitescher raum dann definieren wir $||v||=\sqrt{<v,v>}$ dann ist $|\cdot|$. Hier ist der problem die Dreiecksunglechung
\lemma{13.2.11 Cauchy-Schwarz ungleichung} besagt dass $|<u,v>|\ge||u||\cdot||v||\smspc\forall u,v\in V$ mit gleichheit wenn $u\in<v>$ also wenn $u$ und $v$ linear abhängig sind.
\beweis Wenn $u$ oder $v$ der nullvektor ist dann ist es Klar. Winn nehmen an dass $u\neq 0_v$ WIr definieren jetzt $\lambda=\frac{<u,v>}{||u||^2}$ und $w=v-\lambda u$ dieser Vektor ist was übrig bleibt nach der Projektion von $v$ auf $u$. Wir merken dass $<w,u>=<v,u>-\lambda<u,u>=0$
Nun gilt \[0\ge||w||^2=<v-\lambda u,v-\lambda u>=||v||^2-\lambda\overline{\lambda}||u||^2-\overline{\lambda}\lambda||u||^2+\lambda\overline{\lambda}||u||^2=||v||^2-\frac{|<u,v>|}{||u||^2}\Rightarrow |<u,v>|\le||u||\cdot||v||\]
\beweis \textbf{von Satz 13.2.10} Für Hermitesche VR da für euklidische VR analog gelten. Seien $u,v\in V$ dann gilt $||u+v||^2=<u+v,u+v>=||u||^2+2Re(<u,v>)+||v||^2$ Beachte $\forall z \in \mathbb{C}$ gilt $Re(z)\le |z|$
Dass bedeutet dass $Re(<u,v>)\le|Re(<u,v>)|\le |<u,v>|$ und also $||u+v||^2\le||u||^2+2|<u,v>|+||v||^2\le(||u||+||v||)^2$
\lemma{13.2.12} Sei $(V,<\cdot>)$ ein Euklidischer Raum dann gilt $<u,v>=\frac{1}{2}(||u+v||-||u||^2-||v||^2)$ wobei $||\cdot||$ die induzierte Norm ist. Der Beweis ist durch Expansion gemacht, und mit diesem Lemma kann man zeigen dass nicht alle Normen aus einen Inneren Produkt stammen.
\subsection*{Konstruktion von innere Produkte}
\beispiel{13.3.1} Sei $A\in M_{n\times n}(\mathbb{R})$ Wir definieren $(\cdot,\cdot)_A:\mathbb{R}^n\times\mathbb{R}^n\rightarrow \mathbb{R}$ mit $(u,v)_A=u^tQv$. Diese Paarung ist bilinear aber nicht unbedingt für alle Matrizen $A$ symmetrisch: z.B $A=\begin{pmatrix} 1&1\\0&1\end{pmatrix}$ geht es nicht.
Wir Brauchen eine Invariante matrix unter der transposition
\definition{13.3.2} Eine Matrix $A\in M_{n\times n}(K)$ ist symmetrisch wen $A=A^t$
\lemma{13.3.3} Es sei $A\in M_{n\times n}(K)$ symmetrisch dann gilt dass $(u,v)_A=(v,u)_A$ 
\definition{13.3.4} Eine Matrix $A\in M_{n\times n} (K) $ ist Positiv definit wenn sie Symmetrisch ist und alle ihre Eigenvektoren Positiv sind, anallog kann man sagen dass wenn $\forall x\neq0 \in V$ $x^tAx>0$ gilt.
\satz{13.3.7} Sei $A\in M_{n\times n}(\mathbb{R})$ dann ist $(\cdot,\cdot)_A$ genau dann ein Inneres Produkt wenn $A$ Positiv definit ist 
\definition{13.3.9} Es sei $B\in M_{n\times n}(\mathbb{C})$ dann definieren wir $(\cdot,\cdot):\mathbb{C}^n\times\mathbb{C}^n\rightarrow \mathbb{C}$ mit $(u,v)_B=u^tB\overline{v}$
\definition{13.3.10} Schreibe $\overline{B}=(\overline{b_{ij}})$ die adjungierte Matrix von $B$ ist die matrix $B^*=\overline{B^t}$ wir sagen dass $B$ Hermitesch ist wenn $B=B^*$
\beispiel{13.3.11}
\begin{itemize}
  \item{$A=\begin{pmatrix}-1&1+i&2i\\1-i&0&1-2i\\-2i&1+2i&1\end{pmatrix}\Rightarrow$ Hermitesch}
  \item{$B=\begin{pmatrix}-i&1+1&2i\\1-i&0&1-2i\\-2i&1+2i&i\end{pmatrix}\Rightarrow$ Nicht Hermitesch da auf der Ditagonale $i$ ist}
\end{itemize}
Sei $B=\begin{pmatrix}a&b\\c&d\end{pmatrix}$ ist Hermitesch wenn $a,d\in \mathbb{R}$ und $c=\overline{b}$
\bemerkung{13.3.12} Wenn $B$ hermitesch ist dann erfüllt $(\cdot,\cdot)_B$ alle kriterium des Komplexen inneren Produktes\newline
Und Sei $B$ Hermitesch und $v\in\mathbb{C}^n$ Dann gilt $(v,v)_B\in\mathbb{R}$ um dies zu beweisen Schaut man sich $\overline{(v,v)_B} $ an.
\definition{13.3.13} Eine Hermitesche Matrix $B\in M_{n\times n}(\mathbb{C})$ Wenn $v^tB\overline{v}>0\smspc\forall v\neq0$
\satz{13.3.15} Sei $B\in M_{n\times n}(\mathbb{C})$ dann ist $(\cdot,\cdot)_B$ ein Inneres Produkt wen $B$ Hermitesch und Positiv definit ist
\beweis $"\Leftarrow"$ ist offensichtlich.\newline
$"\Rightarrow"$ Wir nehmen An $(\cdot,\cdot)$ ein Inneres Produkt ist, dann ist $B$ Hermitesch da: $b_{ij}=)\overline{b_{ji}}\smspc\forall i,j$ Da $b_{ij}=(e_i,e_j)_B=\overline{(e_j,e_i)}=\overline{b_{ji}}$
Und Positiv definitheit zu zeigen: \[(v,v)_B>0\smspc\forall v\in\mathbb{C}^n, v\neq 0\Leftrightarrow B\text{ ist positiv definit}\]
\subsection*{13.4 Orthogonalität}
\definition{13.4.1} Sei $V$ ein Inneres Produkt raum. Orthogonalität ist wie folgt definiert
\begin{itemize}
  \item{$v,w\in V$ sind Orthogonal wenn $<v,w>=0$ man Schreibt $u\perp v$}
  \item{$S\subset V$ ist ein Orthogonales system wenn $u\perp v\smspc \forall u,v\in V$ mit $u\neq v$}
  \item{Ein Orthogonales system $S$ ist Orthonormal wenn $||v||=1\smspc\forall v\in S$}
\end{itemize}
\beispiel{13.4.2}
\begin{itemize}
  \item{Die elemente der Standarbasis $e_1,\hdots,e_n$ in $\mathbb{R}^n$ sind alle orthogonal aufeinander.}
  \item{Jeder Vektor ist orthogonal zu $0_v$}
\end{itemize}
\satz{13.4.3 Satz des pythagoras} Sei $(V,<\cdot>)$ ein inneren Produkt raum und seien $u,v\in V$ mit $u\perp v$ dann gilt\[||u+v||^2=||u||^2+||v||^2\]
\beweis Expandieren
\definition{13.4.4} Es sei $(V,<\cdot,\cdot>)$ ein IP raum und $v\neq0\in V$ dann definiren wir die projektion $proj_v:V\rightarrow V$ und mit $u\rightarrow \frac{<u,v>}{<v,v>}\cdot v$
\bemerkung{13.4.5} $u\in V$ ist orthogonal zu $v\Leftrightarrow proj_v(u)=0$
\end{document}
