\documentclass{article}
\title{Analysis II Einsiedler Version}
\author{Benjamin Dropmann}
\usepackage{geometry}
\usepackage{xcolor}
\usepackage{amssymb}
\usepackage{amsfonts}
\usepackage{amsmath}
\usepackage{multicol}
\usepackage{titlesec}
\titlespacing*{\subsubsection}{0pt}{1.2ex}{.1ex plus .2ex minus .2ex}

\newcommand{\mspc}{\hspace{0.7cm}}
\newcommand{\smspc}{\hspace{0.3cm}}

\newcommand{\kk}[1]{\left<\left<{#1}\right>\right>}
\newcommand{\dd}[1]{\hspace{0.2cm}\text{d{#1}}}

\newcommand{\satz}[1]{\subsubsection*{Satz {#1}}}
\newcommand{\korollar}[1]{\subsubsection*{Korollar {#1}}}
\newcommand{\beweis}{\\\textbf{Beweis }}
\newcommand{\beispiel}[1]{\subsubsection*{Beispiele {#1}}}
\newcommand{\bemerkung}[1]{\subsubsection*{Bemerkung {#1}}}
\newcommand{\theorem}[1]{\subsubsection*{Theorem {#1}}}
\newcommand{\lemma}[1]{\subsubsection*{Lemma {#1}}}
\newcommand{\definition}[1]{\subsubsection*{Definition {#1}}}
\newcommand{\behauptung}[1]{\subsubsection*{Behauptung {#1}}}

\geometry{margin=1.5cm}
\begin{document}
\maketitle
\section{Wiederholung}
\definition{Taylor-Polynom} Sei eine funktion $f:(a,b)\rightarrow\mathbb{C}$ um einen punkt $x_0\in (a,b)$ die $n$-mal differenzierbar ist. Dann ist der Polynom 
\[p_{x_0,n}^f(x)=\sin_{k=0}^n\frac{f^{(k)}(x_0)}{k!}(x-x_0)\]
\satz{Taylor Approximation} sei $f:(a,b)\rightarrow\mathbb{C}$ eine $n+1$ mal stetig differenzierbare funktion und $x_0,x\in (a,b)$ dann gilt\[f(x)=P_{x_0,n}^f(x)+R_{x_0,n}^f(x)\]
Wobei diese $R_{x_0,n}^f(x)$ der restglied ist und ich den mit \[R_{x_0,n}^f(x)=\int_{x_0}^xf^{(x+1)}(t)\frac{(x-t)^n}{n!}\dd{t}\]
\end{document}
