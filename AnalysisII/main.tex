\documentclass{article}
\usepackage{amsfonts}
\usepackage{amsmath}
\usepackage{geometry}

\geometry{margin=1.5cm}

\author{Benjamin Dropmann}
\title{Analysis II}
\newcommand{\mspc}{\hspace{0.3cm}}

\begin{document}
\maketitle

\section{Metrische Räume}
\textbf{Skalarproduct} Seien zwei vektoren $x,y\in\mathbb{R}^n$ dann ist der skalaproduktwie Folgt definiert: \[x\cdot y=<x,y>=\sum_{i=1}^nx_iy_i\]
\textbf{Eufklidische Norm} Sei $x\in \mathbb{R}^n$ dann ist die Euklidische norm des Vektors \[||x||=\sqrt{\sum_{i=1}^nx_i^2}\]
\textbf{Euklidischer Abstand  } \[d(x,y)=\sqrt{\sum^n_{i=1}(x_i-y_i)^2}\]
\textbf{Dreiecksungleichung  }$\forall x,y,zin\mathbb{R}^n\rightarrow||x-z||\le||x-y||+||y-z||$
\textbf{Metrische Räume} Ein metrische (M.R) $(X,d)$ ist eine nicht-leere menge $X$ zusammen mit einer funktion $d:X\times X\rightarrow[0;\infty[$ Welche die folgenden Eigenschaften besitzt:
\begin{itemize}
\item[1.]{Positiv definiert $\forall x,y\in X\mspc d(x,y)=0\Leftrightarrow x=y$}
\item[2.]{Symmetrie $\forall x,y \in X\mspc d(x,y)=d(y,x)$}
\item[3.]{Dreiecksungleichung $\forall x,y,z\in X\mspc d(x,z)\le d(x,y)+d(y,z)$}
\end{itemize}
\textbf{Folgen} Sei $X$ eine Menge dann ist $(a_n)_{n\in\mathbb{N}_0}$ eine Folge $\mathbb{N}_0\rightarrow X$ mit dem bild $a_n=a(n)$.\newline
\textbf{Konvergenz einer Folge in X} $(dim(x\ge1)$ Sei $(X,d)$ ein M.R. und $(a_n)_{n\in\mathbb{N}_0}$ eine Folge.  Dann konvergiert die folge auf eine zahl $A\in X$ falls.
\[\forall\varepsilon>0\mspc\exists N>0:\mspc d(a_n,A)<\epsilon\]
Falls es kein $A$ gibt dann divergiert die Folge.\newline
\textbf{Teilfolgen} Sei ein M.R. $(X,d)$ und $(a_n)_{n\in\mathbb{N}_0}$ eine folge, dann existier eine teilfolge $(x_{n_k})_{n\in\mathbb{N}_0}$ wobei $(n_k)_{k\in\mathbb{N}_0}$ eine Folge von reellen Zahlen ist.\newline
\textbf{Häufungspunkt} Sei $(X,d)$ ein M.R. und $(a_n)_{n\in\mathbb{N}_0}$ eine Folge. $A\in X$ ist ein Häufungspunkt der Folge falls es eine Teilfolge die auf $A$ Konvergiert.\newline
\textbf{Satz} $y\subset X$ Teilmenge eines M.R. $(X,d)$ $x\in X$ ist Häufungspunkt von $y$ falls eine Folge $(y_n)_{n\in\mathbb{N}_0}$ existiert welche gegen $x$ konvergiert.
\newline\textbf{Lemma} Es sei $(x_n)_{n\in\mathbb{N}_0}$ eine Folge im Metrischen Raum $(X,d)$ mit $x\in X$ Dann konvergiert $(x_n)_{n\in\mathbb{N}_0}$ genau dann wenn jede Teilfolge $(x_{n_k})_{k\in\mathbb{N}_0}$ eine Teilfolge $(x_{n_{m_k}})_{k\in\mathbb{N}_0}$ die gegen $x$ konvergiert.
\newline\textbf{Lemma}Eine Folfe in $(\mathbb{R}^n, d(x,y)=||x-y||)$ Konvergiert genau dann wenn sie koordinaten weise konvergiert.
\subsection{Cauchy Folge}Eine Folge $(a_n)_{n\in\mathbb{N}_0}$ in einem Metrischen Raum $(X,d)$ heisst cauchy foglge falls:\[\forall \varepsilon >0 \mspc \exists N>0 \text{ So dass } \forall m,n >N: \mspc d(a_n,a_m)<\varepsilon\]
\textbf{Lemma} Analog zur Folgen in $\mathbb{R}$ gilt:
\begin{itemize}
\item{Jede Cauchy-Folge ist beschränkt,$\Leftrightarrow \exists K\in\mathbb{R}$ so dass $ d(a_n,a_0\le K \forall n\in \mathbb{N}_0$}
\item{Jede Konvergente Folge ist eine Cauchy-Folge.}
\item{Eine Cauchy Folge konvergiert genau dann wenn sie eine konvergente Teilfolge besitzt}
\end{itemize}
\textbf{Vollstandigkeit} Eine Metrischer Raum heisst vollständig falls jede Cauchy-Folge in $X$ Konvergiert.
\newline\textbf{Theorem} Für alle $n\ge1$ ist $\mathbb{R}^n$ mit der Standard metrik ist Vollständig.\newline
\textbf{Beweis} Analog zur tatsache, das in $\mathbb{R}^n$ konvergenz im Metrischen Raum äquivalent ist zur Koordinaten-weise konvergenz: Eine Cauchy-Folge in $\mathbb{R}^n$ zu sein ist äquivalent zur Tatsache dass jede Koordinate eine Cauchy-Folge liefert (in den Reelen zahlen).\newline
In den Reellen Zahlen konvergieren alle Cauchy-Folgen, Daher muss die Behauptung stimmen. Q.E.D.
\subsection{Topologie Metrischer Raume}
Es sei $(X,d)$ ein Metrischer Raum, $x\in X$ und $r>0$ eine reelle Zahl. Der offene Ball um den Punkt $x$ mit radius $r$ ist die Menge:\[B_r(x)=B(x,r:=\left[y\in X|\mspc d(x,y)<r\right]\]
\subsection{Innere/Abschulss-mengen und Ränder}
Sei $(X,d)$ ein Metrischer Raum und $A\subset X$ eine Teilmenge
\begin{itemize}
\item{das innere der Menge $A$ ist gegebe durch \[A^o=int(A):=\bigcup\left[E\subset A|\mspc E \text{ ist offen}\right]\] Und ist die grösste offene Menge, welche in $A$ enthalten ist.}
\item{Der Abschluss von A:\[\overline{A}
:=\bigcap \left[A\subset U|U \text{ ist abgeschlossen}\right]\] und ist die kleinste abgeschlossene Menge welche $A$ enthält.}
\item{Dr Tipologische Rabd von $A$ ist $\overline{A}/A^o$}
\end{itemize}
Beispiel, in $\mathbb{R}$ mit $A=]0,1[\rightarrow A^o=]0,1[$ und $\overline{A}=[0,1]$ dann ist der Rand: $\left\lbrace 0\right\rbrace\cup\left\lbrace1\right\rbrace$.\newline
\textbf{Proposition} Es sei $(X,d)$ ein Metrischer Raum:
\begin{itemize}
\item{Eine Teilmenge $A\subset X$ ist genau dann offen, alls für jede konvegente Folge $(x_n)_{n\in\mathbb{N}_0}$ mit grenzwert $x\in A$ gilt:\[\exists N\in\mathbb, \forall n>N \mspc x:n\in A\]}
\item{Eine Teilmenge $A\subset X$ ist genau dann abgeschlossen falls fur jede konvergente Folge $(x_n)_{n\in\mathbb{N}_0}\subset A$ mit grenzwert $x\in A$ ist $x\in A$}
\end{itemize}
\textbf{Beweis} Fall der Offene Menge:\newline
"$\Rightarrow$" Es sei $(x_n)_{n\in\mathbb{N}_0 }$ eine Konvergente Folge mit grenzwert $x\in A$. Gemäss vorassetzung wissen wir, dass die Teilmenge $A$ offen ist Da $A$ offen ist, und $x\in A$ gibt es ein Offenen Ball $B(x,\varepsilon)\subset A$. Für dieses $\varepsilon>0 \exists N\in\mathbb{N}_0$ (wegen der Konvergenz der betrachteten Folge) $d(x,x_n)<\varepsilon$ (Konvergente folgen sind in $X$ Cauchy-Folgen) dies Bedeutet das FOlgeglieder 
mit index $n>N$ in $B(x,\varepsilon)$\newline
"$\Leftarrow$" Wir nehmen jetzt an das $A\subset X$ nicht offen ist. Dies bedeutet dass $\exists x\in A \varepsilon>0 \mspc B(x,\varepsilon)/A\neq\emptyset$
Insbesonde konnen wir $\varepsilon=2^{-1}$ betrachten und eine Folge $(x_n)_{n\in\mathbb{N}}$ konstruiren mit $x_n\in B(x,2^{2{-n}})/A$ Es gilt für diese Folge aber auch dass Folgende $x_n\rightarrow x\in A$ \newline
Der Fall der Geschlossene Menge:\newline
"$\Rightarrow$" Wir nehmen an dass $A$ abgeschlossen ist. Wir betrachten dann eine beliebige FOlge $(x_n)_{n\in\mathbb{N}}$ mit $x_n\rightarrow x\in X$ Da $V:=X/A=A^c$ offen ist, kann der Grenzwert $x$ nicht in dieser offennen Menge liegen, da sonst die Folgenglieder ab einen bestimmten Index ebenfalls in dieser Offenene Menge liegen müssen, damit muss gelten dass:$x\in A$\newline
"$\Leftarrow$" Wir nehmen an dass $A$ nicht abgeschlossen ist, dann ist $A^c$ nicht offen Dies bedeutet dass: $\exists y\in A^c \forall \varepsilon>0 \mspc B(y,\varepsilon) \cap A \neq \emptyset$ Damit kann man eine Folge konstruiren 
$(x_n)_{n\in\mathbb{N}}$ mit $d(y,x_n)<\varepsilon \mspc \Rightarrow\mspc x\rightarrow y\in A^c$ Dann ist de beweis fertig Dann ist de beweis fertig.\newline
\noindent
\textbf{Proposition} Es sei $(X,d)$ ein Metrischer Raum. Eine folge $(x_n)_{n\in\mathbb{N}_0}$ konvergiert genau dann gegen $x$ wenn alle offene Mengen $U$ gilt:\[\exists N\in \mathbb{N}\forall n>N\mspc x_n\in U\]\newline
\textbf{Beweis}
"$\Rightarrow$" Sei $(x_n)_{n\in\mathbb{N}}$ eine gegen $x$ konvergierende Folge, es sei ausserdem $U$ offen mit $x\in U$. Da $U$ offen ist: $\exists \varepsilon >0 B(x,\varepsilon)$ Dann gilt auch \[\forall \varepsilon \exists N\in\mathbb{N} \forall n>N \mspc d(x,x_n)<\varepsilon\]
\newline
"$\Leftarrow$" $\forall \epsilon \mspc B(x, \epsilon)$ ist offen. Und es gilt $\exists N\in\mathbb{N} : x_n\in\forall n>N B(x, \varepsilon)$ Dies bedeutet gemäss definition dass $\underset{n\rightarrow\infty}{lim} x_n=x$. \newline
\textbf{Korollar} Es sei $X$ eine Menge und $d_1, d_2$ zwei verschieden Metriken. Dann haben $(X,d_1), (X, d_2)$ genau dann die selben konvergente Folgen wenn die Topologien von $d_1$ und $d_2$ ubereinstimmen. 
\newline\subsection{Banachscher Fixpunkttheorem}
Sei $(X,d)$ Ein Vollständiger Metrischer raum mit eine Abbildung $f:X\rightarrow X$ die Lipschitz stetig ist mit $L<1$\[\forall x,x'\in X\mspc d(f(x),f(x'))\le Ld(x,x')\]
Dann gibt es ein wert $z\in X$ wofür $f(z)=z$
\newline
\textbf{Beweis der Existenz des Fixpunkts} Wir nehmen $x\in x_0$ beliebig, dann konstruiren wir iterativ eine Folge in $X$ und zwar wie folgt: $x_{n+1}:=f(x_n)$ Dies ergibt tatsächlich eine Folge $(x_n)n\in\mathbb{N}_0$ Als nächstes wollen wir Zeigen dass diese Folge eine Cauchy-Folge ist.
\begin{itemize}
  \item{$d(x_{n+1},x_n)=d(f(x_n),f(x_{n-1}))=d(f^n(x_0),f^n(x_0))\le L^nd(x_1,x_0)$}
  \item{$d(x_m,x_n)\le \sum_{k=n}^{m-1}d(x_{k+1}^,x_k)\le\sum_{k=n}^{m-1}L^kd(x_1,x_0)=d(x_1,x_0)\sum_{k=n}^{m-1}l^k$ Und diese Reihe ist für $L<1$ $\forall n,m\in \mathbb{N}_0$ und sogar $m\rightarrow\infty$ konvergent.
  Und wir finden:\[\forall m,n\in\mathbb{N}_0, d(x_n,x_m)\le d(x_1,x_0)\underset{\text{für }n\rightarrow\infty,\mspc \rightarrow0}{\underbrace{\frac{L^n}{1-L}}} \]Und damit ist unsere Folge eine Cauchy-Folge }
\end{itemize}
Es bleibt noch zu zeigen dass $\overline{x}$ ein Fixpunkt ist:
\[f(\overline{x})=\lim_{}\]
\subsection{Kompaktheit} Ein Intervall $I\subset\mathbb{R}$ ist kompakt genau dann wenn $I$ beschränkt und abgeschlossen ist. Die beschränktheit geht aber nicht trivial in den Metrischen Raum über.
\newline\textbf{Definition} Sei $(X,d)$ ein Metrischer Raum, und $K$ Eine Teilmenge.\begin{itemize}
  \item{K heisst Folgenkompakt falls jede Folge $(x_n)_{n\in\mathbb{N}_0}\subset K$ eine in $K$ konvergente Teilfolge.}
\item{K heisst Topologisch Kompakt falls jede Familie von Offenen Mengen $\mathbb{U}:=\left\lbrace\mathbb{U}_i\right\rbrace_i\in I$, welche $K$ uberdeckt, also dass:\[K\subset\bigcup\mathbb{U}=\bigcup_{i\in I}U_i\] eine Endliche Teilüberdeckung besitzt, (eine endliche familie welche $K$ immer noch überdeckt.)}
\end{itemize}
\textbf{Theorem} 
\end{document}

